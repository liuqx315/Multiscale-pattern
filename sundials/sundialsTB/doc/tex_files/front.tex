\documentclass[titlepage,10pt]{article}

\usepackage{alltt}
\usepackage{color}
\usepackage{listings}

\definecolor{string}{rgb}{0.7,0.0,0.0}
\definecolor{comment}{rgb}{0.13,0.54,0.13}
\definecolor{keyword}{rgb}{0.0,0.0,1.0}


\lstset{
  language=Matlab, 
  frame=leftline, 
  stringstyle=\color{string},
  commentstyle=\color{comment},
  keywordstyle=\color{keyword},
  numbers=left, 
  numberstyle=\tiny, 
  numbersep=10pt
}


\usepackage{calc, chngpage}
\usepackage[nottoc]{tocbibind}
\usepackage{rotating}
\usepackage{multirow}
\usepackage{tabls}
\usepackage{epsfig}
\usepackage{makeidx}


%% Finally, use hyperref package to include links in the PDF
%% Note that since hyperref redefines a bunch of LaTeX commands,
%% to give it a fighting chance, it MUST be included last.

\usepackage[
letterpaper=true, 
dvips, 
ps2pdf, 
hyperindex=true, 
linktocpage=true,
colorlinks=true, 
linkcolor=blue,
citecolor=blue,
bookmarks=true]
{hyperref}


%----- Page formatting

%% \setlength{\oddsidemargin}{0.5in}
%% \setlength{\evensidemargin}{0.0in}
%% \setlength{\paperwidth}{8.5in}
%% \setlength{\hoffset}{0in}
%% \setlength{\textwidth}{\paperwidth-(1in+\hoffset)*2-\oddsidemargin-\evensidemargin}
%% \setlength{\textheight}{9.0in}


\setlength{\paperwidth}{8.5in}
\setlength{\textwidth}{6.1in}
\setlength{\hoffset}{0.0in}
\setlength{\oddsidemargin}{0.2in}
\setlength{\evensidemargin}{0.2in}
\setlength{\marginparsep}{0.25in}
\setlength{\marginparwidth}{0.5in}

\setlength{\paperheight}{11.0in}
\setlength{\textheight}{9.0in}
\setlength{\voffset}{-0.5in}
\setlength{\topmargin}{0.0in}


%---- Tables
\setlength{\extrarulesep}{2pt}

%----- UCRL number, date, versions
\newcommand{\STBucrl}{UCRL-SM-212121}
\newcommand{\STBrelease}{v2.4.0}

%----- LLNL disclaimer
\newcommand{\disclaimer}{%
\changetext{.625in}{}{}{}{}
\thispagestyle{empty}% no number of this page
\vglue5\baselineskip
\begin{center}
{\bf DISCLAIMER}
\end{center}
\noindent
This document was prepared as an account of work sponsored by an agency of 
the United States government. Neither the United States government nor 
Lawrence Livermore National Security, LLC, nor any of their employees makes 
any warranty, expressed or implied, or assumes any legal liability or responsibility 
for the accuracy, completeness, or usefulness of any information, apparatus, product, 
or process disclosed, or represents that its use would not infringe privately owned rights. 
Reference herein to any specific commercial product, process, or service by trade name, 
trademark, manufacturer, or otherwise does not necessarily constitute or imply its endorsement, 
recommendation, or favoring by the United States government or Lawrence Livermore National 
Security, LLC. The views and opinions of authors expressed herein do not necessarily state
or reflect those of the United States government or Lawrence Livermore National Security, LLC, 
and shall not be used for advertising or product endorsement purposes.

\vskip2\baselineskip
\noindent
This work was performed under the auspices of the U.S. Department of Energy by Lawrence 
Livermore National Laboratory under Contract DE-AC52-07NA27344.
\vfill
\begin{center}
Approved for public release; further dissemination unlimited
\end{center}
\clearpage
\changetext{-.625in}{}{}{}{}
}

%----- Clear empty double page
\newcommand{\clearemptydoublepage}{\newpage{\pagestyle{empty}\cleardoublepage}}

%----- SUNDIALS MODULES
\newcommand{\sundialsTB}{{\normalfont\scshape sundialsTB}}
\newcommand{\sundials}{{\normalfont\scshape sundials}}
\newcommand{\nvector}{{\normalfont\scshape nvector}}
\newcommand{\putils}{{\normalfont\scshape putils}}
\newcommand{\cvode}{{\normalfont\scshape cvode}}
\newcommand{\cvodes}{{\normalfont\scshape cvodes}}
\newcommand{\ida}{{\normalfont\scshape ida}}
\newcommand{\idas}{{\normalfont\scshape idas}}
\newcommand{\kinsol}{{\normalfont\scshape kinsol}}

\newcommand{\matlab}{{\normalfont\scshape matlab}}
\newcommand{\mpiTB}{{\normalfont\scshape mpiTB}}

%----- title and author
\title{{\sundialsTB} {\STBrelease}, a {\matlab} Interface to {\sundials}}
\author{
  Radu Serban \\ 
  {\em Center for Applied Scientific Computing} \\ 
  {\em Lawrence Livermore National Laboratory}
}
\date{
  \today
  \vfill 
  {\centerline{\psfig{figure=doc_logo.eps,width=0.5\textwidth}}}
  \vfill \STBucrl}

%------ BEGIN document

\makeindex

\begin{document}

\pagestyle{empty}
\maketitle
\disclaimer

\tableofcontents

\clearemptydoublepage

\pagestyle{plain}\pagenumbering{arabic}

\section{Introduction}

{\sundials}~\cite{HBGLSSW:04}, SUite of Nonlinear and DIfferential/ALgebraic equation Solvers,
is a family of software tools for integration of ODE and DAE initial value problems
and for the solution of nonlinear systems of equations.
It consists of {\cvode}, {\ida}, and {\kinsol}, and variants of these with 
sensitivity analysis capabilities.

{\sundialsTB} is a collection of {\matlab} functions which provide interfaces to
the {\sundials} solvers.

The core of each {\matlab} interface in {\sundialsTB} is a single {\sc mex} 
file which interfaces to the various user-callable functions for that solver.
However, this {\sc mex} file should not be called directly, but rather through the 
user-callable functions provided for each {\matlab} interface.

A major design principle for {\sundialsTB}
was to provide an interface that is, as much as possible, equally familiar to
both {\sundials} users and {\matlab} users. Moreover, we tried to keep the
number of user-callable functions to a minimum. For example, the {\cvodes} {\matlab} 
interface contains only 12 such functions, 2 of which relate to forward sensitivity analysis and
4 more interface solely to the adjoint sensitivity module in {\cvodes}. 
A user who is only interested in integration of ODEs and not in sensitivity analysis
therefore needs to call at most 6 functions.
In tune with the {\matlab} {\sc odeset} function, optional
solver inputs in {\sundialsTB} are specified through a single function; e.g.
{\tt CvodeSetOptions} for {\cvodes} (a similar function is used to specify optional
inputs for forward sensitivity analysis). However, unlike the ODE solvers in {\matlab}, we
have kept the more flexible {\sundials} model in which a separate ``solve'' function 
({\tt CVode} for {\cvodes}) must be called to return the solution at a desired 
output time. Solver statistics, as well as optional outputs (such as
solution and solution derivatives at additional times) can be obtained at any time
with calls to separate functions ({\tt CVodeGetStats} and {\tt CVodeGet} for {\cvodes}).

This document provides a complete documentation for the {\sundialsTB} functions.
For additional details on the methods and underlying {\sundials} software consult
also the coresponding {\sundials} user guides~\cite{cvodes_ug,idas_ug,kinsol_ug}.

\vspace{0.15in}
\noindent{\bf Requirements.}
For parallel support, {\sundialsTB} depends on {\mpiTB} with {\sc lam} v $> 7.1.1$ (for MPI-2 spawning feature).
The required software packages can be obtained from the following addresses.
%
\begin{center}
\begin{tabular}{rl}
{\sundials} & {\tt http://www.llnl.gov/CASC/sundials} \\
{\mpiTB}    & {\tt http://atc.ugr.es/javier-bin/mpitb\_eng}\\
{\sc lam}   & {\tt http://www.lam-mpi.org/}
\end{tabular}
\end{center}

\section{Installation} 

The following steps are required to install and setup {\sundialsTB}:

\subsection{Compilation and installation of sundialsTB}

As of version 2.3.0, {\sundialsTB} is distributed only with the complete {\sundials} package.

In the sequel, we assume that the {\sundials} package was unpacked under the directory {\em srcdir}.
The {\sundialsTB} files are therefore in {\em srcdir}{\tt /sundialsTB}. 

Compilation and installation of the {\sundialsTB} toolbox is done by running the {\matlab} script 
{\tt install\_STB.m} which is present in the {\sundialsTB} top directory.

\begin{enumerate}

\item Launch {\matlab} in sundialsTB
\begin{verbatim}
    % cd srcdir/sundialsTB
    % matlab
\end{verbatim}

\item Run the {\matlab} script install\_STB 

     Note that parallel support will be compiled into the MEX files only if
     \$LAMHOME is defined {\bf and}
     \$MPITB\_ROOT is defined {\bf and}
     {\em srcdir}/src/nvec\_par exists.

     After the MEX files are generated, you will be asked if you wish to install 
     the {\sundialsTB} toolbox. If you answer yes, you will be then asked for the
     installation directory (called in the sequel {\em instdir}). 
     To install {\sundialsTB} for all {\matlab} users (not usual), assuming {\matlab} is 
     installed under {\tt /usr/local/matlab7}, specify
        {\em instdir} = {\tt /usr/local/matlab7/toolbox}.
     To install {\sundialsTB} for just one user (usual configuration), install      
     {\sundialsTB} under a directory of your choice (typically under your {\tt matlab}
     working directory). In other words, specify
        {\em instdir} = {\tt /home/user/matlab}.

\end{enumerate}

\subsection{Configuring Matlab's startup}

After a successful installation, a {\sundialsTB}.m startup script is generated
in {\em instdir}{\tt /sundialsTB}. This file must be called by {\matlab} at initialization.

If {\sundialsTB} was installed for all {\matlab} users (not usual), add the {\sundialsTB} 
startup to the system-wide startup file (by linking or copying):

\begin{verbatim}
     % cd /usr/local/matlab7/toolbox/local
     % ln -s ../sundialsTB/startup_STB.m .
\end{verbatim}

and add these lines to your original local startup.m

\begin{verbatim}
     % SUNDIALS Toolbox startup M-file, if it exists.
     if exist('startup_STB','file')
        startup_STB
     end
\end{verbatim}

If {\sundialsTB} was installed for just one user (usual configuration) and
assuming you do not need to keep any previously existing startup.m, link 
or copy the startup\_STB.m script to your working 'matlab' directory:

\begin{verbatim}
     % cd ~/matlab
     % ln -s sundialsTB/startup_STB.m startup.m
\end{verbatim}

If you already have a startup.m, use the method described above, first linking 
(or copying) {\tt startup\_STB.m} to the destination subdirectory and then editing 
the file {\tt ~/matlab/startup.m} to run {\tt startup\_STB.m}.

\subsection{Testing the installation}

If everything went fine, you should now be able to try one of the {\cvodes}, {\idas},
or {\kinsol} examples (in {\matlab}, type 'help cvodes', 'help idas', or 'help kinsol' 
to see a list of all examples available). For example, go to the {\cvodes} serial 
example directory:
\begin{verbatim}
     % cd instdir/sundialsTB/cvode/examples_ser
\end{verbatim}
and then launch {\matlab} and execute mcvsRoberts\_dns.

