\section{Code Organization}\label{s:organization}

As mentioned before, the SUNDIALS suite consists of the basic solvers
CVODE (for ODE systems), KINSOL (for nonlinear algebraic
systems), and IDA (for DAE systems) and of sensitivity-capable variants,
CVODES, IDAS, and KINSOLS (the last two being currently under development).
%
The overall organization of the CVODES package, as well as its relationship
to SUNDIALS, is shown in Fig.~\ref{f:cvsorg}.  
The basic elements of the CVODES structure are a module for
the basic integration algorithm (including forward sensitivity analysis),
a module for adjoint sensitivity analysis, and a set of modules for the solution
of linear systems that arise in the case of a stiff system.  
\begin{figure}
\centerline{\psfig{figure=cvsorg.eps,width=\textwidth}}
\caption {Overall structure diagram of the CVODES package.
  Modules specific to CVODES are distinguished by rounded boxes, while 
  generic solver and auxiliary modules are in square boxes.}
\label{f:cvsorg}
\end{figure}
Modules which are shared across the entire SUNDIALS suite include
generic linear system solvers, and the NVECTOR modules (described
further below).

The central CVODES integration module deals with the evaluation of
integration coefficients, the functional or Newton iteration process,
estimation of local error, selection of stepsize and order, and
interpolation to user output points, among other issues.  Although
this module contains logic for the basic Newton iteration algorithm,
it has no knowledge of the method being used to solve the linear
systems that arise.  For any given user problem, one of the linear
system modules is specified and is then invoked as needed during the
integration.

In addition, if forward sensitivity analysis is turned on, the main
module will integrate the forward sensitivity equations,
simultaneously with the original IVP.  The sensitivities variables may
or may not be included in the local error control mechanism of the
main integrator.  CVODES provides three different strategies of
dealing with the correction stage for the sensitivity variables,
simultaneous corrector and two variants of staggered corrector (see
Section~\ref{ss:fwd_sensitivity}).  The CVODES package includes an
algorithm for the approximation of the sensitivity equations
right-hand sides by difference quotients, but the user has the option
of supplying these right-hand sides directly.

The adjoint sensitivity module provides the infrastructure needed for
the integration backwards in time of any system of ODEs which depends
on the solution of the original IVP, in particular the adjoint system
and any quadratures required in evaluating the gradient of the
objective functional.  This module deals with the set-up of the check
points, interpolation of the forward solution during the backward
integration, and backward integration of the adjoint equations.

At present, the CVODES package includes four linear system solution
modules, of which three use direct methods, and one uses scaled
preconditioned GMRES, a Krylov subpsace method.  For the latter, two
preconditioner modules are also included, one for use on serial
computers, and one for parallel.  All of these are virtually identical
to the corresponding modules for CVODE, which are described in detail
in ~\cite{HBGLSSW:04}.  In addition, the user of CVODES may supply
his/her own linear solver module, following specifications given in
the user documentation.  Thus an existing linear system solver can be
incorporated by providing short interface functions between CVODES
and the linear system solver.

All state information used by CVODES to solve a given problem is saved
in a structure, and a pointer to that structure is returned to the
user.  There is no global data in the CVODES package, and so in this
respect it is reentrant. State information specific to the linear
solver is saved in a separate structure, a pointer to which resides in
the CVODES memory structure. The reentrancy of CVODES was motivated
by the anticipated multicomputer extension but is also essential
during adjoint sensitivity analysis where the check-pointing algorithm
leads to interleaved forward and backward integration passes. 

Figure~\ref{f:cvsorg} does not show any of the user-supplied functions
for CVODES. At a minimum, the user must provide a function for the
evaluation of the ODE right-hand side and, if performing adjoint
sensitivity analysis, a function for the evaluation of the right-hand
side of the adjoint system.  Optional user-provided functions include,
depending on the options chosen, functions for Jacobian evaluation
(direct cases) or Jacobian-vector products (Krylov case), setup and
solution of Krylov preconditioners, a function providing the integrand
of any additional quadrature equations, and a function for providing
the right-hand side of the sensitivity equations (for forward
sensitivity analysis). Depending on the options selected for the
solution of the adjoint system, the user may have to provide
corresponding Jacobian and/or preconditioner functions.

One of the most important characteristics of the design of CVODES 
(shared by all solvers across SUNDIALS) is the fact that it is implemented 
in a data-independent manner, in that the solver does not need any information
regarding the underlying structure of the data on which it operates.

The CVODES solver acts on vectors through a generic NVECTOR module,
which defines an NVECTOR structure specification, a data-independent
NVECTOR type, a set of abstract vector operations, and a set of
wrappers for accessing the actual vector operations of the
implementation under which an NVECTOR was created. Because details of
vector operations are thus encapsulated within each specific NVECTOR
implementation, CVODES is thus independent of a specific
implementation. This allows the solver to be precompiled as a binary
library and allows more than one NVECTOR implementation to be used
within a single program. This feature is essential for the efficient
integration of quadrature variables (see Section~\ref{ss:integration})
as well as for adjoint sensitivity analysis when, for some problems,
the adjoint variables are more conveniently organized in a structure
different from that of the variables in the forward problem.

A particular NVECTOR implementation, such as the serial and parallel 
implementations included with SUNDIALS or a user-provided implementation,
must provide the following:
(1) actual implementation of the functions for operations on N-vectors, 
such as creation, destruction, summation, and dot product;
(2) a function to construct an NVECTOR specification structure
for this particular implementation, which defines the data necessary
for constructing a new N-vector and attaches the vector operations
to the new structure; and
(3) a destructor for the NVECTOR specification structure.

