\documentstyle[12pt]{letter}
\batchmode

\setlength{\textheight}{9.0in}
\setlength{\topmargin}{-0.5in}

\begin{document}
\begin{letter}


\reference{Manuscript TOMS-2003-0041}

\opening{Dear Dr. Boisvert,}

We are hereby submitting a revised version of our paper, ``CVODES: An
ODE Solver with Sensitivity Analysis Capabilities.''

Attached below, we have reproduced the referees' comments, with our
responses inserted.

An overriding theme in the referee comments was that the paper should
be shortened by cutting material that is not new.  We have done that,
and the paper is now one full page shorter than the original version.
This is been accomplished even with the addition of a section (called
for by two of the referees) giving an example problem and its solution
by CVODES.

Ron, one of the comments you made was: ``With regard to being part of
the special issue on ACTS, it would be good to describe the role
CVODES plays in the overall collection, and how the collection has
influenced the evolution of CVODES.''  In response, we have revised
parts of the Introduction and Code Organization sections to describe
further the interconnections between CVODES and the SUNDIALS collection.


\closing{Sincerely,}

Radu Serban and Alan C. Hindmarsh\\
Center for Applied Scientific Computing\\
Lawrence Livermore National Laboratory

\end{letter}


\newpage
\section{Answers to Referee 1 Comments}

This paper describes an ODE solver with sensitivity analysis 
capabilities. Sensitivity analysis is an extremely important
subject that can be applied in a number of situations such as
optimum design, experimental design, uncertainty analysis etc. 

This paper is very well written and can be decomposed into two
main parts: algorithms and code description. It is difficult to
find something new in the algorithm part. The authors should
give proper citations on the things they have used. For example, on
page 4, the quadrature computation has been proposed and
documented in DASPK3.0 package (see Li and Petzold (2000 and
2001), Cao et al (2002)). 

{\em Response:}
We have greatly shortened the Algorithms section, using references to
our SUNDIALS paper, and we have added references to DASPK3 and
DASPKADJOINT there.

The evaluation of the sensitivity right-hand side is the key
part for users who want to use the sensitivity analysis
capabilities. I suggest the authors describe and compare all the
approaches, not only the finite-difference approaches, and give
appropriate citations. 

{\em Response:}
The mechanism provided in CVODES for user-defined routines
for the evaluation of derivative information (Jacobian-related data
and/or sensitivity right-hand sides) can be used to wrap AD-generated
routines.
We have included a paragraph discussing our own efforts 
in developing complex-step derivative approximations through automatic 
code generation and a short discussion on the challenges in 
designing general interfaces to automatic differentiation tools
for C codes.

In section 2.3, the authors ignore that the check-pointing
scheme has also been used in other adjoint packages (e.g.,
Giering, TAMC) and DAE solver DASPKADJOINT code (see Cao et
al.(2002), Li and Petzold(2001)). 

{\em Response:} 
we have changed this paragraph to mention that we are 
using a checkpointing scheme similar to the one in DASPKADJOINT
and a justification why truly optimal checkpointing (as described
by Griewank and Walther) cannot be employed.

The authors do not mention how to evaluate the adjoint sensitivity
right-hand side (15) or (17) in the paper. This is the key for the
users.  I find that in the code, the analytic input by hand is used,
which is impossible for many of users and applications.

{\em Response:}
The user interface in CVODES provides all the software hooks for 
implementing interfaces to routines that evaluate the adjoint systems via
reverse automatic differentiation. 
Current AD tools for C codes are not mature enough (there are fundamental 
difficulties due to the much higher flexibility of C vs. F77) to allow for 
the definition of generic wrappers in CVODES at this time.

The last sentence about the adjoint PDE on page 10 is wrong. 
You cannot replace the adjoint of the discretized PDE with the 
discretized adjoint PDE system directly without modifying the 
sensitivity evaluation formula (14). The boundary conditions, 
which is absent in ODE solver and formula (14), must be incorporated 
(see Li and Petzold (2003) about adjoint PDE). 

{\em Response:}
We are aware that these two formulations are different, and this is
what we were trying to say. Exactly because of this, we designed
CVODES such that it does not assume one formulation or the other.  We
have now explicitly stated that the two formulations are different
(with two citations).

The code description part is quite clear. I would like to see
the following points to be addressed. 

1) Since there are many FORTRAN ODE/DAE codes available, the authors need to 
state explicitly what the advantages are to rewrite the code to C or C++
language. 

{\em Response:}
We have included a paragraph in the introduction that lists
some of the advantages of using C: portability, availability, standard
dynamic memory allocation and data structures, etc.

2) Many numerical users are still using Fortran in writing their right-hand 
side. The authors need to give a clear and easy-used interface between the 
Fortran user and CVODES package. 

{\em Response:}
In the introduction, we point to the Fortran-C interface provided
with CVODE as an example of using a SUNDIALS solver with Fortran user code.
Moreover, as mentioned in the Conclusions section, we are planning on
providing interlanguage operability for all of the SUNDIALS codes by using
Babel technology.

3) The authors need to describe how to incorporate efficiently the existing 
nonlinear or linear solvers. For example, many PDE packages now use the sparse 
data structure and multi-grid preconditioners. How to use the modern solver
techniques  in CVODES is an interesting field. The author need
to give some related examples in the code. 

{\em Response:}
We have added, under Code Organization, a note about how a user can
easily incorporate his/her own linear system solver into CVODES.  We
have not made such provisions for changing the nonlinear solver, because 
the nonlinear solution algorithm is tightly coupled to the ODE setting.

4) The authors do not state much about the parallelization. I would like to 
see more about that topic in the paper and numerical efficiency and speed-up
comparison with the serial code in the packages.

{\em Response:}
We have included a section on parallel speedup on a sensitivity problem.


\newpage
\section{Answers to Referee 2 Comments}

This paper describes a stiff and nonstiff ODE IVP solver with forward
and adjoint sensitivity analysis capabilities. The presented software,
which is a part of SUNDIALS software suite, is an updated version of
the VODE solver written in C and extended to sensitivity analysis
capabilities.  All the algorithms presented have been published
elsewhere by other workers, and cannot be seen as novel contributions
of the current paper. Therefore the first 10 pages of the paper can be
summarized with appropriate citations. The rest of the paper describes
the code organization and its usage briefly. There are no numerical
examples demonstrating any advantages of using this particular
software. There is no mention of existing forward/adjoint sensitivity
analysis tool(s) with virtually identical capabilities, i.e.,
DASPKADJOINT.

{\em Response:}
We have greatly reduced the size of the sections in the first 10
pages, giving short summaries and referring to our SUNDIALS paper for
details.  We have also added references to DASPK3 and DASPKADJOINT
there.

Considering the editorial charter which states that "..., papers that
are essentially user manuals for a computer program are also not
acceptable" and noting that most of the material presented in the
paper can be found in [1], a significant rewrite of the paper is
recommended emphasizing the advantages and novelties of the software
with specific examples. However, it is not clear to this reviewer that
there are sufficient advantages and novelties in the software to
satisfy the requirements of the editorial charter.

{\em Response:} 
Our revisions have the effect of minimizing the material that appears
elsewhere and emphasizing material that is new.

Detailed Comments

* p. 7 paragraph 3: while it is indeed an important observation that
the staggered corrector method combined with an iterative linear
solver is equivalent to the staggered direct method, this observation
has already been made in [2].

{\em Response:}
We have added the citation.

* p. 14 paragraph 3 from bottom of the page: SPGMR is not a direct
linear solver.

{\em Response:}
We have rephrased this sentence.

* p. 15 paragraph 2:"sensitvity" should be "sensitivity".

{\em Response:}
Fixed the typo.

* the Maly and Petzold reference is from 1996.

{\em Response:} Fixed biblio entry.

Experience with the Code

The code is user friendly and has obvious advantages inherited from
the usage of the C programming language. We did not undertake
performance comparisons.


\newpage
\section{Answers to Referee 3 Comments}

1 Summary of the contents

The paper describes the software tool CVODES as a multistep ODE solver
with comprehensive capabilities for sensitivity analysis. The ODE
integration itself is performed by a BDF method for stiff systems or
an Adams-Moulton formula for non-stiff systems. Arising linear systems
can be solved by either a dense direct method, a banded direct method
or an iterative Krylov subspace method.

The sensitivity analysis features of CVODES comprise forward
sensitivity analysis by the simultaneous corrector method as described
in [5], or two versions of the staggered corrector method as
introduced in [1]. The right-hand sides of the arising sensitivity
systems can be currently obtained through analytical user input,
finite differences or directional derivatives, and an automatic
differentiation option is envisaged in the conclusion section.

Furthermore, CVODES also allows to compute sensitivities via adjoint
sensitivity analysis by implementing a check-pointing strategy.

In Section 3, the paper describes details of the software setup and
code organization, as part of a larger family of solvers, the SUNDIALS
package.  Section 4 of the paper is concerned with giving a detailed
overview about the usage of the tool.

2 Comments on the contents and assessment of relevance for a
publication in TOMS

The features of the tool mentioned in this paper are apparently not
new. A BDF solver for DAE systems with dense direct, banded direct or
Krylov subspace based linear solvers is available as the
well-established code DDASPK [2]. DDASPK also contains the three
common methods for forward sensitivity analysis, i.e. the staggered
direct, the simultaneous corrector and the staggered corrector
method. The code DSL48S [1] contains a BDF solver for DAE systems with
the staggered corrector sensitivity method using the sparse direct
linear algebra solver MA48.

{\em Response:}
We have included appropriate citations.

The statement in Section 2.2.1 of the paper, that the staggered direct
method is always unattractive, is not correct. As shown in [4], it can
outperform the other methods in case that $N_s >> N$.

{\em Response:}
We have changed this paragraph to include the citation [4] and 
clarify that the staggered direct method is not attractive for problems that 
have many states (such as those arising from semidiscretization of PDES).
For the cases in which the number of parameters is very large (comparable
to the problem size) adjoint sensitivity is (generally) a better choice than 
any forward sensitivity method.

In contrast to CVODES, DDASPK already contains automatic
differentiation capabilities according to [2].

{\em Response:} 
DASPK3.0 and DASPKADJOINT do not contain AD capabilities, but rather provide 
generic interfaces to AD-generated routines.
The user interface in CVODES provides all the software hooks for 
implementing interfaces to AD-generated routines. 
Current AD tools for C codes are not mature enough (there are fundamental 
difficulties due to the much higher flexibility of C vs. F77) to allow for 
the definition of generic wrappers in CVODES at this time.

Regarding the adjoint approach, the concepts and implementation in
this paper appear to be close to what has been reported in [3]. The
check-pointing has already been described in that reference.

{\em Response:}
We have changed this paragraph to mention that we are 
using a checkpointing scheme similar to the one in DASPKADJOINT
and a justification why truly optimal checkpointing (as described
by Griewank and Walther) cannot be employed.

DDASPKADJOINT additionally contains optional automatic differentiation
facilities for obtaining the required derivative information.

{\em Response:} 
See above response regarding AD in DASPK3.0.

According to the paper, the key benefit of CVODES appears to be the
modular software implementation, which collects all subelements for
ODE integration and sensitivity evaluation under one common roof. The
modular approach allows an easy exchange of low-level tools such as
linear solvers etc.The editorial charter of TOMS states that "the
purpose of a TOMS communication is the presentation of the results of
novel research and development efforts in support of significant
mathematical computer application". Taking the findings above into
account, it is rarely possible to accept the paper as "novel
research". An acceptable paper should also contain "a demonstration of
superiority compared to alternative approaches", typically carried out
by an experimental evaluation of computer implementations.  The paper
at hand is completely lacking this issue. Basically it is a
description of a new software implementation of well-known numerical
concepts.

{\em Response:}
We have made revisions to minimize the material that appears elsewhere
and emphasize material that is new.


\end{document}