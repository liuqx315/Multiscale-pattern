\section{CVODES Usage}\label{s:usage}

In this section we give an overview of the usage of CVODES for ODE integration, 
forward sensitivity analysis, and adjoint sensitivity analysis. 
Complete documentation of the code usage is given in~\cite{HiSe:04cvodes}.

%------------------------------------------------------
% Integration and forward SA
%------------------------------------------------------
%
One of the guiding principles in designing the user interface to the 
CVODES solver has been to allow user to transit from just integration 
of ODEs to performing sensitivity analysis in as rapid and seamless a 
manner as possible. To achieve this goal, we have opted not to 
modify any of the CVODE user interface to account for the initialization
and set up of sensitivity analysis.

Instrumenting an existing user code for forward sensitivity analysis can 
thus be done by only inserting a few calls to CVODES functions, additional
to those required for setting up and solving the original ODE
(steps \ref{i:sensi_set}, \ref{i:sensi_malloc}, \ref{i:getsensi}, 
and \ref{i:sensi_get} below).
We give below the main steps required to set up, initialize, and solve an 
IVP ODE, and optionally perform forward sensitivity analysis with respect 
to some of the model parameters. 
This sequence of calls is the most natural one, but the order of some of 
the steps below can be changed. 
For example, initialization and allocation for forward sensitivity analysis 
could be performed before attaching and configuring the linear solver module.
Similarly, changing optional inputs to the solver (step \ref{i:set}) could
follow the memory allocation step \ref{i:malloc}.
%
\begin{enumerate}
\item \label{i:nvspec}
  An implementation dependent NVECTOR specification constructor must first
  be called. For the two NVECTOR implementation provided with SUNDIALS, serial
  and MPI parallel, the constructor functions are {\tt NV\_SpecInit\_Serial}
  and {\tt NV\_SpecInit\_Parallel}, respectively.
\item \label{i:create}
  {\tt CVodeCreate} 
  creates the solver object.
  The user must specify the linear multistep method to be used 
  (Adams or BDF) and the nonlinear iteration type (functional or Newton).
  Various options controlling the solver are set to their default values.
\item \label{i:set}
  {\tt CVodeSet*} 
  functions can now be used to change various controls from their default values.
  Choices and default values are given in Table~\ref{t:optional_inputs}.
\item \label{i:malloc}
  {\tt CVodeMalloc}
  must be called next to perform any required memory allocation, after checking 
  the initialized memory block for errors in the default or optional inputs. 
  At this step the user must specify the function providing the ODE right-hand 
  side, the initial time and initial values, as well as the desired
  integration tolerances.
\item \label{i:linear}
  {\tt CVDense}, {\tt CVBand}, {\tt CVDiag}, or {\tt CVSpgmr}.
  If Newton iteration was selected in step (\ref{i:create}), a linear solver  
  is needed for solving the linear systems that arise during the Newton iterations. 
  A linear solver object (dense, band, diagonal, or SPGMR) must now be created
  and attached to the block of memory allocated for the solver.
  Various options controlling the linear solver are set to their default values.
\item \label{i:ls_set}
  {\tt CVDenseSet*}, {\tt CVBandSet*}, or {\tt CVSpgmrSet*}.
  At this stage, the default values in the linear solver memory block can be
  changed if so desired. 
  Choices and default values are given in Table \ref{t:optional_inputs}.
\item \label{i:sensi_set}
  {\tt CVodeSetSens*}
  functions can be called to change from their default values the optional inputs
  that control the integration of the sensitivity systems 
  (see Table~\ref{t:optional_inputs}).
\item \label{i:sensi_malloc} 
  {\tt CVodeSensMalloc}
  must be called if solution sensitivities are desired.
  This function initializes and allocates memory for forward sensitivity calculations.
  At this stage the user specifies the number of sensitivities to be computed, 
  the forward sensitivity method 
  the model parameters, as well as the initial values for the sensitivity variables.
\item \label{i:solve}
  {\tt CVode} 
  solves the problem. 
  The solver function is typically called in a loop over the desired output times.
  The user can have the solver take internal steps until it has reached the
  user-specified $t_{\mbox{\scriptsize out}}$ or return control to the user's
  main program after taking one successful step. Additionally, the user can
  direct the solver to test $t_{\mbox{\scriptsize stop}}$ so that the
  integration never proceeds beyond this value.
\item \label{i:getsensi}
  {\tt CVodeGetSens}
  extracts the sensitivity solution vectors. If forward sensitivity analysis
  had been enabled in step \ref{i:sensi_malloc}, solutions sensitivities are
  computed at the same time as the ODE solution and are available to the user
  through this function.
\item \label{i:get}
  {\tt CVodeGet*}.
  Optional outputs and statistics for the main solver are available through 
  extraction functions. A complete list of the optional outputs from CVODES 
  is given in Table~\ref{t:optional_outputs}.
\item \label{i:sensi_get}
  {\tt CVodeGet*Sens*}.
  Optional outputs and statistics related to the solution of the sensitivity 
  systems are available through additional extraction functions 
  (see Table~\ref{t:optional_outputs}).
\item \label{i:ls_get}  
  {\tt CVDenseGet*}, {\tt CVBandGet*}, {\tt CVDiagGet*} or {\tt CVSpgmrGet*}.
  These functions provide optional statistics from the linear solver module.
\item \label{i:free}
  {\tt CVodeFree} and the vector specification destructor.
  To complete the process, the user must make the appropriate calls to
  free memory that was allocated in the previous steps 
  (vector specification objects, solver memory block, and any user data).
\end{enumerate}
%
%------------------------------------------------------
% Optional quadrature integration
%------------------------------------------------------
%
If there are any quadrature equations that must also be integrated, the user's 
main program must construct an additional NVECTOR specification object. 
Integration of the quadrature variables is activated and initialized through a 
call to the CVODES function {\tt CVodeQuadMalloc}, which must specify the
user-provided function for the evaluation of the quadrature integrands and
the integration tolerances for quadrature variables. As before, the user
has the option of changing from their default values various quantities
controlling the quadrature integration (see Table~\ref{t:optional_inputs}).
All these calls must preceed any call to the main CVODES solver function.
After a successful return from {\tt CVode}, the quadrature variables are
accessible through a call to {\tt CVodeGetQuad}, and solver statistics related
to quadrature integration are available through the functions listed in
Table~\ref{t:optional_outputs}.

%------------------------------------------------------
% Adjoint SA
%------------------------------------------------------
%
Adjoint sensitivity analysis inherently affects to a much greater extent the 
user interface, mainly due to the coupling between the forward and backward 
integration phases. 
%
In designing the user interface to the adjoint sensitivity module in CVODES we 
have strived to maintain the same ``look and feel'' as for that used for
ODE and forward sensitivity solution.
The initialization and set-up of the forward phase is the same as above.
Before calling the main solver for the forward integration, the user must 
call the CVODES function {\tt CVadjMalloc} to initialize and allocate memory 
for the structure holding the check-pointing and interpolation data.
The forward integration and check-point generation is done through a call
to {\tt CVodeF}, a wrapper around the {\tt CVode} function in step 
\ref{i:solve} above.
%
The initialization, set-up, and solution of the adjoint problem is then
done in the same way as for a regular forward ODE integration but
calling CVODES and linear solver wrapper functions that have the names 
mentioned before with the suffix {\tt B} attached. Some examples of such
CVODES functions are: {\tt CVodeCreateB}, {\tt CVSpgmrB}, {\tt CVodeMallocB},
and {\tt CVodeB}.

%------------------------------------------------------
% Tables
%------------------------------------------------------
%
% Optional inputs to CVODES, CVSDENSE, CVSBAND, CVSSPGMR
%
\begin{acmtable}{\textwidth}
\centering
\begin{tabularx}{\textwidth}{lll}
{\bf Optional input} & {\bf Function name} & {\bf Default} \\
\hline
\multicolumn{3}{c}{\bf CVODES solver} \\
\hline
Pointer to an error output file & {\tt CVodeSetErrFile} & NULL  \\
Data for right-hand side function & {\tt CVodeSetFdata} & NULL \\
Maximum order for BDF (or Adams) method & {\tt CVodeSetMaxOrd} & 5 (12) \\
Maximum no. of internal steps before $t_{\mbox{\scriptsize out}}$ & {\tt CVodeSetMaxNumSteps} & 500 \\
Maximum no. of warnings for $h < U$ & {\tt CVodeSetMaxHnilWarns} & 10 \\
Flag to activate stability limit detection & {\tt CVodeSetStabLimDet} & FALSE \\
Initial step size & {\tt CVodeSetInitStep} & estimated \\
Minimum absolute step size & {\tt CVodeSetMinStep} & 0.0 \\
Maximum absolute step size & {\tt CVodeSetMaxStep} & $\infty$ \\
Value of $t_{stop}$ & {\tt CVodeSetStopTime} & $\infty$ \\
Maximum no. of error test failures & {\tt CVodeSetMaxErrTestFails} & 7 \\
Maximum no. of nonlinear iterations & {\tt CVodeSetMaxNonlinIters} & 3 \\
Maximum no. of convergence failures & {\tt CVodeSetMaxConvFails} & 10 \\
Coefficient in the nonlinear convergence test & {\tt CVodeSetNonlinConvCoef} & 0.1 \\
Data for quadrature right-hand side function & {\tt CVodeSetQuadFdata} & NULL\\
Error control on quadrature variables & {\tt CVodeSetQuadErrCon} & FALSE \\
Tolerances for quadrature variables & {\tt CVodeSetQuadTolerances} & none \\
Sensitivity right-hand side function & {\tt CVodeSetSensRhsFn}  & internal DQ  \\
Data for sensitivity right-hand side function & {\tt CVodeSetSensFdata} & NULL \\
Error control on sensitivity variables & {\tt CVodeSetSensErrCon} & FALSE \\
Control for difference quotient approximation & {\tt CVodeSetSensRho}   & 0.0 \\
Vector of problem parameter scalings & {\tt CVodeSetSensPbar}  & NULL \\
Tolerances for sensitivity variables & {\tt CVodeSetSensTolerances} & estimated \\
\hline
\multicolumn{3}{c}{\bf CVSDENSE linear solver} \\
\hline
Dense Jacobian function & {\tt CVDenseSetJacFn} & internal DQ \\
Data for Jacobian function & {\tt CVDenseSetJacData} & NULL \\
\hline
\multicolumn{3}{c}{\bf CVSBAND linear solver} \\
\hline
Band Jacobian function & {\tt CVBandSetJacFn} & internal DQ \\
Data for Jacobian function & {\tt CVBandSetJacData} & NULL \\
\hline
\multicolumn{3}{c}{\bf CVSSPGMR linear solver} \\
\hline
Preconditioner solve function & {\tt CVSpgmrSetPrecSolveFn} & NULL \\
Preconditioner setup function & {\tt CVSpgmrSetPrecSetupFn} & NULL \\
Data for preconditioner functions & {\tt CVSpgmrSetPrecData} & NULL \\
Jacobian times vector function & {\tt CVSpgmrSetJacTimesVecFn} & NULL \\
Data for Jacobian times vector function &{\tt CVSpgmrSetJacData} & NULL \\
Type of Gram-Schmidt orthogonalization & {\tt CVSpgmrSetGSType} & classical GS \\
Ratio between linear and nonlinear tolerances & {\tt CVSpgmrSetDelt} & 0.05 \\
\end{tabularx}
\caption{Optional inputs for CVODES, CVSDENSE, CVSBAND, and CVSSPGMR}
\label{t:optional_inputs}
\end{acmtable}
%
% Optional outputs from CVODES
%
\begin{acmtable}{\textwidth}
\centering
\begin{tabularx}{\textwidth}{ll}
{\bf Optional output} & {\bf Function name} \\
\hline
Size of CVODES integer workspace & {\tt CVodeGetIntWorkSpace} \\
Size of CVODES real workspace & {\tt CVodeGetRealWorkSpace} \\
Cumulative number of internal steps & {\tt CVodeGetNumSteps} \\
No. of calls to r.h.s. function & {\tt CVodeGetNumRhsEvals} \\
No. of calls to linear solver setup function & {\tt CVodeGetNumLinSolvSetups} \\
No. of local error test failures that have occurred & {\tt CVodeGetNumErrTestFails} \\
Order used during the last step & {\tt CVodeGetLastOrder} \\
Order to be attempted on the next step & {\tt CVodeGetCurrentOrder} \\
Order reductions due to stability limit detection & {\tt CVodeGetNumStabLimOrderReds} \\
Actual initial step size used & {\tt CVodeGetActualInitStep} \\
Step size used for the last step & {\tt CVodeGetLastStep} \\
Step size to be attempted on the next step & {\tt CVodeGetCurrentStep} \\
Current internal time reached by the solver & {\tt CVodeGetCurrentTime} \\
Suggested factor for tolerance scaling  & {\tt CVodeGetTolScaleFactor} \\
Error weight vector for state variables & {\tt CVodeGetErrWeights} \\
Estimated local error vector & {\tt CVodeGetEstLocalErrors} \\
No. of nonlinear solver iterations & {\tt CVodeGetNumNonlinSolvIters} \\
No. of nonlinear convergence failures & {\tt CVodeGetNumNonlinSolvConvFails} \\
No. of calls to quadrature r.h.s. function & {\tt CVodeGetNumQuadRhsEvals} \\
No. of quadrature local error test failures & {\tt CVodeGetNumQuadErrTestFails} \\
Error weight vector for quadrature variables & {\tt CVodeGetQuadErrWeights} \\
No. of calls to sensitivity r.h.s. function & {\tt CVodeGetNumSensRhsEvals} \\
No. of calls to r.h.s. function due to (\ref{e:fd2}) or (\ref{e:dd2}) & {\tt CVodeGetNumRhsEvalsSens} \\
No. of sensitivity local error test failures & {\tt CVodeGetNumSensErrTestFails} \\
No. of calls to linear solver setup for forward SA & {\tt CVodeGetNumSensLinSolvSetups} \\
Error weight vectors for sensitivity variables & {\tt CVodeGetSensErrWeights} \\
No. of nonlinear solver iterations for forward SA & {\tt CVodeGetNumSensNonlinSolvIters} \\
No. of sensitivity nonlinear convergence failures & {\tt CVodeGetNumSensNonlinSolvConvFails}
\end{tabularx}
\caption{Principal optional outputs from CVODES. 
Additional optional statistics (not listed) are available for the staggered 
corrector forward sensitivity method.}
\label{t:optional_outputs}
\end{acmtable}
