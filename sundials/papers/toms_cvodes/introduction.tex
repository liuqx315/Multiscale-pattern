\section{Introduction}\label{s:introduction}

Fortran solvers for ODE initial value problems (IVPs) are widespread and heavily used. 
Two solvers that have been written at LLNL in the past are VODE \cite{BBH:89} 
and VODPK \cite{Byr:92}.
%
%% VODE is a general purpose solver that includes methods for stiff
%% and nonstiff systems, and in the stiff case uses direct methods (full or
%% banded) for the solution of the linear systems that arise at each implicit
%% step. Externally, VODE is very similar to the well known solver
%% LSODE \cite{RaHi:94}. 
%% VODPK is a variant of VODE that uses a preconditioned Krylov 
%% (iterative) method for the solution of the linear systems. VODPK is a powerful 
%% tool for large stiff systems because it combines established methods for stiff 
%% integration, nonlinear iteration, and Krylov (linear) iteration with a problem-specific
%% treatment of the dominant source of stiffness, in the form of the user-supplied
%% preconditioner matrix~\cite{BrHi:89}.
%
The capabilities of both VODE and VODPK have been combined in the C-language 
packages CVODE~\cite{CoHi:96} and PVODE~\cite{ByHi:99}, later merged under the
suite SUNDIALS~\cite{HBGLSSW:04} into one solver, CVODE, which runs on both
serial and parallel computers. Besides CVODE, the other two basic solvers in SUNDIALS
are IDA, a solver for differential-algebraic equation (DAE) systems, and
KINSOL, a Newton-Krylov (GMRES) solver for nonlinear algebraic systems.

In recent years, research and development related to the SUNDIALS solvers has
focused on sensitivity analysis to address questions related to unknown 
parameters in the mathematical models under consideration.
Essentially, sensitivity analysis quantifies the relationship between
changes in model parameters and changes in model outputs. Such information 
is crucial for design optimization, parameter estimation, optimal control, data
assimilation, process sensitivity, and experimental design.
%%
%% There are two main approaches to sensitivity analysis.
%% {\em Forward sensitivity analysis} has been proven to be very efficient for
%% problems in which the sensitivities of a (potentially) very large
%% number of quantities with respect to relatively few parameters are
%% needed.  However, for problems where the number of uncertain
%% parameters is large, the forward sensitivity method becomes
%% computationally intractable.  The {\em adjoint sensitivity method}, also called
%% reverse method, is advantageous in the complementary situation where the sensitivities of
%% a few quantities with respect to a large number of parameters are needed.
%%
SUNDIALS is currently being expanded to include sensitivity-capable variants of all
its basic solvers. The first one, CVODES, released in July 2002, 
is written with a functionality and user interface that is a superset of that of CVODE. 
In that sense, CVODES is backward compatible with CVODE.
Sensitivity analysis capabilities, both forward and adjoint, have been added to 
the main integrator. Enabling forward sensititivity computations 
in CVODES will result in the code integrating the so-called 
{\em sensitivity equations} simultaneously with the original IVP, 
yielding both the solution and its sensitivity with respect to parameters in the model. 
Adjoint sensitivity analysis involves integration of the 
original IVP forward in time followed by the integration of the so-called 
{\em adjoint equations} backwards in time. 
CVODES provides the infrastructure needed to integrate any final-condition ODE
dependent on the solution of the original IVP (not only the adjoint system). 

Development of CVODES was concurrent with a redesign of the vector operations module 
(NVECTOR) across the SUNDIALS suite. The key feature of the new NVECTOR module is that it
is written in terms of abstract vector operations with the actual vector kernels attached
by a particular implementation (such as serial or parallel). This allows
writing the SUNDIALS solvers in a manner independent of the actual NVECTOR vector
implementation (which can be user-supplied), as well as allowing more than one 
NVECTOR module linked into an executable file. This feature is essential in certain
sensitivity analysis computations and impossible in Fortran.

Like all the SUNDIALS solvers, CVODES is written in ANSI-C.
Among the advantages of using C, we mention portability of the solver libraries, 
compiler availability, a standard dynamic memory allocation mechanism, and the
ability to define complex data structures.
The design of the SUNDIALS solvers does not impede interlanguage operability. 
As an example of using a SUNDIALS solver with Fortran user code, the reader is 
refered to the Fortran-C interface provided for CVODE \cite{HiSe:04cvode,HBGLSSW:04}.

The rest of this paper is organized as follows. In Section~\ref{s:algorithms}, the
algorithms implemented in CVODES for ODE integration and forward and adjoint 
sensitivity analysis are presented. The CVODES code organization and relationship
to SUNDIALS is discussed in Section~\ref{s:organization}, while Section~\ref{s:usage}
gives a high-level overview of the solver usage and general
philosophy of the user interface. 
Section~\ref{s:parallel} presents an example problem and its solution on a
parallel machine.
We conclude with indications on software availability
in Section~\ref{s:availability} and with some final remarks and directions
of current and future development in Section~\ref{s:conclusions}.
