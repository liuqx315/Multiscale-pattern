\section{Usage} 
\label{s:usage}

The new design and organization of SUNDIALS 
as described in Section~\ref{s:organization}
makes the codes flexible and easy to use.
This versatility is due primarily to the control that the user has
over the modules that comprise SUNDIALS: the specification of vectors;
the linear solver and preconditioner methods; the basic solvers; and
sensitivity analysis.
Default routines are provided for computing Jacobian-vector 
approximations, or the right-hand side of the forward sensitivity
systems, for example. But for these routines and other basic operations, 
SUNDIALS allows the user to provide their own variants that may be better 
suited to their problem-solving needs. 
Additionally, SUNDIALS provides the user with a fine level of control
over various algorithmic parameters, heuristic values, and data
structure pointers contained within the codes.
Finally, SUNDIALS provides optional routines for extracting the
solution, solver statistics, and other useful information from the codes.

A general approach for using SUNDIALS is given below. 
The outline conveys the basic elements of what is needed to properly
specify and solve a problem, the order in which certain tasks must be
done, the opportunities for providing user-supplied
routines or input values, and so on.
Complete details and additional examples are in the documentation that
accompanies each solver in SUNDIALS.

\begin{enumerate}

\item \label{sun_headers}
SUNDIALS contains header files that define various constants,
enumerations, macros, data types, and function prototypes.  At a
minimum, the user must include header files that declare: the SUNDIALS
data types for real, integer, and boolean variables; the NVECTOR
implementation to be used; and the solver functions needed to
set up and initialize the problem, compute, and extract the solution. 
Typically, additional header files will be specified to declare the 
preconditioning and/or linear solver methods to be used.

\item \label{sun_problem}
The user must provide a function for evaluating the equations to be
solved. Optionally, a user-defined data structure can be created and
passed to this function. 

\item \label{sun_size}
To completely specify the problem, the user must 
provide whatever initial guesses and/or initial values are
needed, specify solution error tolerances, and so on. 

\item \label{sun_create}
The next step is to call a routine for initializing a block of memory
that will be used in solving the problem. The memory block is created
with certain default values for the solver, such as the use of
standard output for writing warning and error messages, or NULL
as a default value for the pointer to the user-specified data
structure to be passed in evaluating the user's function.

\item \label{sun_set}
At this stage, the default values in the solver memory block can be
changed if so desired. Choices and default values are given in
Table~\ref{t:optional_input} for each of the basic solvers and
are discussed further below.

\item \label{sun_malloc}
After checking the initialized memory block for
errors in the default or optional input values, the user now calls 
the appropriate routine to perform any required
memory allocation.

\item \label{sun_linear}
Typically, preconditioning and/or linear solver methods are needed for
solving the linear systems that may arise. These methods can now be
attached to the block of memory allocated for the solver.  Likewise,
if rootfinding is to be done (by CVODE or CVODES) along with the
integration, then the user specifications for that task are also
attached at this point.

\item \label{sun_solve}
The appropriate routine is called to solve the problem according to
the tolerances and other settings that have been specified.

\item \label{sun_get}
To extract the solution, solver statistics, and other information,
optional output extraction routines can be called. 
A listing of the optional outputs for the basic solvers is given 
in Table~\ref{t:optional_output}.

\item \label{sun_finalize}
To end the process, the user must make the appropriate calls to
free memory that was allocated in the previous steps. Otherwise, if
applicable, a re-initialization routine can be called for solving
additional problems.

\end{enumerate}

In order to carry out sensitivity analysis, the above outline
needs to be modified at several steps. For forward sensitivities,
Step~\ref{sun_headers} requires that the appropriate header
file for forward sensitivity analysis be used in place of the header
file for the basic solver. At Step~\ref{sun_problem}, the user must
create an array of real parameters upon which the solution depends
and attach a pointer to this array to the user-defined data structure
that is passed to the user's function. Also, the user must specify the
number of sensitivities to be computed and provide an array that
indicates which solution sensitivities are to be
computed. Step~\ref{sun_malloc} requires that the user call the memory
allocation routine for the forward sensitivity version of the basic
solver. As the solution and forward sensitivities are computed,
these results and various solver statistics can be extracted as part
of Steps~\ref{sun_solve}--\ref{sun_get}. Finally, memory space that
has been allocated previously must be freed at
Step~\ref{sun_finalize}. For complete details on performing forward or
adjoint sensitivity analysis for CVODES, the reader is referred to~\cite{SeHi:03}.

If using the parallel NVECTOR module in SUNDIALS, the MPI header file must be
specified in Step~\ref{sun_headers} so that in Step~\ref{sun_size} the MPI
communicator can be initialized, the set of active processors can be
established, and the global and local vector lengths can be set.
In Step~\ref{sun_finalize}, memory allocated for MPI must be freed.

\subsection{Optional inputs and outputs}\label{ss:optional_io}

Within SUNDIALS, an attempt is made to set reasonable defaults for the
various methods, heuristic parameters, and pointers used in the
codes. 
A key feature of SUNDIALS is that it provides a collection of optional
input and output routines so that default settings can be changed, or
various solver statistics and other information can be extracted.
These ``set'' and ``get'' routines are available for each of the solvers, as
noted, as well as for the linear solver and preconditioning methods that
support them.

\subsubsection*{Basic Solvers} 
Table \ref{t:optional_input} lists the various optional 
inputs that the user can set to control the basic solvers within SUNDIALS.
Under each solver column we give the default value for the respective
input. Inputs marked with a ``-'' are not applicable to that particular 
solver. Table \ref{t:optional_output} lists the various optional 
outputs that the user can get to monitor solver performance.
Optional outputs available for a solver are marked with a ``$\checkmark$''
and those not available are marked by a ``-''.

\begin{acmtable}{340pt}
\centering
\begin{tabular}{p{2.75in} c c c }
Optional input & CVODE  & IDA & KINSOL \\
\hline
Pointer to the user-defined data & NULL & NULL& NULL \\
Pointer to an error file & NULL & NULL & NULL \\
Maximum order for BDF method & 5 & 5 & - \\
Maximum order for Adams method& 12  & - & - \\
Maximum number of internal steps before $t_{out}$ & 500 & 500 & - \\
Maximum number of warnings for $h < U$ & 10 & - & - \\
Flag to activate stability limit detection & FALSE & - & - \\
Initial step size & est. & est. & - \\
Minimum absolute step size & 0.0 & - & - \\
Maximum absolute step size & $\infty$ & $\infty$ & - \\
Value of $t_{stop}$ & - & $\infty$ & - \\
Maximum number of Newton iterations & 3 & 4 & 200 \\
Maximum number of convergence failures & 10 & 10 & - \\
Maximum number of error test failures & 7 & 10 & - \\
Coefficient in the nonlinear convergence test & 0.1 & 0.33 & - \\
Flag to exclude algebraic variables from error test & - & FALSE & - \\
Differential-algebraic identification vector & - & NULL & - \\
Vector with additional constraints & - & NULL & NULL \\
Flag to skip initial linear solver setup call & - & - & FALSE \\
Maximum number of prec. solves without setup & - & - & 10 \\
Flag for selection of $\eta$ computation & - & - & choice 1 \\
Constant $\eta$ value & - & - & 0.1 \\
Parameters $\alpha$ and $\gamma$ in $\eta$ choice 2 & - & - & $2.0$,$0.9$\\
Flag to control minimum value for $\epsilon$ & - & - & FALSE \\
Maximum length of Newton step & - & - & est. \\
Relative error in computing $F(u)$ & - & - & $U$ \\
%Constant to restrict solution update & - & - & $\infty$ \\
Stopping tolerance on residual & - & - & $U^{1/3}$ \\
Stopping tolerance on max. scaled step & - & - & $U^{2/3}$ \\
\end{tabular}
\caption{Optional inputs for the basic solvers in SUNDIALS. 
The value of unit roundoff for the machine is denoted by $U$, 
and ``est.'' indicates that a quantity is automatically 
estimated by the code.}
\label{t:optional_input}
\end{acmtable}

\begin{acmtable}{327pt}
\centering
\begin{tabular}{p{2.75in} c c c }
Optional output & CVODE  & IDA & KINSOL \\
\hline
Size of workspace allocated by the solver & $\checkmark$ & $\checkmark$ & $\checkmark$ \\
Cumulative number of internal steps taken & $\checkmark$ & $\checkmark$ & - \\
Number of calls to the user's function & $\checkmark$ & $\checkmark$ & $\checkmark$ \\
Number of calls to the linear solver's setup routine & $\checkmark$ & $\checkmark$ & - \\
Number of local error test failures that have occurred & $\checkmark$ & $\checkmark$ & - \\
Number of nonlinear solver iterations & $\checkmark$ & $\checkmark$ & $\checkmark$ \\
Number of nonlinear convergence failures & $\checkmark$ & $\checkmark$ & - \\
Order used during the last step & $\checkmark$ & $\checkmark$ & - \\
Order to be attempted on the next step & $\checkmark$ & $\checkmark$ & - \\
Order reductions due to stability limit detection & $\checkmark$ & - & - \\
Actual initial step size used & $\checkmark$ & $\checkmark$ & - \\
Step size used for the last step & $\checkmark$ & $\checkmark$ & - \\
Step size to be attempted on the next step & $\checkmark$ & $\checkmark$ & - \\
Current internal time reached by the solver & $\checkmark$ & $\checkmark$ & - \\
%Suggested factor for tolerance scaling & $\checkmark$ & - & - \\
Vector containing the error weights for state variables & $\checkmark$ & $\checkmark$ & - \\
Vector containing the estimated local errors & $\checkmark$ & - & - \\
Number of backtrack operations during linesearch & - & $\checkmark$ & $\checkmark$ \\
Number of times the $\beta$ condition could not be met & - & - & $\checkmark$ \\
Scaled norm at a given iteration & - & - & $\checkmark$ \\
Last step length in the global strategy routine & - & - & $\checkmark$ \\
Information on roots found & $\checkmark$ & - & - \\
\end{tabular}
\caption{Optional outputs for the basic solvers in SUNDIALS.}
\label{t:optional_output}
\end{acmtable}

\subsubsection*{Sensitivity Analysis}

Each sensitivity solver (CVODES and IDAS) offers the complete list of
optional ``set'' and ``get'' routines as the corresponding basic solver (CVODE
and IDA, respectively). In addition, the user has control over various
inputs that affect sensitivity calculations. 
The following are examples of options that can be set by the user
with the default given in parentheses: a user-supplied routine
to compute sensitivity ODEs or DAE sensitivity residuals (CVODES or
IDAS difference quotient approximation); a pointer to user data that
will be passed to this user-supplied ODE or DAE sensitivity routine (NULL); a
pointer to the sensitivity relative error tolerance scalar (same value as
for state variables); and a boolean flag indicating whether the sensitivity
variables are included in the error control mechanism (FALSE).
For more options and details, see \cite{SeHi:03,HiSe:04cvodes}.

\subsubsection*{Linear Solvers and Preconditioners}

For any of the linear solvers, the user can set optional inputs so 
that a user-supplied routine providing Jacobian-related information
is used instead of the default difference quotient routine. 
Also, a pointer can be set so that user data is passed each time this
user-supplied routine is called. In addition, for the SPGMR case,
the following can be optionally changed from their default values
(provided in parentheses): a classical Gram-Schmidt orthogonalization 
(modified Gram-Schmidt), the factor by which the tolerance on the
nonlinear iteration is multiplied to get a tolerance on the linear
iteration (0.05); the preconditioner setup routine (NULL); the
preconditioner solver routine (NULL); and a pointer to the user
preconditioner data (NULL).

The optional outputs for any of the linear solvers are: the amount of 
integer and real workspace used; the number of calls made to the user-supplied 
Jacobian evaluation routine; and the number of calls to the user's function 
within the default difference quotient routine.
In addition, for the SPGMR case the user can 
obtain the number of preconditioner
evaluations, the number of calls made to the preconditioner solve
routine, the number of linear iterations, and the number of linear
convergence failures.

For the band-block-diagonal preconditioner, the optional outputs are:
the amount of integer workspace used; the amount of real workspace
used; and the number of calls to the local function that approximates
the user's function.

\subsection{Fortran Usage} 
\label{ss:Fortran_usage}

Some support is available for using Fortran 77 and Fortran 90
applications with
SUNDIALS.  In particular, a Fortran/C interface package is provided
with CVODE and KINSOL.  Each package is a collection of C header files
and functions that provide interfaces from user Fortran routines to
solver C routines, and the reverse.  These enable the user to write a
main program and all user-supplied routines in Fortran, and then use
either CVODE or KINSOL to solve the problem.  This mixed-language
capability entails some compromises in portability, such as requiring
fixed names for the user-supplied routines, but the restrictions are
minor. For complete details, see the CVODE and KINSOL user 
documentation~(\cite{HiSe:04cvode,HSW:04kinsol}).
