\section{Introduction}

With the ever-increasing capabilities of modern computers, simulation
code developers are challenged to develop fast and robust software
capable of solving problems with increasingly higher resolutions and
modeling more complex
physical phenomena.  At the heart of many numerical simulation codes
lie systems of nonlinear algebraic or time-dependent equations, and
simulation scientists continue to require efficient solvers for these
systems.

To meet this need, Lawrence Livermore National Laboratory
has a long history of research and development in ordinary
differential equation (ODE) methods and software, as well as closely related
areas, with emphasis on applications to partial differential equations
(PDEs).  Among the popular Fortran 77 solvers written at LLNL are the
following:
\begin{itemize}
\item VODE: a solver for ODE initial-value problems for stiff/nonstiff
systems, with direct solution of linear systems, by Brown, Byrne, and
Hindmarsh \cite{BBH:89}.
\item VODPK: a variant of VODE with preconditioned Krylov (GMRES
iteration \cite{SaSc:86}) solution of the linear systems in place
of direct methods, by Brown, Byrne, and Hindmarsh \cite{Byr:92}.
\item NKSOL: a Newton-Krylov (GMRES) solver for nonlinear algebraic
systems, by Brown and Saad \cite{BrSa:90}.
\item DASPK: a solver for differential-algebraic equation (DAE)
systems (a variant of DASSL) with both direct and preconditioned
Krylov solution methods for the linear systems, by Brown, Hindmarsh,
and Petzold \cite{BHP:94}.
\end{itemize}
Starting in 1993, the push to solve large systems in parallel
motivated work to write or rewrite solvers in C. Moving to the C
language was done to: exploit features of C not present in Fortran
77 while using languages with stable compilers (F90/95 were not
yet stable when this work started); achieve a more object-oriented
design; facilitate the use of the codes with other object-oriented
codes being written in C and C++; maximize the reuse of code
modules; and facilitate the extension from a serial to a parallel
implementation. The first result of the C effort was CVODE. This
code was a rewrite in ANSI standard C of the VODE and VODPK
solvers combined, for serial machines \cite{CoHi:94,CoHi:96}.  The
next result of this effort was PVODE, a parallel extension of
CVODE \cite{ByHi:98,ByHi:99}. Similar rewrites of NKSOL and DASPK
followed, using the same general design as CVODE and PVODE.  The
resulting solvers are called KINSOL and IDA, respectively. More
recently, we have merged the PVODE and CVODE codes into a single
solver, CVODE.

The main numerical operations performed in these codes are
operations on data vectors, and the codes have been written in
terms of interfaces to these vector operations.  The result of
this design is that users can relatively easily provide their own
data structures to the solvers by telling the solver about their
structures and providing the required operations on them. The
codes also come with default vector structures
with pre-defined operation implementations for both serial and
distributed memory parallel environments in case a user prefers to
not supply their own structures. In addition, all parallelism is
contained within specific vector operations (norms, dot products,
etc.)  No other operations within the solvers require knowledge of
parallelism. Thus, using a solver in parallel consists of using
a parallel vector implementation, either the one provided with
SUNDIALS, or the user's own parallel vector structure, 
underneath the solver.  
Hence, we no longer make a distinction between parallel and serial 
versions of the codes.

These codes have been combined into the core of SUNDIALS, the
SUite of Nonlinear and DIfferential/Algebraic equation Solvers.
This suite, consisting of CVODE, KINSOL, and IDA (along with
current and future augmentations to include forward and adjoint
sensitivity analysis capabilities), was implemented with the goal
of providing robust time integrators and nonlinear solvers that
can easily be incorporated into existing simulation codes.  The
primary design goals were to require minimal information from the
user, allow users to easily supply their own data structures
underneath the solvers, and allow for easy incorporation of
user-supplied linear solvers and preconditioners.

As simulations have increased in size and complexity, the
relationship between computed results and problem parameters has
become harder to establish.  Scientists have a greater need to
understand answers to questions like the following.  Which model
parameters are most influential? How can parameters be adjusted to
better match experimental data? What is the uncertainty in
solutions given uncertainty in data?  Sensitivity analysis
provides information about the relationship between simulation
results and model data, which can be critical to answering these
questions. In addition, for a given parameter, sensitivities can
be computed in a modest multiple of the computing time of the
simulation itself.

SUNDIALS is being expanded to include forward and adjoint
sensitivity versions of the solvers. The first of these, CVODES,
is complete.  A brief description of the strategy for adding
sensitivity analysis in a way that respects the user interfaces of
the SUNDIALS codes is contained in this paper, and a more thorough
description of the CVODES package (which is distinct from but
built on the same core code as CVODE) is contained in a companion
paper \cite{SeHi:03}. The second sensitivity solver will be IDAS
and is currently under development. Extensions to KINSOL for
sensitivity analysis will be completed if need arises.

The rest of this paper is organized as follows.  In Section
\ref{s:basic_solvers}, the algorithms in the three core solvers of
SUNDIALS are presented. We have attempted to identify many of the
heuristics related to stopping criteria and finite-difference
parameter selection where appropriate, as these items can
sometimes affect algorithm performance significantly. In Section
\ref{s:preconditioning}, the preconditioning packages supplied
with SUNDIALS are described.  Section \ref{s:sensitivity_analysis}
overviews the CVODES package and strategies for adding sensitivity
capabilities to the codes. Sections \ref{s:organization} and
\ref{s:usage} describe the organization of the codes within the
suite and the philosophy of the user interface. Availability of
the codes is given in Section \ref{s:availability}.  Finally,
comments on applications of SUNDIALS, concluding remarks, and
indications of future development are contained in the last
section.
