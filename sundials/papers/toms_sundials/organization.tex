\section{Code Organization}
\label{s:organization}

The writing of CVODE from the Fortran 77 solvers VODE and VODPK
initiated a complete redesign and reorganization of the existing
LLNL solver coding. The features of the design of CVODE include
the following:
\begin{itemize}
\item Memory allocation is heavily used.
\item The linear solver modules are separate from the core integrator,
so that the latter is independent of the method for solving linear
systems.
\item Each linear solver module contains  a generic solver, which is
independent of the ODE context, together with an interface to the CVODE core
integrator module.
\item The vector operations  (linear sums, dot products, norms, etc.) on
$N$-vectors are isolated in a separate NVECTOR module.
\end{itemize}

The process of modularization has continued with the development of CVODE,
KINSOL, and IDA. The SUNDIALS distribution now contains a number of common
modules in a shared directory. Additionally, compilation of SUNDIALS is now
independent of any prior specification of a particular NVECTOR
implementation, facilitating the use of binary libraries. The current
NVECTOR design also allows the use of multiple implementations within the
same code, as may be required to meet user needs.

Figure \ref{fig-sunorg} shows the overall structure of SUNDIALS, with the
various separate modules. The evolution of SUNDIALS has been directed toward
keeping the entire set of solvers in mind. Thus, CVODE, KINSOL, and IDA share
much in their organization and have a number of common modules.  The
separation of the linear solvers from the core integrators allows for easy
addition of linear solvers not currently included in SUNDIALS. At the bottom
level is the NVECTOR module, providing key vector operations such as creation,
duplication, destruction, summation, and dot products on potentially
distributed data vectors. Serial and parallel NVECTOR implementations are
included with SUNDIALS, but a user can substitute his/her own implementation
as useful. Two small modules defining several data types and elementary
mathematical operations are also included.

\begin{figure}[tp]
\centerline{\psfig{figure=sunorg.eps,width=\textwidth}}
\caption{Overall structure of the SUNDIALS  package.}
\label{fig-sunorg}
\end{figure}

A number of necessary and optional user-supplied routines for the solvers
in SUNDIALS
are not shown in \mbox{Figure \ref{fig-sunorg}}. The user must
provide a routine for the evaluation of $f$ (CVODE) or $F$ (KINSOL and
IDA). The user-provided routines may include, depending on the options
chosen, routines for Jacobian evaluation (direct cases) or Jacobian-vector
products (Krylov case), and routines for the setup and solution of Krylov
preconditioners.


\subsection{Shared Modules - Linear Solvers}

As can be seen in \mbox{Figure \ref{fig-sunorg}}, three linear solver
packages are currently included with SUNDIALS: a direct dense matrix solver
(DENSE); a direct band solver (BAND); and an iterative Krylov solver
(SPGMR). These are stand-alone packages in their own right.

The shared linear solvers are accessed from SUNDIALS via solver-specific
wrappers.  Thus, SPGMR is accessed via CVSPGMR, IDASPGMR, and KINSPGMR,
for CVODE (and CVODES), IDA, and KINSOL, respectively. For the DENSE
solver, the wrappers are CVDENSE and IDADENSE for CVODE/CVODES and
IDA, respectively. Similar wrappers for BAND are CVBAND and IDABAND.

Within each solver, each linear solver module consists primarily of the
user-callable function that specifies that linear solver, and four or five
wrapper routines.  The wrappers conform to a fixed set of specifications,
enabling the central solver routines to be independent of the linear
system method used.  Specifically, for each such module, there are
four wrappers --- for initialization, Jacobian/precondioner matrix
setup, linear system solution, and memory freeing.  The IDA modules
have a fifth wrapper, for linear solver performance monitoring.  These
wrapper specifications are fully described in the user documentation
for each solver.  By following those, and using any of the existing modules
as a model, the user can add a linear solver module to the package, if
appropriate.

\subsection{Shared Modules - NVECTOR}

A generic NVECTOR implementation is used within SUNDIALS to
operate on vectors. This generic implementation defines an NVECTOR
structure which consists of an implementation-specific content and a set 
of abstract vector operations. The NVECTOR module also provides a set of 
wrappers for accessing the actual vector operations of the implementation 
under which an NVECTOR was created. 
Because details of vector operations are thus encapsulated within each specific
NVECTOR implementation, the solvers in SUNDIALS are now independent of a specific
implementation. This allows the solvers to be precompiled as binary
libraries and allows more than one NVECTOR implementation to be used within
a single program.

A particular NVECTOR implementation, such as the serial and parallel
implementations included with SUNDIALS or a user-provided implementation,
must provide certain functionalities. 
At a minimum, each implementation must provide functions to create a new
vector, a function to destroy such vectors, and the definitions of
vector operations required by the SUNDIALS solvers, including, for example, 
duplication, summation, element-by-element inversion, and dot product. 

If neither the serial nor parallel NVECTOR implementation provided within
SUNDIALS is suitable, the user can provide one or more NVECTOR implementations.  
For example, it might (and has been) more practical to substitute a more complex
data structure in a parallel implementation.

For complete details, see the user documentation for any of the solvers in 
SUNDIALS~\cite{HiSe:04cvode,HiSe:04cvodes,HiSe:04ida,HSW:04kinsol}.

\subsection{User Interface Design}

When CVODE was initially developed from VODE and VODPK, its user
interface was completely redesigned, and the same design principles
were adopted when the other solvers were added to the suite.  Further
changes in the interface design were made more recently.  Unlike the
typical Fortran solver, where the interface consists of one callable
routine with many arguments, the user interface to each of the
SUNDIALS solvers involves many callable routines, each with only a few
arguments.  The various routines specify the various aspects of the
problem to be solved and of the solution method to be used, or
retrieve information about the solution.  For each solver, there are
separate (required) calls that initialize and allocate memory for
the problem solver and for the linear system solver it will use.  
Then there are optional calls to specify various
optional inputs (ranging from scalars like maximum method order to the
user-supplied Jacobian routine), and optional calls to obtain optional
outputs (mostly performance statistics).  Finally, there is a call to
restart the solver, when a new problem of the same size is to be
solved, but possibly with different initial conditions or a different
right-hand side function.  The interface design is intended to be
fairly simple for the casual user, but also suitably rich and flexible
for the more expert user.
