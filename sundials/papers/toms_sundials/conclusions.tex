\section{Conclusions}\label{s:conclusions}

The time integrators and nonlinear solvers within SUNDIALS have been
developed to take advantage of the long history of research and development
of such codes at LLNL.  The codes feature state-of-the-art
technology for BDF time integration as well as for inexact Newton-Krylov
methods.  The design philosophy of providing clear interfaces to the
user and allowing the user to supply their own data structures makes the
solvers reasonably easy to add into existing simulation codes.  As a result,
these solvers have been used in numerous applications.

In particular, CVODE has been used to solve 3-dimensional radiation diffusion
problems on up to 5,800 processors of the ASCI Red machine and
verifying the scalability
of a fully implicit approach for these problems \cite{BrWo:01}.
The same code using a preliminary sensitivity version of CVODE was
further used to examine behaviors of solution sensitivities to
parameters that characterize material opacities for these
diffusion problems \cite{LWG:03,LHB:00}.  CVODE is also being used in a
3-dimensional tokamak turbulence model within LLNL's
Magnetic Fusion Energy Division to solve fusion energy simulation problems
with approximately 1.1 million unknowns on 60 processors \cite{RXH:02}.
KINSOL is being applied within
LLNL to solve a nonlinear Richards' equation model for pressures in
variably saturated porous media flows.
Fully scalable solution performance of this code has been obtained on
up to 225 processors of ASCI Blue \cite{JoWo:01,Woo:98}.  The same code using
a preliminary sensitivity version was used to quantify
uncertainty due to variations in relative permeability input parameters
within these groundwater problems \cite{WGM:02}.
IDA has been used in a cloud and aerosol microphysics
model at LLNL to study cloud formation processes and to study model
parameter sensitivity.
CVODE, CVODES, KINSOL, and IDA, with multigrid preconditioners,
are being used to
solve 3D neutral particle transport problems within LLNL

Although the SUNDIALS codes have proven to be versatile and
robust, further development of the suite is underway. In
particular, a sensitivity version of IDA, called IDAS, is
currently under development.  This code will have forward and
adjoint sensitivity capabilities similar to CVODES.  Further
nonlinear solver capabilities are being considered for extensions
to KINSOL, including a trust region globalization method as well
as other strategies for choosing finite differencing parameters.
In addition, a Picard iteration package and a BiCGStab Krylov
solver module are also planned for addition to SUNDIALS.
