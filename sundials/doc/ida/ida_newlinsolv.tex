%===================================================================================
\chapter{Providing Alternate Linear Solver Modules}\label{s:new_linsolv}
%===================================================================================
The central {\ida} module interfaces with the linear solver module to be
used by way of calls to five routines.  These are denoted here by 
\id{linit}, \id{lsetup}, \id{lsolve}, \id{lperf}, and \id{lfree}.
Briefly, their purposes are as follows:
%%
%%
\begin{itemize}
\item \id{linit}: initialize and allocate memory specific to the
  linear solver;
\item \id{lsetup}: evaluate and preprocess the Jacobian or preconditioner;
\item \id{lsolve}: solve the linear system;
\item \id{lperf}: monitor performance and issue warnings;
\item \id{lfree}: free the linear solver memory.
\end{itemize}
%%
%%
A linear solver module must also provide a user-callable specification routine
(like those described in \S\ref{sss:lin_solv_init}) which will attach
the above five routines to the main {\ida} memory block. 
The {\ida} memory block is a structure defined in the header file \id{ida\_impl.h}. 
A pointer to such a structure is defined as the type \id{IDAMem}. 
The five fields in a \id{IDAMem} structure that must point to the linear solver's 
functions are \id{ida\_linit}, \id{ida\_lsetup}, \id{ida\_lsolve}, \id{ida\_lperf}, 
and \id{ida\_lfree}, respectively.
Note that of the four interface routines, only the \id{lsolve} routine is required. 
The \id{lfree} routine must be provided only if the solver specification routine
makes any memory allocation.
The linear solver specification function must also set the value of
the field \id{ida\_setupNonNull} in the {\ida} memory block --- to
\id{TRUE} if \id{lsetup} is used, or \id{FALSE} otherwise.

For consistency with the existing {\ida} linear solver modules, we
recommend that the return value of the specification function be 0 for
a successful return or a negative value if an error occurs (the
pointer to the main {\ida} memory block is \id{NULL}, an input is
illegal, the {\nvector} implementation is not compatible, a memory
allocation fails, etc.)

To facilitate data exchange between the five interface functions, the
field \id{ida\_lmem} in the {\ida} memory block can be used to attach
a linear solver-specific memory block.
That memory should be allocated in the linear solver specification function.

\vspace{0.1in}
These five routines, which interface between {\ida} and the linear solver module,
necessarily have fixed call sequences.  Thus a user wishing to implement another 
linear solver within the {\ida} package must adhere to this set of interfaces.
The following is a complete description of the call list for each of
these routines.  Note that the call list of each routine includes a
pointer to the main {\ida} memory block, by which the routine can access
various data related to the {\ida} solution.  The contents of this memory
block are given in the file \id{ida.h} (but not reproduced here, for
the sake of space).

%------------------------------------------------------------------------------

\section{Initialization function}
The type definition of \id{linit} is
\usfunction{linit}
{
  int (*linit)(IDAMem IDA\_mem);
}
{
  The purpose of \id{linit} is to complete initializations for      
  a specific linear solver, such as counters and statistics.        
}
{
  \begin{args}[IDA\_mem]
  \item[IDA\_mem]
    is the {\ida} memory pointer of type \id{IDAMem}.
  \end{args}
}
{
  An \id{linit} function should return $0$ if it 
  has successfully initialized the {\ida} linear solver and 
  a negative value otherwise. 
}
{}

%------------------------------------------------------------------------------

\section{Setup routine} 
The type definition of \id{lsetup} is
\usfunction{lsetup}
{
   int (*lsetup)(&IDAMem IDA\_mem, N\_Vector yyp, N\_Vector ypp,\\
                 &N\_Vector resp,\\
                 &N\_Vector vtemp1, N\_Vector vtemp2, N\_Vector vtemp3); 
}
{
  The job of \id{lsetup} is to prepare the linear solver for subsequent 
  calls to \id{lsolve}. It may re-compute Jacobian-related data if it 
  deems necessary. 
}
{
   \begin{args}[IDA\_mem]
  
   \item[IDA\_mem] 
     is the {\ida} memory pointer of type \id{IDAMem}.
  
   \item[yyp]
     is the predicted $y$ vector for the current {\ida} internal step.
  
   \item[ypp]
     is the predicted $\dot{y}$ vector for the current {\ida} internal step.
  
   \item[resp]
     is the value of the residual function at \id{yyp} and \id{ypp}, i.e.
     $F(t_n, y_{pred}, \dot{y}_{pred})$.
  
   \item[vtemp1] 
   \item[vtemp2]
   \item[vtemp3] 
     are temporary variables of type \id{N\_Vector} provided for use by \id{lsetup}.      
  
   \end{args}
}
{
  The \id{lsetup} routine should return \id{0} if successful,            
  a positive value for a recoverable error, and a negative value  
  for an unrecoverable error.  
}
{}

%------------------------------------------------------------------------------

\section{Solve routine}
The type definition of \id{lsolve} is
\usfunction{lsolve}
{
  int (*lsolve)(&IDAMem IDA\_mem, N\_Vector b, N\_Vector weight, \\
                &N\_Vector ycur, N\_Vector ypcur, N\_Vector rescur);  
}
{
  The routine \id{lsolve} must solve the linear equation $M x = b$, where         
  $M$ is some approximation to
  $J = \partial F / \partial y + cj ~ \partial F / \partial \dot{y}$  
  (see Eqn. (\ref{e:DAE_Jacobian})), and the right-hand side vector $b$ is input. 
  Here $cj$ is available as \id{IDA\_mem->ida\_cj}.
}
{
  \begin{args}[IDA\_mem]
  \item[IDA\_mem]
    is the {\ida} memory pointer of type \id{IDAMem}.
  \item[b]
    is the right-hand side vector $b$. The solution is to be    
    returned in the vector \id{b}.
  \item[weight]
    is a vector that contains the error weights.
    These are the $W_i$ of (\ref{e:errwt}).
  \item[ycur]
    is a vector that contains the solver's current approximation to $y(t_n)$.
  \item[ypcur]
    is a vector that contains the solver's current approximation to $\dot{y}(t_n)$.
  \item[rescur]
    is a vector that contains $F(t_n,y_{cur},\dot{y}_{cur})$. 
  \end{args}
}
{
  \id{lsolve} returns a positive value    
  for a recoverable error and a negative value for an             
  unrecoverable error. Success is indicated by a \id{0} return value.
}
{}

%------------------------------------------------------------------------------

\section{Performance monitoring routine}
The type definition of \id{lperf} is
\usfunction{lperf}
{
  int (*lperf)(IDAMem IDA\_mem, int perftask);
}
{
  The routine \id{lperf} is to monitor the performance of the linear solver.
}
{
  \begin{args}[IDA\_mem]
  \item[IDA\_mem]
    is the {\ida} memory pointer of type \id{IDAMem}.
  \item[perftask]
    is a task flag.  \id{perftask} = 0 means initialize needed counters.
    \id{perftask} = 1 means evaluate performance and issue warnings if needed.
  \end{args}
}
{
  The \id{lperf} return value is ignored.
}
{}

%------------------------------------------------------------------------------

\section{Memory deallocation routine}
The type definition of \id{lfree} is
\usfunction{lfree}
{
  void (*lfree)(IDAMem IDA\_mem);
}
{
  The routine \id{lfree} should free up any memory allocated by the linear
  solver.
}
{
  The argument \id{IDA\_mem} is the {\ida} memory pointer of type \id{IDAMem}.
}
{
  This routine has no return value.
}
{
  This routine is called once a problem has been completed and the 
  linear solver is no longer needed.
}