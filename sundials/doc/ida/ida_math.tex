%===================================================================================
\chapter{Mathematical Considerations}\label{s:math}
%===================================================================================

{\ida} solves the initial-value problem (IVP) for a DAE system of the
general form
\begin{equation}\label{e:DAE}
  F(t,y,\dot{y}) = 0 \, ,
  \quad y(t_0) = y_0 \, ,~ \dot{y}(t_0) = \dot{y}_0 \, ,
\end{equation}
where $y$, $\dot{y}$, and $F$ are vectors in ${\bf R}^N$, $t$ is the independent
variable, $\dot{y} = dy/dt$, and initial values $y_0$, $\dot{y}_0$ 
are given.  (Often $t$ is time, but it certainly need not be.)


%------------------------
\section{IVP solution}\label{ss:ivp_sol}
%------------------------

% Initial condition calculation
%------------------------------
Prior to integrating a DAE initial-value problem, an important requirement 
is that the pair of vectors $y_0$ and $\dot{y}_0$ are both initialized to
satisfy the DAE residual $F(t_0,y_0, \dot{y}_0) = 0$.
For a class of problems that includes so-called
semi-explicit index-one systems, {\ida} provides a routine that computes
consistent initial conditions from a user's initial guess~\cite{BHP:98}.  
For this, the user must identify sub-vectors of $y$
(not necessarily contiguous), denoted $y_d$ and $y_a$, which are its
differential and algebraic parts, respectively, such that $F$ depends
on $\dot{y}_d$ but not on any components of $\dot{y}_a$.  The assumption that
the system is ``index one'' means that for a given $t$ and $y_d$, the
system $F(t,y,\dot{y}) = 0$ defines $y_a$ uniquely.  In this case, a solver
within {\ida} computes $y_a$ and $\dot{y}_d$ at $t = t_0$, given $y_d$ and an
initial guess for $y_a$.  A second available option with this solver
also computes all of $y(t_0)$ given $\dot{y}(t_0)$; this is intended mainly
for quasi-steady-state problems, where $\dot{y}(t_0) = 0$ is given.
In both cases, {\ida} solves the system $F(t_0,y_0, \dot{y}_0) = 0$ for the
unknown components of $y_0$ and $\dot{y}_0$, using Newton iteration
augmented with a line search global strategy.  In doing this, it makes
use of the existing machinery that is to be used for solving the
linear systems during the integration, in combination with certain
tricks involving the step size (which is set artificially for this
calculation).
For problems that do not fall into either of these categories, the
user is responsible for passing consistent values, or risk failure in
the numerical integration.

% Integration method and nonlinear system
%----------------------------------------
The integration method used in {\ida} is the variable-order, variable-coefficient
BDF (Backward Differentiation Formula), in fixed-leading-coefficient
form~\cite{BCP:96}.
The method order ranges from 1 to 5, with the BDF of order $q$
given by the multistep formula
\begin{equation}\label{e:BDF}
  \sum_{i=0}^q \alpha_{n,i}y_{n-i} = h_n \dot{y}_n \, ,
\end{equation}
where $y_n$ and $\dot{y}_n$ are the computed approximations to $y(t_n)$
and $\dot{y}(t_n)$, respectively, and the step size is $h_n = t_n - t_{n-1}$.  
The coefficients $\alpha_{n,i}$ are uniquely determined by the order
$q$, and the history of the step sizes.  The application of the BDF
(\ref{e:BDF}) to the DAE system (\ref{e:DAE}) results in a nonlinear
algebraic system to be solved at each step:
\begin{equation}\label{e:DAE_nls}
  G(y_n) \equiv 
  F \left( t_n , \, y_n , \, 
    h_n^{-1} \sum_{i=0}^q \alpha_{n,i}y_{n-i} \right) = 0 \, .
\end{equation}
%
Regardless of the method options, the solution of the nonlinear system
(\ref{e:DAE_nls}) is accomplished with some form of Newton iteration.
This leads to a linear system for each Newton correction, of the form
\begin{equation}\label{e:DAE_Newtoncorr}
  J [y_{n(m+1)} - y_{n(m)}] = -G(y_{n(m)})  \, , 
\end{equation}
where $y_{n(m)}$ is the $m$-th approximation to $y_n$. 
%
Here $J$ is some approximation to the system Jacobian
\begin{equation}\label{e:DAE_Jacobian}
  J = \frac{\partial G}{\partial y}
  = \frac{\partial F}{\partial y} + 
  \alpha\frac{\partial F}{\partial \dot{y}} \, ,
\end{equation}
where $\alpha = \alpha_{n,0}/h_n$.  The scalar $\alpha$ changes 
whenever the step size or method order changes.

For the solution of the linear systems within the Newton corrections, 
{\ida} provides several choices, including the option of an user-supplied
linear solver module. The linear solver modules distributed with {\sundials}
are organized in three families, a {\em direct} family comprising direct linear 
solvers for dense or banded matrices, a {\em sparse} family comprising
direct linear solvers for matrices stored in compressed-sparse-column
format, and a {\em spils} family comprising
scaled preconditioned iterative (Krylov) linear solvers. 
%%
The methods offered through these modules are as follows:
\begin{itemize}
\item dense direct solvers, using either an internal implementation or 
  a Blas/Lapack implementation (serial or threaded vector modules only),
\item band direct solvers, using either an internal implementation or 
  a Blas/Lapack implementation (serial or threaded vector modules only),
\item sparse direct solvers, using either the KLU sparse solver
  library \cite{KLU_site}, or the PThreads-enabled SuperLU\_MT sparse
  solver library \cite{SuperLUMT_site} (serial or threaded vector modules only),
\item {\spgmr}, a scaled preconditioned GMRES (Generalized Minimal Residual method)
  solver without restarts,
\item {\spbcg}, a scaled preconditioned Bi-CGStab (Bi-Conjugate Gradient Stable
  method) solver, or
\item {\sptfqmr}, a scaled preconditioned TFQMR (Transpose-Free Quasi-Minimal
  Residual method) solver.
\end{itemize}
For large stiff systems, where direct methods are not feasible, the
combination of a BDF integrator and any of the preconditioned Krylov
methods ({\spgmr}, {\spbcg}, or {\sptfqmr}) yields a powerful tool
because it combines established methods for stiff integration,
nonlinear iteration, and Krylov (linear) iteration with a
problem-specific treatment of the dominant source of stiffness, in the
form of the user-supplied preconditioner matrix \cite{BrHi:89}. 
For the {\em spils} linear solvers,  preconditioning is allowed
only on the left (see \S\ref{s:preconditioning}).
%%
Note that the direct linear solvers (dense, band and sparse) can only be 
used with serial and threaded vector representations.

% WRMS Norm
%----------
\index{weighted root-mean-square norm|(}
In the process of controlling errors at various levels, {\ida} uses a
weighted root-mean-square norm, denoted $\|\cdot\|_{\mbox{\scriptsize WRMS}}$,
for all error-like quantities.  The multiplicative weights used are based on
the current solution and on the relative and absolute tolerances input by the
user, namely
\index{tolerances}
\begin{equation}\label{e:errwt}
 W_i = 1 / [ \mbox{\sc rtol} \cdot |y_i| + \mbox{\sc atol}_i ] \, .
\end{equation}
Because $1/W_i$ represents a tolerance in the component $y_i$, a vector
whose norm is 1 is regarded as ``small.''  For brevity, we will
usually drop the subscript WRMS on norms in what follows.
\index{weighted root-mean-square norm|)}

% Newton iteration
%-----------------
In the case of a direct linear solver (dense, band or sparse), the nonlinear 
iteration (\ref{e:DAE_Newtoncorr}) is a Modified Newton iteration, in
that the Jacobian $J$ is fixed (and usually out of date), with a coefficient 
$\bar\alpha$ in place of $\alpha$ in $J$. When using one of the Krylov methods
{\spgmr}, {\spbcg}, or {\sptfqmr} as the linear solver, the iteration is an
Inexact Newton iteration, using the current Jacobian (through matrix-free products
$Jv$), in which the linear residual $J\Delta y + G$ is nonzero but controlled.
The Jacobian matrix $J$ (direct cases) or preconditioner matrix $P$ 
({\spgmr}/{\spbcg}/{\sptfqmr} case) is updated when:
\begin{itemize}
\item starting the problem,
\item the value $\bar\alpha$ at the last update is such that
  $\alpha / {\bar\alpha} < 3/5$ or $\alpha / {\bar\alpha} > 5/3$, or
\item a non-fatal convergence failure occurred with an out-of-date $J$ or $P$.
\end{itemize}
The above strategy balances the high cost of frequent matrix evaluations
and preprocessing with the slow convergence due to infrequent updates.
To reduce storage costs on an update, Jacobian information is always
reevaluated from scratch.

We note that with the sparse direct solvers, the Jacobian {\em must}
be supplied by a user routine in compressed-sparse-column format, as
this is not approximated automatically within {\ida}.


% Newton convergence test
%------------------------
The stopping test for the Newton iteration
in {\ida} ensures that the iteration error $y_n - y_{n(m)}$ is small relative
to $y$ itself. For this, we estimate the linear convergence rate at all 
iterations $m>1$ as
\begin{equation*}
R = \left( \frac{\delta_m}{\delta_1} \right)^{\frac{1}{m-1}} \, , 
\end{equation*}
where the $\delta_m = y_{n(m)} - y_{n(m-1)}$ is the correction at
iteration $m=1,2,\ldots$. The Newton iteration is halted if $R>0.9$.
The convergence test at the $m$-th iteration is then
\begin{equation}\label{e:DAE_nls_test}
S \| \delta_m \| < 0.33 \, ,
\end{equation}
where $S = R/(R-1)$ whenever $m>1$ and $R\le 0.9$. The user has the
option of changing the constant in the convergence test from its default 
value of $0.33$.
%
The quantity $S$ is set to $S=20$ initially and whenever $J$ or $P$ is
updated, and it is reset to $S=100$ on a step with $\alpha \neq \bar\alpha$.
Note that at $m=1$, the convergence test (\ref{e:DAE_nls_test}) uses an old 
value for $S$. Therefore, at the first Newton iteration, we make an additional
test and stop the iteration if $\|\delta_1\| < 0.33 \cdot 10^{-4}$
(since such a $\delta_1$ is probably just noise and therefore not appropriate 
for use in evaluating $R$).
%
We allow only a small number (default value 4) of Newton iterations.
If convergence fails with $J$ or $P$ current, 
we are forced to reduce the step size $h_n$, and we replace $h_n$ by $h_n/4$.
The integration is halted after a preset number (default value 10)
of convergence failures. Both the maximum allowable Newton iterations
and the maximum nonlinear convergence failures can be changed by the user
from their default values.

When {\spgmr}, {\spbcg}, or {\sptfqmr} is used to solve the linear system, to
minimize the effect of linear iteration errors on the nonlinear and local integration
error controls, we require the preconditioned linear residual to be small relative to
the allowed error in the Newton iteration, i.e., 
$\| P^{-1}(Jx+G) \| < 0.05 \cdot 0.33$.
The safety factor $0.05$ can be changed by the user.

% Jacobian DQ approximations
%---------------------------
In the direct linear solver 
cases, the Jacobian $J$ defined in (\ref{e:DAE_Jacobian}) 
can be either supplied by the user or have {\ida} compute one internally 
by difference quotients. In the latter case, we use the approximation
\begin{gather*}
  J_{ij} = [F_i(t,y+\sigma_j e_j,\dot{y}+\alpha\sigma_j e_j) - 
            F_i(t,y,\dot{y})]/\sigma_j \, , \text{ with}\\
  \sigma_j = \sqrt{U} \max \left\{ |y_j|, |h\dot{y}_j|,1/W_j \right\}
             \mbox{sign}(h \dot{y}_j) \, ,
\end{gather*}
where $U$ is the unit roundoff, $h$ is the current step size, and $W_j$ is 
the error weight for the component $y_j$ defined by (\ref{e:errwt}).
In the {\spgmr}/{\spbcg}/{\sptfqmr} case, if a routine for $Jv$ is not
supplied, such products are approximated by
\begin{equation*}
Jv = [F(t,y+\sigma v,\dot{y}+\alpha\sigma v) - F(t,y,\dot{y})]/\sigma \, ,
\end{equation*}
where the increment $\sigma$ is $1/\|v\|$.  As an option, the user can
specify a constant factor that is inserted into this expression for $\sigma$.

% Error control
%--------------
During the course of integrating the system, {\ida} computes an estimate
of the local truncation error, LTE, at the $n$-th time step, and
requires this to satisfy the inequality
\begin{equation*}
  \| \mbox{LTE} \|_{\mbox{\scriptsize WRMS}} \leq 1 \, .               
\end{equation*}
Asymptotically, LTE varies as $h^{q+1}$ at step size $h$ and order $q$, as
does the predictor-corrector difference $\Delta_n \equiv y_n-y_{n(0)}$.  
Thus there is a constant $C$ such that
\[ \mbox{LTE} = C \Delta_n + O(h^{q+2}) \, , \]
and so the norm of LTE is estimated as $|C| \cdot \|\Delta_n\|$.
In addition, {\ida} requires that the error in the associated polynomial
interpolant over the current step be bounded by 1 in norm.  The
leading term of the norm of this error is bounded by
$\bar{C} \|\Delta_n\|$ for another constant $\bar{C}$.  Thus the local
error test in {\ida} is
\begin{equation}\label{lerrtest}
   \max\{ |C|, \bar{C} \} \|\Delta_n\| \leq 1 \, .
\end{equation}
A user option is available by which the algebraic components of the
error vector are omitted from the test (\ref{lerrtest}), if these have
been so identified.

% Step/order selection
%---------------------
In {\ida}, the local error test is tightly coupled with the logic for
selecting the step size and order.  First, there is an initial phase
that is treated specially; for the first few steps, the step size is
doubled and the order raised (from its initial value of 1) on every
step, until (a) the local error test (\ref{lerrtest}) fails, (b) the
order is reduced (by the rules given below), or (c) the order reaches
5 (the maximum).  For step and order selection on the general step,
{\ida} uses a different set of local error estimates, based on the
asymptotic behavior of the local error in the case of fixed step sizes.
At each of the orders $q'$ equal to $q$, $q-1$ (if $q > 1$), $q-2$ (if
$q > 2$), or $q+1$ (if $q < 5$), there are constants $C(q')$ such that
the norm of the local truncation error at order $q'$ satisfies
\[ \mbox{LTE}(q') = C(q') \| \phi(q'+1) \| + O(h^{q'+2}) \, , \]
where $\phi(k)$ is a modified divided difference of order $k$ that is
retained by {\ida} (and behaves asymptotically as $h^k$).
Thus the local truncation errors are estimated as
ELTE$(q') = C(q')\|\phi(q'+1)\|$ to select step sizes.  But the choice
of order in {\ida} is based on the requirement that the scaled derivative
norms, $\|h^k y^{(k)}\|$, are monotonically decreasing with $k$, for
$k$ near $q$.  These norms are again estimated using the $\phi(k)$,
and in fact
\[ \|h^{q'+1} y^{(q'+1)}\| \approx T(q') \equiv (q'+1) \mbox{ELTE}(q') \, . \]
The step/order selection begins with a test for monotonicity that is
made even {\em before} the local error test is performed.  Namely,
the order is reset to $q' = q-1$ if (a) $q=2$ and $T(1)\leq T(2)/2$,
or (b) $q > 2$ and $\max\{T(q-1),T(q-2)\} \leq T(q)$; otherwise 
$q' = q$.  Next the local error test (\ref{lerrtest}) is performed,
and if it fails, the step is redone at order $q\leftarrow q'$ and a
new step size $h'$.  The latter is based on the $h^{q+1}$ asymptotic
behavior of $\mbox{ELTE}(q)$, and, with safety factors, is given by
\[ \eta = h'/h = 0.9/[2 \, \mbox{ELTE}(q)]^{1/(q+1)} \, . \]
The value of $\eta$ is adjusted so that $0.25 \leq \eta \leq 0.9$
before setting $h \leftarrow h' = \eta h$.  If the local error test
fails a second time, {\ida} uses $\eta = 0.25$, and on the third
and subsequent failures it uses $q = 1$ and $\eta = 0.25$.  After
10 failures, {\ida} returns with a give-up message.

As soon as the local error test has passed, the step and order for the
next step may be adjusted.  No such change is made if $q' = q-1$ from
the prior test, if $q = 5$, or if $q$ was increased on the previous
step.  Otherwise, if the last $q+1$ steps were taken at a constant
order $q < 5$ and a constant step size, {\ida} considers raising the order
to $q+1$.  The logic is as follows: (a) If $q = 1$, then reset $q = 2$
if $T(2) < T(1)/2$.  (b) If $q > 1$ then 
\begin{itemize}
\item reset $q \leftarrow q-1$ if $T(q-1) \leq \min\{T(q),T(q+1)\}$;
\item else reset $q \leftarrow q+1$ if $T(q+1) < T(q)$;
\item leave $q$ unchanged otherwise $[$then $T(q-1) > T(q) \leq T(q+1)]$.
\end{itemize}
In any case, the new step size $h'$ is set much as before:
\[ \eta = h'/h = 1/[2 \, \mbox{ELTE}(q)]^{1/(q+1)} \, . \]
The value of $\eta$ is adjusted such that (a) if $\eta > 2$, $\eta$ is
reset to 2; (b) if $\eta \leq 1$, $\eta$ is restricted to 
$0.5 \leq \eta \leq 0.9$; and (c) if $1 < \eta < 2$ we use $\eta = 1$.
Finally $h$ is reset to $h' = \eta h$.  Thus we do not increase the
step size unless it can be doubled.  See \cite{BCP:96} for details.

% Additional constraints on y components
%---------------------------------------
{\ida} permits the user to impose optional inequality constraints on individual 
components of the solution vector $y$. Any of the following four constraints 
can be imposed: $y_i > 0$, $y_i < 0$, $y_i \geq 0$, or $y_i \leq 0$. 
The constraint satisfaction is tested after a successful nonlinear system solution. 
If any constraint fails, we declare a convergence failure of the Newton iteration 
and reduce the step size. Rather than cutting the step size by some arbitrary factor, 
{\ida} estimates a new step size $h'$ using a linear approximation of the components 
in $y$ that failed the constraint test (including a safety factor of $0.9$ to 
cover the strict inequality case). These additional constraints are also imposed
during the calculation of consistent initial conditions.

Normally, {\ida} takes steps until a user-defined output value $t =
t_{\mbox{\scriptsize out}}$ is overtaken, and then computes
$y(t_{\mbox{\scriptsize out}})$ by interpolation.  However, a
``one step'' mode option is available, where control returns to the
calling program after each step.  There are also options to force {\ida}
not to integrate past a given stopping point $t = t_{\mbox{\scriptsize
stop}}$.


%------------------------
\section{Preconditioning}\label{s:preconditioning}
%------------------------
\index{preconditioning!advice on|(}
When using a Newton method to solve the nonlinear system (\ref{e:DAE_Newtoncorr}),
{\ida} makes repeated use of a linear solver to solve linear systems of the form
$J \Delta y = - G$.
If this linear system solve is done with one of the scaled preconditioned iterative 
linear solvers, these solvers are rarely successful if used without preconditioning;
it is generally necessary to precondition the system in order to obtain acceptable efficiency.  
A system $A x = b$ can be preconditioned on the left, on the right,
or on both sides. The Krylov method is then applied to a system with the matrix $P^{-1}A$, 
or $AP^{-1}$, or $P_L^{-1} A P_R^{-1}$, instead of $A$.  
However, within {\ida}, preconditioning is allowed {\em only} on the left,
so that the iterative method is applied to systems $(P^{-1}J)\Delta y = -P^{-1}G$.  
Left preconditioning is required to make the norm of the linear residual in the Newton 
iteration meaningful; in general, $\| J \Delta y + G \|$ is meaningless, since the 
weights used in the WRMS-norm correspond to $y$.

In order to improve the convergence of the Krylov iteration, the preconditioner matrix 
$P$ should in some sense approximate the system matrix $A$.  
Yet at the same time, in order to be cost-effective, the matrix $P$ should
be reasonably efficient to evaluate and solve.  Finding a good point
in this tradeoff between rapid convergence and low cost can be very
difficult.  Good choices are often problem-dependent (for example, see
\cite{BrHi:89} for an extensive study of preconditioners for
reaction-transport systems).

Typical preconditioners used with {\ida} are based on approximations to the Newton 
iteration matrix of the systems involved; in other words, 
$P \approx \frac{\partial F}{\partial y} + \alpha\frac{\partial F}{\partial \dot{y}}$,
where $\alpha$ is a scalar inversely proportional to the integration step size $h$.
Because the Krylov iteration occurs within a Newton iteration and further
also within a time integration, and since each of these iterations has
its own test for convergence, the preconditioner may use a very crude
approximation, as long as it captures the dominant numerical
feature(s) of the system.  We have found that the combination of a
preconditioner with the Newton-Krylov iteration, using even a fairly
poor approximation to the Jacobian, can be surprisingly superior to
using the same matrix without Krylov acceleration (i.e., a modified
Newton iteration), as well as to using the Newton-Krylov method with
no preconditioning.
\index{preconditioning!advice on|)}

%------------------------
\section{Rootfinding}\label{ss:rootfinding}
%------------------------

\index{Rootfinding}
The {\ida} solver has been augmented to include a rootfinding
feature.  This means that, while integrating the Initial Value Problem
(\ref{e:DAE}), {\ida} can also find the roots of a set of user-defined
functions $g_i(t,y,\dot{y})$ that depend on $t$, the solution vector 
$y = y(t)$, and its $t-$derivative $\dot{y}(t)$.  The number of these root
functions is arbitrary, and if more than one $g_i$ is found to have a
root in any given interval, the various root locations are found and
reported in the order that they occur on the $t$ axis, in the
direction of integration.

Generally, this rootfinding feature finds only roots of odd
multiplicity, corresponding to changes in sign of $g_i(t,y(t),\dot{y}(t))$,
denoted $g_i(t)$ for short.  If a user root function has a root of
even multiplicity (no sign change), it will probably be missed by
{\ida}.  If such a root is desired, the user should reformulate the
root function so that it changes sign at the desired root.

The basic scheme used is to check for sign changes of any $g_i(t)$ over
each time step taken, and then (when a sign change is found) to home
in on the root (or roots) with a modified secant method \cite{HeSh:80}.  
In addition, each time $g$ is computed, {\ida} checks to see if 
$g_i(t) = 0$ exactly, and if so it reports this as a root.  However,
if an exact zero of any $g_i$ is found at a point $t$, {\ida}
computes $g$ at $t + \delta$ for a small increment $\delta$, slightly
further in the direction of integration, and if any $g_i(t + \delta)=0$ 
also, {\ida} stops and reports an error.  This way, each time
{\ida} takes a time step, it is guaranteed that the values of all
$g_i$ are nonzero at some past value of $t$, beyond which a search for
roots is to be done.

At any given time in the course of the time-stepping, after suitable
checking and adjusting has been done, {\ida} has an interval
$(t_{lo},t_{hi}]$ in which roots of the $g_i(t)$ are to be sought, such
that $t_{hi}$ is further ahead in the direction of integration, and
all $g_i(t_{lo}) \neq 0$.  The endpoint $t_{hi}$ is either $t_n$,
the end of the time step last taken, or the next requested output time
$t_{\mbox{\scriptsize out}}$ if this comes sooner.  The endpoint
$t_{lo}$ is either $t_{n-1}$, or the last output time
$t_{\mbox{\scriptsize out}}$ (if this occurred within the last
step), or the last root location (if a root was just located within
this step), possibly adjusted slightly toward $t_n$ if an exact zero
was found.  The algorithm checks $g$ at $t_{hi}$ for zeros and for
sign changes in $(t_{lo},t_{hi})$.  If no sign changes are found, then
either a root is reported (if some $g_i(t_{hi}) = 0$) or we proceed to
the next time interval (starting at $t_{hi}$).  If one or more sign
changes were found, then a loop is entered to locate the root to
within a rather tight tolerance, given by
\[ \tau = 100 * U * (|t_n| + |h|)~~~ (U = \mbox{unit roundoff}) ~. \]
Whenever sign changes are seen in two or more root functions, the one
deemed most likely to have its root occur first is the one with the
largest value of $|g_i(t_{hi})|/|g_i(t_{hi}) - g_i(t_{lo})|$,
corresponding to the closest to $t_{lo}$ of the secant method values.
At each pass through the loop, a new value $t_{mid}$ is set, strictly
within the search interval, and the values of $g_i(t_{mid})$ are
checked.  Then either $t_{lo}$ or $t_{hi}$ is reset to $t_{mid}$
according to which subinterval is found to have the sign change.  If
there is none in $(t_{lo},t_{mid})$ but some $g_i(t_{mid}) = 0$, then
that root is reported.  The loop continues until 
$|t_{hi}-t_{lo}| < \tau$, and then the reported root location is
$t_{hi}$.

In the loop to locate the root of $g_i(t)$, the formula for $t_{mid}$
is
\[ t_{mid} = t_{hi} - (t_{hi} - t_{lo})
             g_i(t_{hi}) / [g_i(t_{hi}) - \alpha g_i(t_{lo})] ~, \] 
where $\alpha$ a weight parameter.  On the first two passes through
the loop, $\alpha$ is set to $1$, making $t_{mid}$ the secant method
value.  Thereafter, $\alpha$ is reset according to the side of the
subinterval (low vs high, i.e. toward $t_{lo}$ vs toward $t_{hi}$)
in which the sign change was found in the previous two passes.  If the
two sides were opposite, $\alpha$ is set to 1.  If the two sides were
the same, $\alpha$ is halved (if on the low side) or doubled (if on
the high side).  The value of $t_{mid}$ is closer to $t_{lo}$ when
$\alpha < 1$ and closer to $t_{hi}$ when $\alpha > 1$.  If the above
value of $t_{mid}$ is within $\tau/2$ of $t_{lo}$ or $t_{hi}$, it is
adjusted inward, such that its fractional distance from the endpoint
(relative to the interval size) is between .1 and .5 (.5 being the
midpoint), and the actual distance from the endpoint is at least
$\tau/2$.
