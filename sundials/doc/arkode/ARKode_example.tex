% Generated by Sphinx.
\def\sphinxdocclass{report}
\documentclass[letterpaper,10pt,english]{sphinxmanual}
\usepackage[utf8]{inputenc}
\DeclareUnicodeCharacter{00A0}{\nobreakspace}
\usepackage{cmap}
\usepackage[T1]{fontenc}
\usepackage{babel}
\usepackage{times}
\usepackage[Bjarne]{fncychap}
\usepackage{longtable}
\usepackage{sphinx}
\usepackage{multirow}


\title{ARKode Example Documentation}
\date{November 18, 2013}
\release{1.0}
\author{Daniel R. Reynolds}
\newcommand{\sphinxlogo}{}
\renewcommand{\releasename}{Release}
\makeindex

\makeatletter
\def\PYG@reset{\let\PYG@it=\relax \let\PYG@bf=\relax%
    \let\PYG@ul=\relax \let\PYG@tc=\relax%
    \let\PYG@bc=\relax \let\PYG@ff=\relax}
\def\PYG@tok#1{\csname PYG@tok@#1\endcsname}
\def\PYG@toks#1+{\ifx\relax#1\empty\else%
    \PYG@tok{#1}\expandafter\PYG@toks\fi}
\def\PYG@do#1{\PYG@bc{\PYG@tc{\PYG@ul{%
    \PYG@it{\PYG@bf{\PYG@ff{#1}}}}}}}
\def\PYG#1#2{\PYG@reset\PYG@toks#1+\relax+\PYG@do{#2}}

\def\PYG@tok@gd{\def\PYG@tc##1{\textcolor[rgb]{0.63,0.00,0.00}{##1}}}
\def\PYG@tok@gu{\let\PYG@bf=\textbf\def\PYG@tc##1{\textcolor[rgb]{0.50,0.00,0.50}{##1}}}
\def\PYG@tok@gt{\def\PYG@tc##1{\textcolor[rgb]{0.00,0.25,0.82}{##1}}}
\def\PYG@tok@gs{\let\PYG@bf=\textbf}
\def\PYG@tok@gr{\def\PYG@tc##1{\textcolor[rgb]{1.00,0.00,0.00}{##1}}}
\def\PYG@tok@cm{\let\PYG@it=\textit\def\PYG@tc##1{\textcolor[rgb]{0.25,0.50,0.56}{##1}}}
\def\PYG@tok@vg{\def\PYG@tc##1{\textcolor[rgb]{0.73,0.38,0.84}{##1}}}
\def\PYG@tok@m{\def\PYG@tc##1{\textcolor[rgb]{0.13,0.50,0.31}{##1}}}
\def\PYG@tok@mh{\def\PYG@tc##1{\textcolor[rgb]{0.13,0.50,0.31}{##1}}}
\def\PYG@tok@cs{\def\PYG@tc##1{\textcolor[rgb]{0.25,0.50,0.56}{##1}}\def\PYG@bc##1{\colorbox[rgb]{1.00,0.94,0.94}{##1}}}
\def\PYG@tok@ge{\let\PYG@it=\textit}
\def\PYG@tok@vc{\def\PYG@tc##1{\textcolor[rgb]{0.73,0.38,0.84}{##1}}}
\def\PYG@tok@il{\def\PYG@tc##1{\textcolor[rgb]{0.13,0.50,0.31}{##1}}}
\def\PYG@tok@go{\def\PYG@tc##1{\textcolor[rgb]{0.19,0.19,0.19}{##1}}}
\def\PYG@tok@cp{\def\PYG@tc##1{\textcolor[rgb]{0.00,0.44,0.13}{##1}}}
\def\PYG@tok@gi{\def\PYG@tc##1{\textcolor[rgb]{0.00,0.63,0.00}{##1}}}
\def\PYG@tok@gh{\let\PYG@bf=\textbf\def\PYG@tc##1{\textcolor[rgb]{0.00,0.00,0.50}{##1}}}
\def\PYG@tok@ni{\let\PYG@bf=\textbf\def\PYG@tc##1{\textcolor[rgb]{0.84,0.33,0.22}{##1}}}
\def\PYG@tok@nl{\let\PYG@bf=\textbf\def\PYG@tc##1{\textcolor[rgb]{0.00,0.13,0.44}{##1}}}
\def\PYG@tok@nn{\let\PYG@bf=\textbf\def\PYG@tc##1{\textcolor[rgb]{0.05,0.52,0.71}{##1}}}
\def\PYG@tok@no{\def\PYG@tc##1{\textcolor[rgb]{0.38,0.68,0.84}{##1}}}
\def\PYG@tok@na{\def\PYG@tc##1{\textcolor[rgb]{0.25,0.44,0.63}{##1}}}
\def\PYG@tok@nb{\def\PYG@tc##1{\textcolor[rgb]{0.00,0.44,0.13}{##1}}}
\def\PYG@tok@nc{\let\PYG@bf=\textbf\def\PYG@tc##1{\textcolor[rgb]{0.05,0.52,0.71}{##1}}}
\def\PYG@tok@nd{\let\PYG@bf=\textbf\def\PYG@tc##1{\textcolor[rgb]{0.33,0.33,0.33}{##1}}}
\def\PYG@tok@ne{\def\PYG@tc##1{\textcolor[rgb]{0.00,0.44,0.13}{##1}}}
\def\PYG@tok@nf{\def\PYG@tc##1{\textcolor[rgb]{0.02,0.16,0.49}{##1}}}
\def\PYG@tok@si{\let\PYG@it=\textit\def\PYG@tc##1{\textcolor[rgb]{0.44,0.63,0.82}{##1}}}
\def\PYG@tok@s2{\def\PYG@tc##1{\textcolor[rgb]{0.25,0.44,0.63}{##1}}}
\def\PYG@tok@vi{\def\PYG@tc##1{\textcolor[rgb]{0.73,0.38,0.84}{##1}}}
\def\PYG@tok@nt{\let\PYG@bf=\textbf\def\PYG@tc##1{\textcolor[rgb]{0.02,0.16,0.45}{##1}}}
\def\PYG@tok@nv{\def\PYG@tc##1{\textcolor[rgb]{0.73,0.38,0.84}{##1}}}
\def\PYG@tok@s1{\def\PYG@tc##1{\textcolor[rgb]{0.25,0.44,0.63}{##1}}}
\def\PYG@tok@gp{\let\PYG@bf=\textbf\def\PYG@tc##1{\textcolor[rgb]{0.78,0.36,0.04}{##1}}}
\def\PYG@tok@sh{\def\PYG@tc##1{\textcolor[rgb]{0.25,0.44,0.63}{##1}}}
\def\PYG@tok@ow{\let\PYG@bf=\textbf\def\PYG@tc##1{\textcolor[rgb]{0.00,0.44,0.13}{##1}}}
\def\PYG@tok@sx{\def\PYG@tc##1{\textcolor[rgb]{0.78,0.36,0.04}{##1}}}
\def\PYG@tok@bp{\def\PYG@tc##1{\textcolor[rgb]{0.00,0.44,0.13}{##1}}}
\def\PYG@tok@c1{\let\PYG@it=\textit\def\PYG@tc##1{\textcolor[rgb]{0.25,0.50,0.56}{##1}}}
\def\PYG@tok@kc{\let\PYG@bf=\textbf\def\PYG@tc##1{\textcolor[rgb]{0.00,0.44,0.13}{##1}}}
\def\PYG@tok@c{\let\PYG@it=\textit\def\PYG@tc##1{\textcolor[rgb]{0.25,0.50,0.56}{##1}}}
\def\PYG@tok@mf{\def\PYG@tc##1{\textcolor[rgb]{0.13,0.50,0.31}{##1}}}
\def\PYG@tok@err{\def\PYG@bc##1{\fcolorbox[rgb]{1.00,0.00,0.00}{1,1,1}{##1}}}
\def\PYG@tok@kd{\let\PYG@bf=\textbf\def\PYG@tc##1{\textcolor[rgb]{0.00,0.44,0.13}{##1}}}
\def\PYG@tok@ss{\def\PYG@tc##1{\textcolor[rgb]{0.32,0.47,0.09}{##1}}}
\def\PYG@tok@sr{\def\PYG@tc##1{\textcolor[rgb]{0.14,0.33,0.53}{##1}}}
\def\PYG@tok@mo{\def\PYG@tc##1{\textcolor[rgb]{0.13,0.50,0.31}{##1}}}
\def\PYG@tok@mi{\def\PYG@tc##1{\textcolor[rgb]{0.13,0.50,0.31}{##1}}}
\def\PYG@tok@kn{\let\PYG@bf=\textbf\def\PYG@tc##1{\textcolor[rgb]{0.00,0.44,0.13}{##1}}}
\def\PYG@tok@o{\def\PYG@tc##1{\textcolor[rgb]{0.40,0.40,0.40}{##1}}}
\def\PYG@tok@kr{\let\PYG@bf=\textbf\def\PYG@tc##1{\textcolor[rgb]{0.00,0.44,0.13}{##1}}}
\def\PYG@tok@s{\def\PYG@tc##1{\textcolor[rgb]{0.25,0.44,0.63}{##1}}}
\def\PYG@tok@kp{\def\PYG@tc##1{\textcolor[rgb]{0.00,0.44,0.13}{##1}}}
\def\PYG@tok@w{\def\PYG@tc##1{\textcolor[rgb]{0.73,0.73,0.73}{##1}}}
\def\PYG@tok@kt{\def\PYG@tc##1{\textcolor[rgb]{0.56,0.13,0.00}{##1}}}
\def\PYG@tok@sc{\def\PYG@tc##1{\textcolor[rgb]{0.25,0.44,0.63}{##1}}}
\def\PYG@tok@sb{\def\PYG@tc##1{\textcolor[rgb]{0.25,0.44,0.63}{##1}}}
\def\PYG@tok@k{\let\PYG@bf=\textbf\def\PYG@tc##1{\textcolor[rgb]{0.00,0.44,0.13}{##1}}}
\def\PYG@tok@se{\let\PYG@bf=\textbf\def\PYG@tc##1{\textcolor[rgb]{0.25,0.44,0.63}{##1}}}
\def\PYG@tok@sd{\let\PYG@it=\textit\def\PYG@tc##1{\textcolor[rgb]{0.25,0.44,0.63}{##1}}}

\def\PYGZbs{\char`\\}
\def\PYGZus{\char`\_}
\def\PYGZob{\char`\{}
\def\PYGZcb{\char`\}}
\def\PYGZca{\char`\^}
\def\PYGZsh{\char`\#}
\def\PYGZpc{\char`\%}
\def\PYGZdl{\char`\$}
\def\PYGZti{\char`\~}
% for compatibility with earlier versions
\def\PYGZat{@}
\def\PYGZlb{[}
\def\PYGZrb{]}
\makeatother

\begin{document}

\maketitle
\tableofcontents
\phantomsection\label{index::doc}


This is the documentation for the ARKode examples.  ARKode is an
adaptive step time integration package for stiff, nonstiff and
multi-rate systems of ordinary differential equations (ODEs).
The ARKode solver is a component of the \href{https://computation.llnl.gov/casc/sundials/main.html}{SUNDIALS} suite of
nonlinear and differential/algebraic equation solvers. It is designed
to have a similar user experience to the \href{https://computation.llnl.gov/casc/sundials/description/description.html\#descr\_cvode}{CVODE}
solver, with user modes to allow adaptive integration to specified
output times, return after each internal step and root-finding
capabilities, for calculations both in serial and parallel (via
MPI). The default integration and solver options should apply to most
users, though complete control over all internal parameters and time
adaptivity algorithms is enabled through optional interface routines.

ARKode is developed by \href{http://www.smu.edu}{Southern Methodist University}, with support by the \href{http://www.doe.gov}{US Department of Energy} through the \href{http://www.fastmath-scidac.org/}{FASTMath} SciDAC Institute, under subcontract
B598130 from \href{http://www.llnl.gov}{Lawrence Livermore National Laboratory}.

Along with the ARKode solver, we have created a suite of example
problems demonstrating its usage on applications written in C, C++ and
Fortran 77 and Fortran 90.  These examples demonstrate a large variety
of ARKode solver options, including explicit, implicit and ImEx
solvers, root-finding, Newton and fixed-point nonlinear solvers,
direct and iterative linear solvers, adaptive resize capabilities, and
the Fortran solver interface.  While these examples are not an
exhaustive set of all possible usage scenarios, they are designed to
show a variety of exemplars, and can be used as templates for new
problems using ARKode's solvers.

The following table summarizes the salient features of each of the
following example problems.  Each example is designed to be relatively
self-contained, so that you need only study and/or emulate the problem
that is most closely related to your own.

\begin{tabulary}{\linewidth}{|L|L|L|L|L|L|L|}
\hline
\textbf{\relax 
Problem
} & \textbf{\relax 
Integrator
} & \textbf{\relax 
Nonlinear Solver
} & \textbf{\relax 
Linear Solver
} & \textbf{\relax 
Size
} & \textbf{\relax 
Language
} & \textbf{\relax 
Extras
}\\\hline

{\hyperref[ark_analytic:ark-analytic]{\emph{ark\_analytic}}}
 & 
DIRK
 & 
Newton
 & 
Dense
 & 
1
 & 
C
 & 
Analytical solution, variable stiffness
\\\hline

{\hyperref[ark_analytic_nonlin:ark-analytic-nonlin]{\emph{ark\_analytic\_nonlin}}}
 & 
ERK
 & 
N.A.
 & 
N.A.
 & 
1
 & 
C
 & 
Nonlinear, analytical solution
\\\hline

{\hyperref[ark_analytic_sys:ark-analytic-sys]{\emph{ark\_analytic\_sys}}}
 & 
DIRK
 & 
Newton
 & 
Dense
 & 
3
 & 
C++
 & 
ODE system, analytical solution, variable stiffness
\\\hline

{\hyperref[ark_brusselator:ark-brusselator]{\emph{ark\_brusselator}}}
 & 
DIRK
 & 
Newton
 & 
Dense
 & 
3
 & 
C
 & 
Stiff, nonlinear, ODE system, ``standard'' test problem
\\\hline

{\hyperref[ark_bruss:ark-bruss]{\emph{ark\_bruss}}}
 & 
ARK
 & 
Newton
 & 
Dense
 & 
3
 & 
F90
 & 
Stiff, nonlinear, ODE system, ``standard'' test problem
\\\hline

{\hyperref[ark_robertson:ark-robertson]{\emph{ark\_robertson}}}
 & 
DIRK
 & 
Newton
 & 
Dense
 & 
3
 & 
C
 & 
Stiff, nonlinear, ODE system, ``standard'' test problem
\\\hline

{\hyperref[ark_robertson_root:ark-robertson-root]{\emph{ark\_robertson\_root}}}
 & 
DIRK
 & 
Newton
 & 
Dense
 & 
3
 & 
C
 & 
Utilizes rootfinding capabilities
\\\hline

{\hyperref[ark_brusselator1D:ark-brusselator1d]{\emph{ark\_brusselator1D}}}
 & 
DIRK
 & 
Newton
 & 
Band
 & 
3N
 & 
C
 & 
Stiff, nonlinear, reaction-diffusion PDE system
\\\hline

{\hyperref[ark_heat1D:ark-heat1d]{\emph{ark\_heat1D}}}
 & 
DIRK
 & 
Newton
 & 
PCG
 & 
N
 & 
C
 & 
Stiff, linear, diffusion PDE, iterative linear solver
\\\hline

{\hyperref[ark_heat2D:ark-heat2d]{\emph{ark\_heat2D}}}
 & 
DIRK
 & 
Newton
 & 
PCG
 & 
$nx*ny$
 & 
C++
 & 
Parallel, stiff, linear, diffusion PDE, iterative linear solver
\\\hline

{\hyperref[ark_KrylovDemo_prec:ark-krylovdemo-prec]{\emph{ark\_KrylovDemo\_prec}}}
 & 
DIRK
 & 
Newton
 & 
SPGMR
 & 
216
 & 
C
 & 
Stiff, nonlinear, rx-diff PDE system, different preconditioners
\\\hline

{\hyperref[ark_brusselator_fp:ark-brusselator-fp]{\emph{ark\_brusselator\_fp}}}
 & 
ARK
 & 
Fixed-point
 & 
N.A.
 & 
3
 & 
C
 & 
Stiff, nonlinear, ODE system
\\\hline

{\hyperref[ark_diurnal_kry_bbd_p:ark-diurnal-kry-bbd-p]{\emph{ark\_diurnal\_kry\_bbd\_p}}}
 & 
DIRK
 & 
Newton
 & 
SPGMR
 & 
200
 & 
C
 & 
Stiff, nonlinear, PDE system, parallel, BBD preconditioner
\\\hline

{\hyperref[ark_diurnal_kry_p:ark-diurnal-kry-p]{\emph{ark\_diurnal\_kry\_p}}}
 & 
DIRK
 & 
Newton
 & 
SPGMR
 & 
200
 & 
C
 & 
Stiff, nonlinear, PDE system, parallel, block-diagonal precond.
\\\hline

{\hyperref[ark_heat1D_adapt:ark-heat1d-adapt]{\emph{ark\_heat1D\_adapt}}}
 & 
DIRK
 & 
Newton
 & 
PCG
 & 
(dynamic)
 & 
C
 & 
Stiff, linear, diffusion, PCG solver, adaptive vector resizing
\\\hline

{\hyperref[fark_diag_kry_bbd_p:fark-diag-kry-bbd-p]{\emph{fark\_diag\_kry\_bbd\_p}}}
 & 
DIRK
 & 
Newton
 & 
SPGMR
 & 
10*NProcs
 & 
F77
 & 
Stiff, linear, diagonal ODE system, BBD preconditioner
\\\hline

{\hyperref[fark_diag_non_p:fark-diag-non-p]{\emph{fark\_diag\_non\_p}}}
 & 
ERK
 & 
N.A.
 & 
N.A.
 & 
10*NProcs
 & 
F77
 & 
Nonstiff, linear, diagonal ODE system
\\\hline

{\hyperref[fark_diurnal_kry_bp:fark-diurnal-kry-bp]{\emph{fark\_diurnal\_kry\_bp}}}
 & 
DIRK
 & 
Newton
 & 
SPGMR
 & 
10
 & 
F77
 & 
Stiff, nonlinear, PDE system, banded preconditioner
\\\hline

{\hyperref[fark_heat2D:fark-heat2d]{\emph{fark\_heat2D}}}
 & 
DIRK
 & 
Newton
 & 
PCG
 & 
$nx*ny$
 & 
F90
 & 
Parallel, stiff, linear, diffusion PDE, iterative linear solver
\\\hline

{\hyperref[fark_roberts_dnsL:fark-roberts-dnsl]{\emph{fark\_roberts\_dnsL}}}
 & 
DIRK
 & 
Newton
 & 
Dense
 & 
3
 & 
F77
 & 
Stiff, nonlinear, ODE system, LAPACK dense solver, rootfinding
\\\hline
\end{tabulary}


Further details on each of the above-listed examples, including both
source code and plots of the computed results, are provided in the
following sub-sections:


\chapter{ark\_analytic}
\label{ark_analytic:ark-analytic}\label{ark_analytic::doc}\label{ark_analytic:arkode-example-documentation}\label{ark_analytic:id1}
This is a very simple C example that merely shows how to use the
ARKode solver interface.

The problem is that of a scalar-valued initial value problem (IVP)
that is linear in the dependent variable $y$, but nonlinear in
the independent variable $t$:
\begin{gather}
\begin{split}\frac{dy}{dt} = \lambda y + \frac{1}{1+t^2} - \lambda \arctan(t),\end{split}\notag
\end{gather}
where $0\le t\le 10$ and $y(0)=0$.  The stiffness of the
problem may be tuned via the parameter $\lambda$, which is
specified (along with the relative and absolute tolerances,
$rtol$ and $atol$) in the input file
\code{input\_analytic.txt}.  The value of $\lambda$ must be negative
to result in a well-posed problem; for values with magnitude larger
than 100 or so the problem becomes quite stiff.  In the provided input
file, we choose $\lambda=-100$ and tolerances
$rtol=10^{-6}$ and $atol=10^{-10}$.    After each unit
time interval, the solution is output to the screen.


\section{Numerical method}
\label{ark_analytic:numerical-method}
The example routine solves this problem using a diagonally-implicit
Runge-Kutta method.  Each stage is solved using the built-in modified
Newton iteration, but since the ODE is linear in $y$ these
should only require a single iteration per stage.  Internally, Newton
will use the ARKDENSE dense linear solver, which in the case of this
scalar-valued problem is just division.  The example file contains
functions to evaluate both $f(t,y)$ and $J(t,y)=\lambda$.

Aside from the input tolerance values, this problem uses only the
default parameters for the ARKode solver.


\section{Solutions}
\label{ark_analytic:solutions}
This problem is included both as a simple example, but also because it
has an analytical solution, $y(t) = \arctan(t)$.  As seen in the
plots below, the computed solution tracks the analytical solution
quite well (left), and results in errors below those specified by the input
error tolerances (right).

\includegraphics[width=0.450\linewidth]{plot-ark_analytic.png}

\includegraphics[width=0.450\linewidth]{plot-ark_analytic_error.png}


\chapter{ark\_analytic\_nonlin}
\label{ark_analytic_nonlin:ark-analytic-nonlin}\label{ark_analytic_nonlin::doc}\label{ark_analytic_nonlin:id1}
This example problem is only marginally more difficult than the
preceding problem, in that the ODE right-hand side function is
nonlinear in the solution $y$.  While the implicit solver from
the preceding problem would also work on this example, because it is
not stiff we use this to demonstrate how to use ARKode's explicit
solver interface.

The ODE problem is
\begin{gather}
\begin{split}\frac{dy}{dt} = (t+1) e^{-y},\end{split}\notag
\end{gather}
for the interval $t \in [0.0, 10.0]$, with initial condition
$y(0)=0$.  This has analytical solution $y(t) =
\log\left(\frac{t^2}{2} + t + 1\right)$.


\section{Numerical method}
\label{ark_analytic_nonlin:numerical-method}
This program solves the problem with the ERK method.
Output is printed every 1.0 units of time (10 total).
Run statistics (optional outputs) are printed at the end.


\section{Solutions}
\label{ark_analytic_nonlin:solutions}
This problem is included both as a simple example to test the
nonlinear solvers within ARKode, but also because it has an analytical
solution, $y(t) = \log\left(\frac{t^2}{2} + t + 1\right)$.  As
seen in the plots below, the computed solution tracks the analytical solution
quite well (left), and results in errors comparable with those
specified by the requested error tolerances (right).

\includegraphics[width=0.450\linewidth]{plot-ark_analytic_nonlin.png}

\includegraphics[width=0.450\linewidth]{plot-ark_analytic_nonlin_error.png}


\chapter{ark\_analytic\_sys}
\label{ark_analytic_sys:ark-analytic-sys}\label{ark_analytic_sys::doc}\label{ark_analytic_sys:id1}
This example demonstrates the use of ARKode's fully implicit solver on
a stiff ODE system, again having an analytical solution.  The problem
is that of a linear ODE system,
\begin{gather}
\begin{split}\frac{dy}{dt} = Ay\end{split}\notag
\end{gather}
where $A = V D V^{-1}$.  In this example, we use
\begin{gather}
\begin{split}V = \left[\begin{array}{rrr} 1 & -1 & 1\\ -1 & 2 & 1\\ 0 & -1 & 2
    \end{array}\right], \qquad
V^{-1} = \frac14 \left[\begin{array}{rrr} 5 & 1 & -3\\ 2 & 2 & -2\\
    1 & 1 & 1 \end{array}\right], \qquad
D = \left[\begin{array}{rrr} -1/2 & 0 & 0\\ 0 & -1/10 & 0\\ 0 & 0 &
    \lambda \end{array}\right].\end{split}\notag
\end{gather}
where $\lambda$ is a large negative number. The analytical
solution to this problem may be computed using the matrix exponential,
\begin{gather}
\begin{split}Y(t) = V e^{Dt} V^{-1} Y(0).\end{split}\notag
\end{gather}
We evolve the problem for $t$ in the interval $\left[0,\,
\frac{1}{20}\right]$, with initial condition $Y(0) = \left[1,\,
1,\, 1\right]^T$.


\section{Numerical method}
\label{ark_analytic_sys:numerical-method}
The stiffness of the problem is directly proportional to the
value of $\lambda$, which is specified through an input file,
along with the desired relative and absolute tolerances.  The value of
$\lambda$ should be negative to result in a well-posed ODE; for
values with magnitude larger than 100 the problem becomes quite stiff.

In the example input file, we choose $\lambda = -100$.

This program solves the problem with the DIRK method,
Newton iteration with the ARKDENSE dense linear solver, and a
user-supplied Jacobian routine.
Output is printed every 0.005 units of time (10 total).
Run statistics (optional outputs) are printed at the end.


\section{Solutions}
\label{ark_analytic_sys:solutions}
This problem is included both as a simple example to test systems of
ODE within ARKode on a problem having an analytical
solution, $Y(t) = V e^{Dt} V^{-1} Y(0)$.  As
seen in the plots below, the computed solution tracks the analytical solution
quite well (left), and results in errors with exactly the magnitude as
specified by the requested error tolerances (right).

\includegraphics[width=0.450\linewidth]{plot-ark_analytic_sys.png}

\includegraphics[width=0.450\linewidth]{plot-ark_analytic_sys_error.png}


\chapter{ark\_brusselator}
\label{ark_brusselator:ark-brusselator}\label{ark_brusselator::doc}\label{ark_brusselator:id1}
We now wish to exercise the ARKode solvers on more challenging
nonlinear ODE systems.  The following test simulates a brusselator
problem from chemical kinetics, and is used throughout the community
as a standard benchmark problem for new solvers.  The ODE system has
with 3 components, $Y = [u,\, v,\, w]$, satisfying the equations,
\begin{gather}
\begin{split}\frac{du}{dt} &= a - (w+1)u + v u^2, \\
\frac{dv}{dt} &= w u - v u^2, \\
\frac{dw}{dt} &= \frac{b-w}{\varepsilon} - w u.\end{split}\notag
\end{gather}
We integrate over the interval $0 \le t \le 10$, with the
initial conditions $u(0) = u_0$, $v(0) = v_0$, $w(0) = w_0$.
After each unit time interval, the solution is output to the screen.

We have 3 different testing scenarios:

Test 1:  $u_0=3.9$,  $v_0=1.1$,  $w_0=2.8$,
$a=1.2$, $b=2.5$, and $\varepsilon=10^{-5}$

Test 2:  $u_0=1.2$, $v_0=3.1$, $w_0=3$, $a=1$,
$b=3.5$, and $\varepsilon=5\cdot10^{-6}$

Test 3:  $u_0=3$, $v_0=3$, $w_0=3.5$, $a=0.5$,
$b=3$, and $\varepsilon=5\cdot10^{-4}$

These tests are selected within the input file (test = \{1,2,3\}),
with the default set to test 2 in case the input is invalid.
Also in the input file, we allow specification of the desired
relative and absolute tolerances.


\section{Numerical method}
\label{ark_brusselator:numerical-method}
This program solves the problem with the DIRK method, using a
Newton iteration with the ARKDENSE dense linear solver, and a
user-supplied Jacobian routine.

100 outputs are printed at equal intervals, and run statistics
are printed at the end.


\section{Solutions}
\label{ark_brusselator:solutions}
The computed solutions will of course depend on which test is
performed:

Test 1:  Here, all three components exhibit a rapid transient change
during the first 0.2 time units, followed by a slow and smooth evolution.

Test 2: Here, $w$ experiences a fast initial transient, jumping
0.5 within a few steps.  All values proceed smoothly until around
$t=6.5$, when both $u$ and $v$ undergo a sharp
transition, with $u$ increaseing from around 0.5 to 5 and
$v$ decreasing from around 6 to 1 in less than 0.5 time units.
After this transition, both $u$ and $v$ continue to evolve
somewhat rapidly for another 1.4 time units, and finish off smoothly.

Test 3: Here, all components undergo very rapid initial transients
during the first 0.3 time units, and all then proceed very smoothly
for the remainder of the simulation.

Unfortunately, there are no known analytical solutions to the
Brusselator problem, but the following results have been verified
in code comparisons against both CVODE and the built-in ODE solver
\code{ode15s} from Matlab:

\includegraphics[width=0.300\linewidth]{plot-ark_brusselator1.png}

\includegraphics[width=0.300\linewidth]{plot-ark_brusselator2.png}

\includegraphics[width=0.300\linewidth]{plot-ark_brusselator3.png}

Brusselator solution plots: left is test 1, center is test 2, right is
test 3.


\chapter{ark\_bruss}
\label{ark_bruss::doc}\label{ark_bruss:ark-bruss}\label{ark_bruss:id1}
This test problem is a Fortran-90 version of the same brusselator
problem as above, in which the ``test 2'' parameters are hard-coded into
the solver.  As with the previous test, this problem has 3 dependent
variables $u$, $v$ and $w$, that depend on the
independent variable $t$ via the IVP system
\begin{gather}
\begin{split}\frac{du}{dt} &= a - (w+1)u + v u^2, \\
\frac{dv}{dt} &= w u - v u^2, \\
\frac{dw}{dt} &= \frac{b-w}{\varepsilon} - w u.\end{split}\notag
\end{gather}
We integrate over the interval $0 \le t \le 10$, with the
initial conditions $u(0) = 3.9$, $v(0) = 1.1$, $w(0) = 2.8$,
and parameters $a=1.2$, $b=2.5$ and
$\varepsilon=10^{-5}$.  After each unit time interval, the
solution is output to the screen.


\section{Numerical method}
\label{ark_bruss:numerical-method}
Since this driver and utility functions are written in Fortran-90,
this example demonstrates the use of the FARKODE interface for the
ARKode solver.  For time integration, this example uses the
fourth-order additive Runge-Kutta method, where the right-hand sides
are broken up as
\begin{gather}
\begin{split}f_E(t,u,v,w) = \left(\begin{array}{c} a - (w+1)u + v u^2 \\
  w u - v u^2 \\ - w u  \end{array}\right), \quad\text{and}\quad
f_I(t,u,v,w) = \left(\begin{array}{c} 0\\0\\
  \frac{b-w}{\varepsilon}\end{array}\right).\end{split}\notag
\end{gather}
The implicit systems are solved using the built-in modified Newton
iteration, with the ARKDENSE dense linear solver.  Both the Jacobian
routine and right-hand side functions are supplied by functions
provided in the example file.

The only non-default solver options are the tolerances
$atol=10^{-10}$ and $rtol=10^{-6}$, adaptivity method 2 (I
controller), a maximum of 8 Newton iterations per step, a nonlinear
solver convergence coefficient $nlscoef=10^{-8}$, and a maximum
of 1000 internal time steps.


\section{Solutions}
\label{ark_bruss:solutions}
With this setup, all three solution components exhibit a rapid
transient change during the first 0.2 time units, followed by a slow
and smooth evolution, as seen in the figure below.  Note that these
results identically match those from the previous C example with the
same equations (test 1).
\begin{figure}[htbp]
\centering

\scalebox{0.700000}{\includegraphics{plot-ark_bruss1.png}}
\end{figure}


\chapter{ark\_robertson}
\label{ark_robertson:ark-robertson}\label{ark_robertson::doc}\label{ark_robertson:id1}
Our next two tests simulate the Robertson problem, corresponding to the
kinetics of an autocatalytic reaction, corresponding to the CVODE
example of the same name.  This is an ODE system with 3
components, $Y = [u,\, v,\, w]^T$, satisfying the equations,
\begin{gather}
\begin{split}\frac{du}{dt} &= -0.04 u + 10^4 v w, \\
\frac{dv}{dt} &= 0.04 u - 10^4 v w - 3\cdot10^7 v^2, \\
\frac{dw}{dt} &= 3\cdot10^7 v^2.\end{split}\notag
\end{gather}
We integrate over the interval $0\le t\le 10^{11}$, with initial
conditions  $Y(0) = [1,\, 0,\, 0]^T$.


\section{Numerical method}
\label{ark_robertson:numerical-method}
In the input file, \code{input\_robertson.txt}, we allow specification of
the desired relative and absolute tolerances.

This program solves the problem with one of the solvers, ERK, DIRK or
ARK.  For DIRK and ARK, implicit subsystems are solved using a Newton
iteration with the ARKDENSE dense linear solver, and a user-supplied
Jacobian routine.

100 outputs are printed at equal intervals, and run statistics are
printed at the end.


\section{Solutions}
\label{ark_robertson:solutions}
Due to the linearly-spaced requested output times in this example, and
since we plot in a log-log scale, by the first output at
$t=10^9$, the solutions have already undergone a sharp
transition from their initial values of $(u,v,w) = (1, 0, 0)$.
For additional detail on the early evolution of this problem, see the
following example, that requests logarithmically-spaced output times.

From the plot here, it is somewhat difficult to see the solution
values for $w$, which here all have a value of
$1\pm10^{-5}$.  Additionally, we see that near the end of the
evolution, the values for $v$ begin to exhibit oscillations;
this is due to the fact that by this point those values have fallen
below their specified absolute tolerance.  A smoother behavior (with
an increase in time steps) may be obtained by reducing the absolute
tolerance for that variable.
\begin{figure}[htbp]
\centering

\scalebox{0.700000}{\includegraphics{plot-ark_robertson.png}}
\end{figure}


\chapter{ark\_robertson\_root}
\label{ark_robertson_root:ark-robertson-root}\label{ark_robertson_root::doc}\label{ark_robertson_root:id1}
We again test the Robertson problem, but in this example we will
utilize both a logarithmically-spaced set of output times (to properly
show the solution behavior), as well as ARKode's root-finding
capabilities.  Again, the Robertson problem consists of an ODE system
with 3 components, $Y = [u,\, v,\, w]^T$, satisfying the equations,
\begin{gather}
\begin{split}\frac{du}{dt} &= -0.04 u + 10^4 v w, \\
\frac{dv}{dt} &= 0.04 u - 10^4 v w - 3\cdot10^7 v^2, \\
\frac{dw}{dt} &= 3\cdot10^7 v^2.\end{split}\notag
\end{gather}
We integrate over the interval $0\le t\le 10^{11}$, with initial
conditions  $Y(0) = [1,\, 0,\, 0]^T$.

Additionally, we supply the following two root-finding equations:
\begin{gather}
\begin{split}g_1(u) = u - 10^{-4}, \\
g_2(w) = w - 10^{-2}.\end{split}\notag
\end{gather}
While these are not inherently difficult nonlinear equations, they
easily serve the purpose of determining the times at which our
solutions attain desired target values.


\section{Numerical method}
\label{ark_robertson_root:numerical-method}
In the input file, \code{input\_robertson.txt}, we allow specification of
the desired relative and absolute tolerances.

This program solves the problem with one of the solvers, ERK, DIRK or
ARK.  For DIRK and ARK, implicit subsystems are solved using a Newton
iteration with the ARKDENSE dense linear solver, and a user-supplied
Jacobian routine.

100 outputs are printed at equal intervals, and run statistics are
printed at the end.

However, unlike in the previous problem, while integrating the system,
we use the rootfinding feature of ARKode to find the times at which
either $u=10^{-4}$ or $w=10^{-2}$.


\section{Solutions}
\label{ark_robertson_root:solutions}
In the solutions below, we now see the early-time evolution of the
solution components for the Robertson ODE system.
\begin{figure}[htbp]
\centering

\scalebox{0.700000}{\includegraphics{plot-ark_robertson_root.png}}
\end{figure}

We note that when running this example, the root-finding capabilities
of ARKode report outside of the typical logarithmically-spaced output
times to declare that at time $t=0.264019$ the variable
$w$ attains the value $10^{-2}$, and that at time
$t=2.07951\cdot10^{7}$ the variable $u$ attains the value
$10^{-4}$; both of our thresholds specified by the root-finding
function \code{g()}.


\chapter{ark\_brusselator1D}
\label{ark_brusselator1D:ark-brusselator1d}\label{ark_brusselator1D::doc}\label{ark_brusselator1D:id1}
We now investigate a time-dependent system of partial differential
equations.  We adapt the previously brusselator test problem by adding
diffusion into the chemical reaction network.  We again have a system
with 3 components, $Y = [u,\, v,\, w]^T$ that satisfy the equations,
\begin{gather}
\begin{split}\frac{\partial u}{\partial t} &= d_u \frac{\partial^2 u}{\partial
   x^2} + a - (w+1) u + v u^2, \\
\frac{\partial v}{\partial t} &= d_v \frac{\partial^2 v}{\partial
   x^2} + w u - v u^2, \\
\frac{\partial w}{\partial t} &= d_w \frac{\partial^2 w}{\partial
   x^2} + \frac{b-w}{\varepsilon} - w u.\end{split}\notag
\end{gather}
However, now these solutions are also spatially dependent.  We
integrate for $t \in [0, 80]$, and $x \in [0, 1]$, with
initial conditions
\begin{gather}
\begin{split}u(0,x) &=  a + \frac{1}{10} \sin(\pi x),\\
v(0,x) &= \frac{b}{a} + \frac{1}{10}\sin(\pi x),\\
w(0,x) &=  b + \frac{1}{10}\sin(\pi x),\end{split}\notag
\end{gather}
and with stationary boundary conditions, i.e.
\begin{gather}
\begin{split}\frac{\partial u}{\partial t}(t,0) &= \frac{\partial u}{\partial t}(t,1) = 0,\\
\frac{\partial v}{\partial t}(t,0) &= \frac{\partial v}{\partial t}(t,1) = 0,\\
\frac{\partial w}{\partial t}(t,0) &= \frac{\partial w}{\partial t}(t,1) = 0.\end{split}\notag
\end{gather}
We note that these can also be implemented as Dirichlet boundary
conditions with values identical to the initial conditions.


\section{Numerical method}
\label{ark_brusselator1D:numerical-method}
We employ a \emph{method of lines} approach, wherein we first
semi-discretize in space to convert the system of 3 PDEs into a larger
system of ODEs.  To this end, the spatial derivatives are computed
using second-order centered differences, with the data distributed
over $N$ points on a uniform spatial grid.  Resultingly, ARKode
approaches the problem as one involving $3N$ coupled ODEs.

The number of spatial points $N$, the parameters $a$,
$b$, $d_u$, $d_v$, $d_w$ and
$\varepsilon$, as well as the desired relative and absolute
solver tolerances, are provided in the input file \code{input\_brusselator1D.txt}.

This program solves the problem with a DIRK method, using a Newton
iteration with the ARKBAND banded linear solver, and a user-supplied
Jacobian routine.

100 outputs are printed at equal intervals, and run statistics
are printed at the end.


\section{Solutions}
\label{ark_brusselator1D:solutions}
\includegraphics[width=0.300\linewidth]{plot-ark_brusselator1D_1.png}

\includegraphics[width=0.300\linewidth]{plot-ark_brusselator1D_2.png}

\includegraphics[width=0.300\linewidth]{plot-ark_brusselator1D_3.png}

Brusselator PDE solution snapshots: left is at time $t=0$,
center is at time $t=2.9$, right is at time $t=8.8$.


\chapter{ark\_heat1D}
\label{ark_heat1D:ark-heat1d}\label{ark_heat1D::doc}\label{ark_heat1D:id1}
As with the previous brusselator problrem, this example simulates a
simple one-dimensional heat equation,
\begin{gather}
\begin{split}\frac{\partial u}{\partial t} = k \frac{\partial^2 u}{\partial x^2} + f,\end{split}\notag
\end{gather}
for $t \in [0, 10]$, and $x \in [0, 1]$, with initial
condition $u(0,x) = 0$, stationary boundary conditions,
\begin{gather}
\begin{split}\frac{\partial u}{\partial t}(t,0) = \frac{\partial u}{\partial t}(t,1) = 0,\end{split}\notag
\end{gather}
and a point-source heating term,
\begin{gather}
\begin{split}f(t,x) = \begin{cases} 1 & \text{if}\;\; x=1/2, \\
                       0 & \text{otherwise}. \end{cases}\end{split}\notag
\end{gather}

\section{Numerical method}
\label{ark_heat1D:numerical-method}
As with the \code{brusselator1D.c} test problem, this test computes
spatial derivatives using second-order centered differences, with the
data distributed over $N$ points on a uniform spatial grid.

The number of spatial points $N$ and the heat conductivity
parameter $k$, as well as the desired relative and absolute
solver tolerances, are provided in the input file \code{input\_heat1D.txt}.

This program solves the problem with a DIRK method, utilizing a Newton
iteration.  The primary utility in including this example is that each
Newton system is now solved with the PCG iterative linear solver, and
a user-supplied Jacobian-vector product routine, in order to provide
examples of their use.


\section{Solutions}
\label{ark_heat1D:solutions}
\includegraphics[width=0.300\linewidth]{plot-ark_heat1d_1.png}

\includegraphics[width=0.300\linewidth]{plot-ark_heat1d_2.png}

\includegraphics[width=0.300\linewidth]{plot-ark_heat1d_3.png}

One-dimensional heat PDE solution snapshots: left is at time $t=0.01$,
center is at time $t=0.13$, right is at time $t=1.0$.


\chapter{ark\_heat2D}
\label{ark_heat2D:ark-heat2d}\label{ark_heat2D::doc}\label{ark_heat2D:id1}
Our final example problem extends the previous test to now simulate a
simple two-dimenaional heat equation,
\begin{gather}
\begin{split}\frac{\partial u}{\partial t} = k_x \frac{\partial^2 u}{\partial x^2}
                              + k_y \frac{\partial^2 u}{\partial y^2} + h,\end{split}\notag
\end{gather}
for $t \in [0, 0.3]$, and $(x,y) \in [0, 1]^2$, with initial
condition $u(0,x,y) = 0$, stationary boundary conditions,
\begin{gather}
\begin{split}\frac{\partial u}{\partial t}(t,0,y) = \frac{\partial u}{\partial t}(t,1,y) =
\frac{\partial u}{\partial t}(t,x,0) = \frac{\partial u}{\partial t}(t,x,1) = 0,\end{split}\notag
\end{gather}
and a periodic heat source,
\begin{gather}
\begin{split}h(x,y) = \sin(\pi x) \sin(2\pi y).\end{split}\notag
\end{gather}
Under these conditions, the problem has an analytical solution of the
form
\begin{gather}
\begin{split}u(t,x,y) = \frac{1 - e^{-(k_x+4k_y)\pi^2 t}}{(k_x+4k_y)\pi^2} \sin(\pi x) sin(2\pi y).\end{split}\notag
\end{gather}

\section{Numerical method}
\label{ark_heat2D:numerical-method}
The spatial derivatives are computed using second-order
centered differences, with the data distributed over $nx\times
ny$ points on a uniform spatial grid.

The spatial grid parameters $nx$ and $ny$, the heat
conductivity parameters $k_x$ and $k_y$, as well as the
desired relative and absolute solver tolerances, are provided in the
input file \code{input\_heat2D.txt}.

As with the 1D version, this program solves the problem with a DIRK
method, that itself uses a Newton iteration and PCG iterative linear
solver.  However, unlike the previous example, here the PCG solver is
preconditioned using a single Jacobi iteration, and uses the
built-in finite-difference Jacobian-vector product routine.
Additionally, this problem uses MPI and the \code{NVECTOR\_PARALLEL}
module for parallelization.


\section{Solutions}
\label{ark_heat2D:solutions}
Top row: 2D heat PDE solution snapshots, the left is at time $t=0$,
center is at time $t=0.03$, right is at time $t=0.3$.
Bottom row, absolute error in these solutions.  Note that the relative
error in these results is on the order $10^{-4}$, corresponding
to the spatial accuracy of the relatively coarse finite-difference mesh.

\includegraphics[width=0.300\linewidth]{plot-ark_heat2d_1.png}

\includegraphics[width=0.300\linewidth]{plot-ark_heat2d_2.png}

\includegraphics[width=0.300\linewidth]{plot-ark_heat2d_3.png}

\includegraphics[width=0.300\linewidth]{plot-ark_heat2d_err_1.png}

\includegraphics[width=0.300\linewidth]{plot-ark_heat2d_err_2.png}

\includegraphics[width=0.300\linewidth]{plot-ark_heat2d_err_3.png}


\chapter{ark\_KrylovDemo\_prec}
\label{ark_KrylovDemo_prec:ark-krylovdemo-prec}\label{ark_KrylovDemo_prec::doc}\label{ark_KrylovDemo_prec:id1}
This problem is an ARKode clone of the CVODE problem,
\code{cv\_KrylovDemo\_prec}.  This is a demonstration program using the
GMRES linear solver.  The program solves a stiff ODE system that arises
from a system of PDEs modeling a six-species food web population
model, with predator-prey interaction and diffusion on the unit square
in two dimensions. We have a system with 6 components, $C =
[c^1,\, c^2,\,\ldots, c^6]^T$ that satisfy the equations,
\begin{gather}
\begin{split}\frac{\partial c^i}{\partial t} &= d_i \left(\frac{\partial^2 c^i}{\partial
   x^2} + \frac{\partial^2 c^i}{\partial y^2}\right) +
   f_i(x,y,c),\quad i=1,\ldots,6.\end{split}\notag
\end{gather}
where
\begin{gather}
\begin{split}f_i(x,y,c) = c^i\left( b_i + \sum_{j=1}^{ns} a_{i,j} c^j\right).\end{split}\notag
\end{gather}
Here, the first three species are prey and the last three are
predators.  The coefficients $a_{i,j}, b_i, d_i$ are:
\begin{gather}
\begin{split}a_{i,j} = \begin{cases}
            -1, \quad & i=j,\\
            -0.5\times10^{-6}, \quad & i\le 3, j>3, \\
             10^4, \quad & i>3, j\le3
          \end{cases}
b_i = \begin{cases}
         (1+xy), \quad & i\le 3,\\
        -(1+xy), \quad & i>3
      \end{cases}
d_i = \begin{cases}
         1, \quad & i\le 3,\\
         \frac12, \quad & i>3
      \end{cases}\end{split}\notag
\end{gather}
The spatial domain is $(x,y) \in [0, 1]^2$; the time domain is
$t \in [0,10]$, with initial conditions
\begin{gather}
\begin{split}c^i(x,y) &=  10 + i \sqrt{4x(1-x)}\sqrt{4y(1-y)}\end{split}\notag
\end{gather}
and with homogeneous Neumann boundary conditions,
$\nabla c^i \cdot \vec{n} = 0$.


\section{Numerical method}
\label{ark_KrylovDemo_prec:numerical-method}
We employ a method of lines approach, wherein we first
semi-discretize in space to convert the system of 6 PDEs into a larger
system of ODEs.  To this end, the spatial derivatives are computed
using second-order centered differences, with the data distributed
over $Mx*My$ points on a uniform spatial grid.  Resultingly, ARKode
approaches the problem as one involving $6*Mx*My$ coupled ODEs.

This program solves the problem with a DIRK method, using a Newton
iteration with the preconditioned ARKSPGMR iterative linear solver.
The preconditioner matrix used is the product of two matrices:
\begin{enumerate}
\item {} 
A matrix, only defined implicitly, based on a fixed number of
Gauss-Seidel iterations using the diffusion terms only.

\item {} 
A block-diagonal matrix based on the partial derivatives of the
interaction terms $f$ only, using block-grouping (computing
only a subset of the $3\times3$ blocks).

\end{enumerate}

Four different runs are made for this problem.  The product
preconditoner is applied on the left and on the right.  In each case,
both the modified and classical Gram-Schmidt orthogonalization options
are tested.  In the series of runs, \code{ARKodeInit} and \code{ARKSpgmr}
are called only for the first run, whereas \code{ARKodeReInit},
\code{ARKSpilsSetPrecType} and \code{ARKSpilsSetGSType} are called for each
of the remaining three runs.

A problem description, performance statistics at selected output
times, and final statistics are written to standard output.  On the
first run, solution values are also printed at output times.  Error
and warning messages are written to standard error, but there should
be no such messages.


\chapter{ark\_brusselator\_fp}
\label{ark_brusselator_fp:ark-brusselator-fp}\label{ark_brusselator_fp::doc}\label{ark_brusselator_fp:id1}
This test problem is a duplicate the \code{ark\_brusselator} problem
above, but with a few key changes in the methods used for time
integration and nonlinear solver.  As with the previous test, this
problem has 3 dependent variables $u$, $v$ and $w$,
that depend on the independent variable $t$ via the IVP system
\begin{gather}
\begin{split}\frac{du}{dt} &= a - (w+1)u + v u^2, \\
\frac{dv}{dt} &= w u - v u^2, \\
\frac{dw}{dt} &= \frac{b-w}{\varepsilon} - w u.\end{split}\notag
\end{gather}
We integrate over the interval $0 \le t \le 10$, with the
initial conditions $u(0) = u_0$, $v(0) = v_0$, $w(0) = w_0$.
After each unit time interval, the solution is output to the screen.

We have 3 different testing scenarios:

Test 1:  $u_0=3.9$,  $v_0=1.1$,  $w_0=2.8$,
$a=1.2$, $b=2.5$, and $\varepsilon=10^{-5}$

Test 2:  $u_0=1.2$, $v_0=3.1$, $w_0=3$, $a=1$,
$b=3.5$, and $\varepsilon=5\cdot10^{-6}$

Test 3:  $u_0=3$, $v_0=3$, $w_0=3.5$, $a=0.5$,
$b=3$, and $\varepsilon=5\cdot10^{-4}$

These tests are selected within the input file (test = \{1,2,3\}),
with the default set to test 2 in case the input is invalid.
Also in the input file, we allow specification of the desired
relative and absolute tolerances.


\section{Numerical method}
\label{ark_brusselator_fp:numerical-method}
This program solves the problem with the ARK method, in which we have
split the right-hand side into stiff ($f_i(t,y)$) and non-stiff
($f_e(t,y)$) components,
\begin{gather}
\begin{split}f_i(t,y) = \left[\begin{array}{c}
   0 \\ 0 \\ \frac{b-w}{\varepsilon}
\end{array}\right]
\qquad
f_e(t,y) = \left[\begin{array}{c}
   a - (w+1)u + v u^2 \\ w u - v u^2 \\ - w u
\end{array}\right].\end{split}\notag
\end{gather}
Also unlike the previous test problem, we solve the resulting implicit
stages using the available accelerated fixed-point solver, enabled
through a call to \code{ARKodeSetFixedPoint}, with an acceleration
subspace of dimension 3.

100 outputs are printed at equal intervals, and run statistics
are printed at the end.


\chapter{ark\_diurnal\_kry\_bbd\_p}
\label{ark_diurnal_kry_bbd_p:ark-diurnal-kry-bbd-p}\label{ark_diurnal_kry_bbd_p::doc}\label{ark_diurnal_kry_bbd_p:id1}
This problem is an ARKode clone of the CVODE problem,
\code{cv\_diurnal\_kry\_bbd\_p}.  This test problem models a two-species
diurnal kinetics advection-diffusion PDE system in two spatial
dimensions,
\begin{gather}
\begin{split}\frac{\partial c_i}{\partial t} &=
  K_h \frac{\partial^2 c_i}{\partial x^2} +
  V \frac{\partial     c_i}{\partial x} +
  \frac{\partial}{\partial y}\left( K_v(y)
  \frac{\partial c_i}{\partial y}\right) +
  R_i(c_1,c_2,t),\quad i=1,2\end{split}\notag
\end{gather}
where
\begin{gather}
\begin{split}R_1(c_1,c_2,t) &= -q_1*c_1*c_3 - q_2*c_1*c_2 + 2*q_3(t)*c_3 + q_4(t)*c_2, \\
R_2(c_1,c_2,t) &=  q_1*c_1*c_3 - q_2*c_1*c_2 - q_4(t)*c_2, \\
K_v(y) &= K_{v0} e^{y/5}.\end{split}\notag
\end{gather}
Here $K_h$, $V$, $K_{v0}$, $q_1$, $q_2$,
and $c_3$ are constants, and $q_3(t)$ and $q_4(t)$
vary diurnally.  The problem is posed on the square spatial domain
$(x,y) \in [0,20]\times[30,50]$, with homogeneous Neumann
boundary conditions, and for time interval $t\in [0,86400]$ sec
(1 day).

We enforce the initial conditions
\begin{gather}
\begin{split}c_1(x,y) &=  10^6 \chi(x)\eta(y) \\
c_2(x,y) &=  10^{12} \chi(x)\eta(y) \\
\chi(x) &= 1 - \sqrt{\frac{x - 10}{10}} + \frac12 \sqrt[4]{\frac{x - 10}{10}} \\
\eta(y) &= 1 - \sqrt{\frac{y - 40}{10}} + \frac12 \sqrt[4]{\frac{x - 10}{10}}.\end{split}\notag
\end{gather}

\section{Numerical method}
\label{ark_diurnal_kry_bbd_p:numerical-method}
We employ a method of lines approach, wherein we first
semi-discretize in space to convert the system of 2 PDEs into a larger
system of ODEs.  To this end, the spatial derivatives are computed
using second-order centered differences, with the data distributed
over $Mx*My$ points on a uniform spatial grid.  Resultingly, ARKode
approaches the problem as one involving $2*Mx*My$ coupled ODEs.

The problem is decomposed in parallel into uniformly-sized subdomains,
with two subdomains in each direction (four in total), and where each
subdomain has five points in each direction (i.e. $Mx=My=10$).

This program solves the problem with a DIRK method, using a Newton
iteration with the preconditioned ARKSPGMR iterative linear solver.

The preconditioner matrix used is block-diagonal, with banded blocks,
constructed using the ARKBBDPRE module.  Each block is generated using
difference quotients, with half-bandwidths \code{mudq = mldq = 10}, but
the retained banded blocks have half-bandwidths \code{mukeep = mlkeep = 2}.
A copy of the approximate Jacobian is saved and conditionally reused
within the preconditioner routine.

Two runs are made for this problem, first with left and then with
right preconditioning.

Performance data and sampled solution values are printed at
selected output times, and all performance counters are printed
on completion.


\chapter{ark\_diurnal\_kry\_p}
\label{ark_diurnal_kry_p:id1}\label{ark_diurnal_kry_p::doc}\label{ark_diurnal_kry_p:ark-diurnal-kry-p}
This problem is an ARKode clone of the CVODE problem,
\code{cv\_diurnal\_kry\_p}.  This test problem models a two-species
diurnal kinetics advection-diffusion PDE system in two spatial
dimensions,
\begin{gather}
\begin{split}\frac{\partial c_i}{\partial t} &=
  K_h \frac{\partial^2 c_i}{\partial x^2} +
  V \frac{\partial     c_i}{\partial x} +
  \frac{\partial}{\partial y}\left( K_v(y)
  \frac{\partial c_i}{\partial y}\right) +
  R_i(c_1,c_2,t),\quad i=1,2\end{split}\notag
\end{gather}
where
\begin{gather}
\begin{split}R_1(c_1,c_2,t) &= -q_1*c_1*c_3 - q_2*c_1*c_2 + 2*q_3(t)*c_3 + q_4(t)*c_2, \\
R_2(c_1,c_2,t) &=  q_1*c_1*c_3 - q_2*c_1*c_2 - q_4(t)*c_2, \\
K_v(y) &= K_{v0} e^{y/5}.\end{split}\notag
\end{gather}
Here $K_h$, $V$, $K_{v0}$, $q_1$, $q_2$,
and $c_3$ are constants, and $q_3(t)$ and $q_4(t)$
vary diurnally.  The problem is posed on the square spatial domain
$(x,y) \in [0,20]\times[30,50]$, with homogeneous Neumann
boundary conditions, and for time interval $t\in [0,86400]$ sec
(1 day).

We enforce the initial conditions
\begin{gather}
\begin{split}c^1(x,y) &=  10^6 \chi(x)\eta(y) \\
c^2(x,y) &=  10^{12} \chi(x)\eta(y) \\
\chi(x) &= 1 - \sqrt{\frac{x - 10}{10}} + \frac12 \sqrt[4]{\frac{x - 10}{10}} \\
\eta(y) &= 1 - \sqrt{\frac{y - 40}{10}} + \frac12 \sqrt[4]{\frac{x - 10}{10}}.\end{split}\notag
\end{gather}

\section{Numerical method}
\label{ark_diurnal_kry_p:numerical-method}
We employ a method of lines approach, wherein we first
semi-discretize in space to convert the system of 2 PDEs into a larger
system of ODEs.  To this end, the spatial derivatives are computed
using second-order centered differences, with the data distributed
over $Mx*My$ points on a uniform spatial grid.  Resultingly, ARKode
approaches the problem as one involving $2*Mx*My$ coupled ODEs.

The problem is decomposed in parallel into uniformly-sized subdomains,
with two subdomains in each direction (four in total), and where each
subdomain has five points in each direction (i.e. $Mx=My=10$).

This program solves the problem with a DIRK method, using a Newton
iteration with the preconditioned ARKSPGMR iterative linear solver.

The preconditioner matrix used is block-diagonal, with block-diagonal
portion of the Newton matrix used as a left preconditioner.  A copy of
the block-diagonal portion of the Jacobian is saved and conditionally
reused within the preconditioner routine.

Performance data and sampled solution values are printed at
selected output times, and all performance counters are printed
on completion.


\chapter{ark\_heat1D\_adapt}
\label{ark_heat1D_adapt:id1}\label{ark_heat1D_adapt::doc}\label{ark_heat1D_adapt:ark-heat1d-adapt}
This problem is a clone of the \code{ark\_heat1D} test problem except that
unlike the previous uniform-grid problem, this test problem allows a
dynamically-evolving spatial mesh.  The PDE under consideration is a
simple one-dimensional heat equation,
\begin{gather}
\begin{split}\frac{\partial u}{\partial t} = k \frac{\partial^2 u}{\partial x^2} + f,\end{split}\notag
\end{gather}
for $t \in [0, 10]$, and $x \in [0, 1]$, with initial
condition $u(0,x) = 0$, stationary boundary conditions,
\begin{gather}
\begin{split}\frac{\partial u}{\partial t}(t,0) = \frac{\partial u}{\partial t}(t,1) = 0,\end{split}\notag
\end{gather}
and a point-source heating term,
\begin{gather}
\begin{split}f(t,x) = \begin{cases} 1 & \text{if}\;\; x=1/2, \\
                       0 & \text{otherwise}. \end{cases}\end{split}\notag
\end{gather}

\section{Numerical method}
\label{ark_heat1D_adapt:numerical-method}
We again employ a method-of-lines discretization approach.  The
spatial derivatives are computed using a three-point centered stencil,
that is accurate to $O(\Delta x_i^2)$ if the neighboring points are
equidistant from the central point, i.e. $x_{i+1} - x_i = x_i -
x_{i-1}$, though if these are unequal the approximation reduces to
first-order accuracy.  The spatial mesh is initially distributed
uniformly over 21 points in $[0,1]$, but as the simulation
proceeds the mesh is {[}crudely{]} adapted to add points to the center of
subintervals bordering any node where
$\left|\frac{\partial^2 u}{\partial x^2}\right| > 3\times10^{-3}$.

This program solves the problem with a DIRK method, utilizing a Newton
iteration and the PCG iterative linear solver.  Additionally, the test
problem utilizes ARKode's spatial adaptivity support (via
\code{ARKodeResize}), allowing retention of the major ARKode data
structures across vector length changes.


\chapter{fark\_diag\_kry\_bbd\_p}
\label{fark_diag_kry_bbd_p:fark-diag-kry-bbd-p}\label{fark_diag_kry_bbd_p::doc}\label{fark_diag_kry_bbd_p:id1}
This problem is an ARKode clone of the CVODE problem,
\code{fcv\_diag\_kry\_bbd\_p}.  This test problem models a stiff, linear,
diagonal ODE system,
\begin{gather}
\begin{split}\frac{\partial y_i}{\partial t} &= -\alpha i y_i, \quad i=1,\ldots N.\end{split}\notag
\end{gather}
Here $\alpha=10$ and $N=10 N_P$, where $N_P$ is the
number of MPI tasks used for the problem.  The problem has initial
conditions $y_i=1$ and evolves for the time interval $t\in [0,1]$.


\section{Numerical method}
\label{fark_diag_kry_bbd_p:numerical-method}
This program solves the problem with a DIRK method, using a Newton
iteration with the preconditioned ARKSPGMR iterative linear solver.

A diagonal preconditioner matrix is used, formed automatically through
difference quotients within the ARKBBDPRE module.  Since ARKBBDPRE is
developed for use of a block-banded preconditioner, in this solver
each block is set to have half-bandwidths \code{mudq = mldq = 0} to
retain only the diagonal portion.

Two runs are made for this problem, first with left and then with
right preconditioning (\code{IPRE} is first set to 1 and then to 2).

Performance data is printed at selected output times, and maximum
errors and final performance counters are printed on completion.


\chapter{fark\_diag\_non\_p}
\label{fark_diag_non_p:fark-diag-non-p}\label{fark_diag_non_p::doc}\label{fark_diag_non_p:id1}
This problem is an ARKode clone of the CVODE problem,
\code{fcv\_diag\_non\_p}.  This test problem models a nonstiff, linear,
diagonal ODE system,
\begin{gather}
\begin{split}\frac{\partial y_i}{\partial t} &= -\alpha i y_i, \quad i=1,\ldots N.\end{split}\notag
\end{gather}
Here $\alpha=\frac{10}{N}$ and $N=10 N_P$, where $N_P$ is the
number of MPI tasks used for the problem.  The problem has initial
conditions $y_i=1$ and evolves for the time interval $t\in [0,1]$.


\section{Numerical method}
\label{fark_diag_non_p:numerical-method}
This program solves the problem with an ERK method, and hence does not
require either a nonlinear or linear solver for integration.

Performance data is printed at selected output times, and maximum
errors and final performance counters are printed on completion.


\chapter{fark\_diurnal\_kry\_bp}
\label{fark_diurnal_kry_bp::doc}\label{fark_diurnal_kry_bp:fark-diurnal-kry-bp}\label{fark_diurnal_kry_bp:id1}
This problem is an ARKode clone of the CVODE problem,
\code{fcv\_diurnal\_kry\_bp}.  This test problem models a two-species
diurnal kinetics advection-diffusion PDE system in two spatial
dimensions,
\begin{gather}
\begin{split}\frac{\partial c_i}{\partial t} &=
  K_h \frac{\partial^2 c_i}{\partial x^2} +
  V \frac{\partial     c_i}{\partial x} +
  \frac{\partial}{\partial y}\left( K_v(y)
  \frac{\partial c_i}{\partial y}\right) +
  R_i(c_1,c_2,t),\quad i=1,2\end{split}\notag
\end{gather}
where
\begin{gather}
\begin{split}R_1(c_1,c_2,t) &= -q_1*c_1*c_3 - q_2*c_1*c_2 + 2*q_3(t)*c_3 + q_4(t)*c_2, \\
R_2(c_1,c_2,t) &=  q_1*c_1*c_3 - q_2*c_1*c_2 - q_4(t)*c_2, \\
K_v(y) &= K_{v0} e^{y/5}.\end{split}\notag
\end{gather}
Here $K_h$, $V$, $K_{v0}$, $q_1$, $q_2$,
and $c_3$ are constants, and $q_3(t)$ and $q_4(t)$
vary diurnally.  The problem is posed on the square spatial domain
$(x,y) \in [0,20]\times[30,50]$, with homogeneous Neumann
boundary conditions, and for time interval $t\in [0,86400]$ sec
(1 day).

We enforce the initial conditions
\begin{gather}
\begin{split}c^1(x,y) &=  10^6 \chi(x)\eta(y) \\
c^2(x,y) &=  10^{12} \chi(x)\eta(y) \\
\chi(x) &= 1 - \sqrt{\frac{x - 10}{10}} + \frac12 \sqrt[4]{\frac{x - 10}{10}} \\
\eta(y) &= 1 - \sqrt{\frac{y - 40}{10}} + \frac12 \sqrt[4]{\frac{x - 10}{10}}.\end{split}\notag
\end{gather}

\section{Numerical method}
\label{fark_diurnal_kry_bp:numerical-method}
We employ a method of lines approach, wherein we first
semi-discretize in space to convert the system of 2 PDEs into a larger
system of ODEs.  To this end, the spatial derivatives are computed
using second-order centered differences, with the data distributed
over $Mx*My$ points on a uniform spatial grid.  Resultingly, ARKode
approaches the problem as one involving $2*Mx*My$ coupled ODEs.
In this problem, we use a relatively coarse uniform mesh with
$Mx=My=10$.

This program solves the problem with a DIRK method, using a Newton
iteration with the preconditioned ARKSPGMR iterative linear solver.

The left preconditioner used is a banded matrix, constructed using
the ARKBP module.  The preconditioner matrix is generated using
difference quotients, with half-bandwidths \code{mu = ml = 2}.

Performance data and sampled solution values are printed at
selected output times, and all performance counters are printed
on completion.


\chapter{fark\_heat2D}
\label{fark_heat2D:fark-heat2d}\label{fark_heat2D::doc}\label{fark_heat2D:id1}
This test problem is a Fortran-90 version of the same two-dimensional
heat equation problem as above, {\hyperref[ark_heat2D:ark-heat2d]{\emph{ark\_heat2D}}}.  This models a
simple two-dimenaional heat equation,
\begin{gather}
\begin{split}\frac{\partial u}{\partial t} = k_x \frac{\partial^2 u}{\partial x^2}
                              + k_y \frac{\partial^2 u}{\partial y^2} + h,\end{split}\notag
\end{gather}
for $t \in [0, 0.3]$, and $(x,y) \in [0, 1]^2$, with initial
condition $u(0,x,y) = 0$, stationary boundary conditions,
\begin{gather}
\begin{split}\frac{\partial u}{\partial t}(t,0,y) = \frac{\partial u}{\partial t}(t,1,y) =
\frac{\partial u}{\partial t}(t,x,0) = \frac{\partial u}{\partial t}(t,x,1) = 0,\end{split}\notag
\end{gather}
and a periodic heat source,
\begin{gather}
\begin{split}h(x,y) = \sin(\pi x) \sin(2\pi y).\end{split}\notag
\end{gather}
Under these conditions, the problem has an analytical solution of the
form
\begin{gather}
\begin{split}u(t,x,y) = \frac{1 - e^{-(k_x+4k_y)\pi^2 t}}{(k_x+4k_y)\pi^2} \sin(\pi x) sin(2\pi y).\end{split}\notag
\end{gather}

\section{Numerical method}
\label{fark_heat2D:numerical-method}
The spatial derivatives are computed using second-order
centered differences, with the data distributed over $nx\times
ny$ points on a uniform spatial grid.

The spatial grid is set to $nx=60$ and $ny=120$.  The heat
conductivity parameters are $k_x=0.5$ and $k_y=0.75$.

As with the C++ version, this program solves the problem with a DIRK
method, that itself uses a Newton iteration and PCG iterative linear
solver.  Also like the C++ version, the PCG solver is preconditioned
using a single Jacobi iteration, and uses the built-in
finite-difference Jacobian-vector product routine within the PCG
solver.  Additionally, this problem uses MPI and the Fortran interface
to the \code{NVECTOR\_PARALLEL} module for parallelization.


\chapter{fark\_roberts\_dnsL}
\label{fark_roberts_dnsL::doc}\label{fark_roberts_dnsL:fark-roberts-dnsl}\label{fark_roberts_dnsL:id1}
This problem is an ARKode clone of the CVODE problem,
\code{fcv\_roberts\_dnsL}.  This test problem models the kinetics of a
three-species autocatalytic reaction.  This is an ODE system with 3
components, $Y = [y_1,\, y_2,\, y_3]^T$, satisfying the equations,
\begin{gather}
\begin{split}\frac{d y_1}{dt} &= -0.04 y_1 + 10^4 y_2 y_3, \\
\frac{d y_2}{dt} &= 0.04 y_1 - 10^4 y_2 y_3 - 3\cdot10^7 y_2^2, \\
\frac{d y_3}{dt} &= 3\cdot10^7 y_2^2.\end{split}\notag
\end{gather}
We integrate over the interval $0\le t\le 4\cdot10^{10}$, with initial
conditions  $Y(0) = [1,\, 0,\, 0]^T$.

Additionally, we supply the following two root-finding equations:
\begin{gather}
\begin{split}g_1(u) = u - 10^{-4}, \\
g_2(w) = w - 10^{-2}.\end{split}\notag
\end{gather}
While these are not inherently difficult nonlinear equations, they
easily serve the purpose of determining the times at which our
solutions attain desired target values.


\section{Numerical method}
\label{fark_roberts_dnsL:numerical-method}
This program solves the problem with a DIRK method, using a Newton
iteration with the dense LAPACK linear solver module.

As with the {\hyperref[ark_robertson_root:ark-robertson-root]{\emph{ark\_robertson\_root}}} problem, we enable ARKode's
rootfinding module to find the times at which
either $u=10^{-4}$ or $w=10^{-2}$.

Performance data and solution values are printed at
selected output times, along with additional output at rootfinding
events.  All performance counters are printed on completion.



\renewcommand{\indexname}{Index}
\printindex
\end{document}
