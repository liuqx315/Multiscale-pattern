These generic linear solver modules in {\sundials} are organized in
two families of solvers, the {\em dls} family, which includes direct
linear solvers appropriate for sequential computations; and the {\em spils}
family, which includes scaled preconditioned iterative (Krylov) linear solvers.
The solvers in each family share common data structures and functions.

The {\em dls} family contains the following two generic linear solvers:
\begin{itemize}
\item The {\dense} package, a linear solver for dense matrices either specified 
  through a matrix type (defined below) or as simple arrays.
\item The {\band} package, a linear solver for banded matrices either specified 
  through a matrix type (defined below) or as simple arrays.
\end{itemize}
Note that this family also includes the Blas/Lapack linear solvers (dense and band) 
available to the {\sundials} solvers, but these are not discussed here.

The {\em spils} family contains the following three generic linear solvers:
\begin{itemize}
\item The {\spgmr} package, a solver for the scaled preconditioned GMRES method.
\item The {\spbcg} package, a solver for the scaled preconditioned Bi-CGStab method.
\item The {\sptfqmr} package, a solver for the scaled preconditioned TFQMR method.
\end{itemize}

For reasons related to installation, the names of the files involved
in these generic solvers begin with the prefix \id{sundials\_}.  But
despite this, each of the solvers is in fact generic, in that it is
usable completely independently of {\sundials}.

For the sake of space, the functions for the \id{dense} and \id{band} modules
that work with a matrix type and the functions in the {\spgmr}, {\spbcg}, and {\sptfqmr}
modules are only summarized briefly, since they are less likely to be of direct use
in connection with a {\sundials} solver.  However, the functions for dense matrices 
treated as simple arrays are fully described, because we expect that they will be 
useful in the implementation of preconditioners used with the combination of one of
the {\sundials} solvers and one of the {\em spils} linear solvers.



% ====================================================================================
\section{The DLS modules: DENSE and BAND}\label{s:dls}
% ====================================================================================

\index{generic linear solvers!DENSE@{\dense}}
\index{generic linear solvers!BAND@{\band}}
The files comprising the {\dense} generic linear solver, and their locations
in the {\sundials} {\em srcdir}, are as follows:
\begin{itemize}
\item header files (located in {\em srcdir}\id{/include/sundials})\\
  \id{sundials\_direct.h} \id{sundials\_dense.h} \\
  \id{sundials\_types.h} \id{sundials\_math.h}  \id{sundials\_config.h}
\item source files (located in {\em srcdir}\id{/src/sundials})\\
  \id{sundials\_direct.c} \id{sundials\_dense.c} \id{sundials\_math.c}
\end{itemize}
The files comprising the {\band} generic linear solver are as follows:
\begin{itemize}
\item header files (located in {\em srcdir}\id{/include/sundials})\\
  \id{sundials\_direct.h} \id{sundials\_band.h} \\
  \id{sundials\_types.h} \id{sundials\_math.h}  \id{sundials\_config.h}
\item source files (located in {\em srcdir}\id{/src/sundials})\\
  \id{sundials\_direct.c} \id{sundials\_band.c} \id{sundials\_math.c}
\end{itemize}
%%
Only two of the preprocessing directives in the header file \id{sundials\_config.h} 
are relevant to the {\dense} and {\band} packages by themselves (see
\S\ref{ss:no_config} for details):
\begin{itemize}
\item (required) definition of the precision of the {\sundials} type \id{realtype}. 
  One of the following lines must be present:\\
  \id{\#define SUNDIALS\_DOUBLE\_PRECISION 1}\\
  \id{\#define SUNDIALS\_SINGLE\_PRECISION 1}\\
  \id{\#define SUNDIALS\_EXTENDED\_PRECISION 1}
\item (optional) use of generic math functions:
  \id{\#define SUNDIALS\_USE\_GENERIC\_MATH 1}
\end{itemize}
The \id{sundials\_types.h} header file defines the {\sundials}
\id{realtype} and \id{booleantype} types and the macro \id{RCONST}, while the 
\id{sundials\_math.h} header file is needed for the \id{MIN}, \id{MAX}, and 
\id{ABS} macros and \id{RAbs} function.

The files listed above for either module can be extracted  from the {\sundials} 
{\em srcdir} and compiled by themselves into a separate library or into a larger user code.

% ------------------------------------------------------------------------------
\subsection{Type DlsMat}
% ------------------------------------------------------------------------------
%%
\index{DENSE@{\dense} generic linear solver!type \id{DlsMat}|(}
\index{BAND@{\band} generic linear solver!type \id{DlsMat}|(}
The type \ID{DlsMat}, defined in \id{sundials\_direct.h} is a pointer to a 
structure defining a generic matrix, and is used with all linear solvers in 
the {\em dls} family:
\begin{verbatim}
typedef struct _DlsMat {
  int type;
  long int M;
  long int N;
  long int ldim;
  long int mu;
  long int ml;
  long int s_mu;
  realtype *data;
  long int ldata;
  realtype **cols;
} *DlsMat;
\end{verbatim}
%%
For the {\dense} module, the relevant fields of this structure are as follows.
Note that a dense matrix of type \id{DlsMat} need not be square.
\begin{description}
  \item[type]  - \id{SUNDIALS\_DENSE} (=1)
  \item[M]  - number of rows
  \item[N]  - number of columns
  \item[ldim]  - leading dimension (\id{ldim} $\ge$ \id{M})
  \item[data]  - pointer to a contiguous block of \id{realtype} variables
  \item[ldata] - length of the data array ($=$ \id{ldim}$\cdot$\id{N}).
    The (\id{i},\id{j})-th element of a dense matrix \id{A} of type \id{DlsMat}
    (with $0 \le$ \id{i} $<$ \id{M} and $ 0 \le$ \id{j} $<$ \id{N}) 
    is given by the expression \id{(A->data)[0][j*M+i]}
  \item[cols]  - array of pointers. \id{cols[j]} points to the first element 
    of the j-th column of the matrix in the array data.
    The (\id{i},\id{j})-th element of a dense matrix \id{A} of type \id{DlsMat}
    (with $0 \le$ \id{i} $<$ \id{M} and $ 0 \le$ \id{j} $<$ \id{N}) 
    is given by the expression \id{(A->cols)[j][i]} 
\end{description}
%%
For the {\band} module, the relevant fields of this structure are as follows
(see Figure \ref{f:bandmat} for a diagram of the underlying data representation
in a banded matrix of type \id{DlsMat}). Note that only square band matrices are 
allowed.
\begin{description}
  \item[type]  - \id{SUNDIALS\_BAND} (=2)
  \item[M]  - number of rows
  \item[N]  - number of columns (\id{N} = \id{M})
  \item[mu]    - upper half-bandwidth, $0 \le$ \id{mu} $<$ min(\id{M},\id{N})
  \item[ml]    - lower half-bandwidth, $0 \le$ \id{ml} $<$ min(\id{M},\id{N})
  \item[s\_mu]  - storage upper bandwidth, \id{mu} $\le$ \id{s\_mu} $<$ \id{N}.
    The LU decomposition routine writes the LU factors into the storage 
    for A. The upper triangular factor U, however, may have 
    an upper bandwidth as big as min(\id{N}-1,\id{mu}+\id{ml}) because of 
    partial pivoting. The \id{s\_mu} field holds the upper half-bandwidth allocated for A.
  \item[ldim]  - leading dimension (\id{ldim} $\ge$ \id{s\_mu})
  \item[data]  - pointer to a contiguous block of \id{realtype} variables.
    The elements of a banded matrix of type \id{DlsMat} are      
    stored columnwise (i.e. columns are stored one on top  
    of the other in memory). Only elements within the      
    specified half-bandwidths are stored.     
    \id{data} is a pointer to \id{ldata} contiguous locations   
    which hold the elements within the band of A.  
  \item[ldata] - length of the data array ($=$ \id{ldim}$\cdot$(\id{s\_mu}+\id{ml}+1)
  \item[cols]  - array of pointers. \id{cols[j]} is a pointer to the uppermost element 
    within the band  in the j-th column. This pointer may be treated as   
    an array indexed from \id{s\_mu}$-$\id{mu} (to access the uppermost element within the 
    band in the j-th column) to \id{s\_mu}$+$\id{ml} (to access the lowest element     
    within the band in the j-th column). Indices from $0$ to \id{s\_mu}$-$\id{mu}$-1$ give 
    access to extra storage elements required by the LU decomposition function.
    Finally, \id{cols[j][i-j+s\_mu]} is the $(i,j)$-th element, $j-$\id{mu} $\le i \le j+$\id{ml}. 
\end{description}
%%
%%
\begin{figure}
\centerline{\psfig{figure=bandmat.eps,width=4.5 in}}
\caption[Diagram of the storage for a banded matrix of type \id{DlsMat}]
  {Diagram of the storage for a banded matrix of type \id{DlsMat}. Here \id{A} is an
  $N \times N$ band matrix of type \id{DlsMat} with upper and lower half-bandwidths \id{mu}
  and \id{ml}, respectively. The rows and columns of \id{A} are numbered from $0$ to $N-1$
  and the ($i,j$)-th element of \id{A} is denoted \id{A(i,j)}. The greyed out areas of
  the underlying component storage are used by the \id{BandGBTRF} and
  \id{BandGBTRS} routines.}\label{f:bandmat}
\end{figure}
%%
%%

% ------------------------------------------------------------------------------
\subsection{Accessor macros for the DLS modules}
% ------------------------------------------------------------------------------

The macros below allow a user to efficiently access individual matrix           
elements without writing out explicit data structure           
references and without knowing too much about the underlying   
element storage. The only storage assumption needed is that    
elements are stored columnwise and that a pointer to the \id{j}-th 
column of elements can be obtained via the \id{DENSE\_COL} or \id{BAND\_COL} macros.    
Users should use these macros whenever possible.               
\index{BAND@{\band} generic linear solver!type \id{DlsMat}|)}
\index{DENSE@{\dense} generic linear solver!type \id{DlsMat}|)}

\index{DENSE@{\dense} generic linear solver!macros|(}
The following two macros are defined by the {\dense} module to provide
access to data in the \id{DlsMat} type:
\begin{itemize}
\item \ID{DENSE\_ELEM}
  \par Usage : \id{DENSE\_ELEM(A,i,j) = a\_ij;} or
  \id{a\_ij = DENSE\_ELEM(A,i,j);}
  \par \id{DENSE\_ELEM} references the (\id{i},\id{j})-th element of the $M \times N$
  \id{DlsMat} \id{A}, $0 \le$ \id{i} $< M$, $0 \le$ \id{j} $< N$.
  
\item \ID{DENSE\_COL}
  \par Usage : \id{col\_j = DENSE\_COL(A,j);}
  \par \id{DENSE\_COL} references the \id{j}-th column of the $M \times N$
  \id{DlsMat} \id{A}, $0 \le$ \id{j} $< N$. The type of the expression          
  \id{DENSE\_COL(A,j)} is \id{realtype *} . After the assignment in the usage    
  above, \id{col\_j} may be treated as an array indexed from $0$ to $M-1$. 
  The (\id{i}, \id{j})-th element of \id{A} is referenced by \id{col\_j[i]}.  
\end{itemize}
\index{DENSE@{\dense} generic linear solver!macros|)}

\index{BAND@{\band} generic linear solver!macros|(}
The following three macros are defined by the {\band} module to provide
access to data in the \id{DlsMat} type:
\begin{itemize}
\item \ID{BAND\_ELEM}
  \par Usage : \id{BAND\_ELEM(A,i,j) = a\_ij;} or \id{a\_ij = BAND\_ELEM(A,i,j);}
  \par \id{BAND\_ELEM} references the (\id{i},\id{j})-th element of the
  $N \times N$ band matrix \id{A}, where $0 \le$ \id{i}, \id{j} $\le N-1$.
  The location (\id{i},\id{j}) should further satisfy 
  \id{j}$-$\id{(A->mu)} $\le$ \id{i} $\le$ \id{j}$+$\id{(A->ml)}.
\item \ID{BAND\_COL}
  \par Usage : \id{col\_j = BAND\_COL(A,j);}
  \par \id{BAND\_COL} references the diagonal element of the \id{j}-th
  column of the $N \times N$ band matrix \id{A}, $0 \le$ \id{j} $\le N-1$.
  The type of the expression \id{BAND\_COL(A,j)} is \id{realtype *}. 
  The pointer returned by the call \id{BAND\_COL(A,j)} can be treated as 
  an array which is indexed from $-$\id{(A->mu)} to \id{(A->ml)}.
\item \ID{BAND\_COL\_ELEM}
  \par Usage : \id{BAND\_COL\_ELEM(col\_j,i,j) = a\_ij;} or
  \id{a\_ij = BAND\_COL\_ELEM(col\_j,i,j);}
  \par This macro references the (\id{i},\id{j})-th entry of the band matrix \id{A}
  when used in conjunction with \id{BAND\_COL} to reference the \id{j}-th column through
  \id{col\_j}. The index (\id{i},\id{j}) should satisfy 
  \id{j}$-$\id{(A->mu)} $\le$ \id{i} $\le$ \id{j}$+$\id{(A->ml)}.
\end{itemize}
\index{BAND@{\band} generic linear solver!macros|)}



% ------------------------------------------------------------------------------
\subsection{Functions in the DENSE module}\label{ss:dense}
% ------------------------------------------------------------------------------

The {\dense} module defines two sets of functions with corresponding names.
The first set contains functions (with names starting with a capital letter)
that act on dense matrices of type \id{DlsMat}.  The second set contains functions
(with names starting with a lower case letter) that act on matrices represented 
as simple arrays.

\index{DENSE@{\dense} generic linear solver!functions!large matrix|(}
The following functions for \id{DlsMat} dense matrices are available
in the {\dense} package.  For full details, see the header files
\id{sundials\_direct.h} and \id{sundials\_dense.h}.
\begin{itemize}
\item \id{NewDenseMat}: allocation of a \id{DlsMat} dense matrix;
\item \id{DestroyMat}: free memory for a \id{DlsMat} matrix;
\item \id{PrintMat}: print a \id{DlsMat} matrix to standard output.
\item \id{NewLintArray}: allocation of an array of \id{long int} integers for use
  as pivots with \id{DenseGETRF} and \id{DenseGETRS};
\item \id{NewIntArray}: allocation of an array of \id{int} integers for use
  as pivots with the Lapack dense solvers;
\item \id{NewRealArray}: allocation of an array of \id{realtype} for use
  as right-hand side with \id{DenseGETRS};
\item \id{DestroyArray}: free memory for an array;
\item \id{SetToZero}: load a matrix with zeros;
\item \id{AddIdentity}: increment a square matrix by the identity matrix;
\item \id{DenseCopy}: copy one matrix to another;
\item \id{DenseScale}: scale a matrix by a scalar;
\item \id{DenseGETRF}: LU factorization with partial pivoting;
\item \id{DenseGETRS}: solution of $Ax = b$ using LU factorization (for square matrices $A$);
\item \id{DensePOTRF}: Cholesky factorization of a real symmetric positive matrix;
\item \id{DensePOTRS}: solution of $Ax = b$ using the Cholesky factorization of $A$;
\item \id{DenseGEQRF}: QR factorization of an $m \times n$ matrix, with $m \ge n$;
\item \id{DenseORMQR}: compute the product $w = Qv$, with $Q$ calculated using \id{DenseGEQRF};
\end{itemize}
\index{DENSE@{\dense} generic linear solver!functions!large matrix|)}

\index{DENSE@{\dense} generic linear solver!functions!small matrix|(}
The following functions for small dense matrices are available in the
{\dense} package:
%
\begin{itemize}

\item \ID{newDenseMat}
  \par \id{newDenseMat(m,n)} allocates storage for an \id{m} by \id{n}
  dense matrix. It returns a pointer to the newly allocated storage if            
  successful. If the memory request cannot be satisfied, then    
  \id{newDenseMat} returns \id{NULL}. The underlying type of the dense matrix 
  returned is \id{realtype**}. If we allocate a dense matrix \id{realtype** a} by 
  \id{a = newDenseMat(m,n)}, then \id{a[j][i]} references the (\id{i},\id{j})-th element   
  of the matrix \id{a}, $0 \le$ \id{i} $<$ \id{m}, $0 \le$ \id{j} $<n$, and \id{a[j]} 
  is a pointer to the first element in the \id{j}-th column of \id{a}. 
  The location \id{a[0]} contains a pointer to \id{m} $\times$ \id{n} contiguous 
  locations which contain the elements of \id{a}.

\item \ID{destroyMat}
  \par \id{destroyMat(a)} frees the dense matrix \id{a} allocated by \id{newDenseMat};

\item \ID{newLintArray}
  \par \id{newLintArray(n)} allocates an array of \id{n} integers, all \id{long int}.
  It returns a pointer to the first element in the array if successful. 
  It returns \id{NULL} if the memory request could not be satisfied.

\item \ID{newIntArray}
  \par \id{newIntArray(n)} allocates an array of \id{n} integers, all \id{int}.
  It returns a pointer to the first element in the array if successful. 
  It returns \id{NULL} if the memory request could not be satisfied.

\item \ID{newRealArray}
  \par \id{newRealArray(n)} allocates an array of \id{n} \id{realtype} values. 
  It returns a pointer to the first element in the array if successful. 
  It returns \id{NULL} if the memory request could not be satisfied.

\item \ID{destroyArray}
  \par \id{destroyArray(p)} frees the array \id{p} allocated by \id{newLintArray},
  \id{newIntArray}, or \id{newRealArray};

\item \ID{denseCopy}
  \par \id{denseCopy(a,b,m,n)} copies the \id{m} by \id{n} dense matrix \id{a} into the
  \id{m} by \id{n} dense matrix \id{b};

\item \ID{denseScale}
  \par \id{denseScale(c,a,m,n)} scales every element in the \id{m} by \id{n} dense
  matrix \id{a} by the scalar \id{c};

\item \ID{denseAddIdentity}
  \par \id{denseAddIdentity(a,n)} increments the {\em square} \id{n} by \id{n} dense matrix 
  \id{a} by the identity matrix $I_n$;

\item \ID{denseGETRF}
  \par \id{denseGETRF(a,m,n,p)} factors the \id{m} by \id{n} dense matrix \id{a},
  using Gaussian elimination with row pivoting. 
  It overwrites the elements of \id{a} with its LU factors and keeps track of the
  pivot rows chosen in the pivot array \id{p}.

  A successful LU factorization leaves the matrix \id{a} and the      
  pivot array \id{p} with the following information:                  
  \begin{enumerate}
  \item 
    \id{p[k]} contains the row number of the pivot element chosen   
    at the beginning of elimination step \id{k}, 
    \id{k} $ = 0, 1, ..., $\id{n}$-1$.  

  \item 
    If the unique LU factorization of \id{a} is given by $Pa = LU$,   
    where $P$ is a permutation matrix, $L$ is an \id{m} by \id{n}
    lower trapezoidal matrix with all diagonal elements equal to $1$, 
    and $U$ is an \id{n} by \id{n} upper triangular matrix, 
    then the upper triangular part of \id{a} (including its diagonal) 
    contains $U$ and the strictly lower trapezoidal part of \id{a} 
    contains the multipliers, $I-L$. 
    If \id{a} is square, $L$ is a unit lower triangular matrix.
                      
    \id{denseGETRF} returns 0 if successful. Otherwise it encountered a zero  
    diagonal element during the factorization, indicating that the matrix \id{a}
    does not have full column rank.
    In this case it returns the column index (numbered from one) at which it       
    encountered the zero.
    \end{enumerate}

\item \ID{denseGETRS}
  \par \id{denseGETRS(a,n,p,b)} solves the \id{n} by \id{n} linear system $ax = b$. 
  It assumes that \id{a} (of size \id{n} $\times$ \id{n}) has been LU-factored 
  and the pivot array \id{p} has been set by a successful call to 
  \id{denseGETRF(a,n,n,p)}. The solution $x$ is written into the \id{b} array.

\item \ID{densePOTRF}
  \par \id{densePOTRF(a,m)} calculates the Cholesky decomposition of the \id{m} by \id{m}
  dense matrix \id{a}, assumed to be symmetric positive definite. Only the lower triangle
  of \id{a} is accessed and overwritten with the Cholesky factor.

\item \ID{densePOTRS}
  \par \id{densePOTRS(a,m,b)} solves the \id{m} by \id{m} linear system $ax = b$.
  It assumes that the Cholesky factorization of \id{a} has been calculated in the
  lower triangular part of \id{a} by a successful call to \id{densePOTRF(a,m)}.

\item \ID{denseGEQRF}
  \par \id{denseGEQRF(a,m,n,beta,wrk)} calculates the QR decomposition of the \id{m} by \id{n}
  matrix \id{a} (\id{m} $\ge$ \id{n}) using Householder reflections. On exit, the elements
  on and above the diagonal of \id{a} contain the \id{n} by \id{n} upper triangular matrix \id{R};
  the elements below the diagonal, with the array \id{beta}, represent the orthogonal matrix \id{Q}
  as a product of elementary reflectors. The real array \id{wrk}, of length \id{m}, must be provided
  as temporary workspace.

\item \ID{denseORMQR}
  \par \id{denseORMQR(a,m,n,beta,v,w,wrk)} calculates the product $w = Qv$ for a given vector
  \id{v} of length \id{n}, where the orthogonal matrix $Q$ is encoded in the \id{m} by \id{n}
  matrix \id{a} and the vector \id{beta} of length \id{n}, after a successful call to
  \id{denseGEQRF(a,m,n,beta,wrk)}. The real array \id{wrk}, of length \id{m}, must be provided
  as temporary workspace.

\end{itemize}
\index{DENSE@{\dense} generic linear solver!functions!small matrix|)}


% ------------------------------------------------------------------------------
\subsection{Functions in the BAND module}\label{ss:band}
% ------------------------------------------------------------------------------

The {\band} module defines two sets of functions with corresponding names.
The first set contains functions (with names starting with a capital letter)
that act on band matrices of type \id{DlsMat}.  The second set contains functions
(with names starting with a lower case letter) that act on matrices represented 
as simple arrays.

\index{BAND@{\band} generic linear solver!functions|(}
The following functions for \id{DlsMat} banded matrices are available
in the {\band} package.  For full details, see the header files
\id{sundials\_direct.h} and \id{sundials\_band.h}.
\begin{itemize}
\item \id{NewBandMat}: allocation of a \id{DlsMat} band matrix;
\item \id{DestroyMat}: free memory for a \id{DlsMat} matrix;
\item \id{PrintMat}: print a \id{DlsMat} matrix to standard output.
\item \id{NewLintArray}: allocation of an array of \id{int} integers for use
  as pivots with \id{BandGBRF} and \id{BandGBRS};
\item \id{NewIntArray}: allocation of an array of \id{int} integers for use
  as pivots with the Lapack band solvers;
\item \id{NewRealArray}: allocation of an array of \id{realtype} for use
  as right-hand side with \id{BandGBRS};
\item \id{DestroyArray}: free memory for an array;
\item \id{SetToZero}: load a matrix with zeros;
\item \id{AddIdentity}: increment a square matrix by the identity matrix;
\item \id{BandCopy}: copy one matrix to another;
\item \id{BandScale}: scale a matrix by a scalar;
\item \id{BandGBTRF}: LU factorization with partial pivoting;
\item \id{BandGBTRS}: solution of $Ax = b$ using LU factorization;
\end{itemize}
\index{BAND@{\band} generic linear solver!functions|)}

\index{BAND@{\band} generic linear solver!functions!small matrix|(}
The following functions for small band matrices are available in the
{\band} package:
\begin{itemize}

\item \ID{newBandMat}
  \par \id{newBandMat(n, smu, ml)} allocates storage for an \id{n} by \id{n}
  band matrix with lower half-bandwidth \id{ml}.

\item \ID{destroyMat}
  \par \id{destroyMat(a)} frees the band matrix \id{a} allocated by \id{newBandMat};

\item \ID{newLintArray}
  \par \id{newLintArray(n)} allocates an array of \id{n} integers, all \id{long int}. 
  It returns a pointer to the first element in the array if successful. 
  It returns \id{NULL} if the memory request could not be satisfied.

\item \ID{newIntArray}
  \par \id{newIntArray(n)} allocates an array of \id{n} integers, all \id{int}.
  It returns a pointer to the first element in the array if successful. 
  It returns \id{NULL} if the memory request could not be satisfied.

\item \ID{newRealArray}
  \par \id{newRealArray(n)} allocates an array of \id{n} \id{realtype} values. 
  It returns a pointer to the first element in the array if successful. 
  It returns \id{NULL} if the memory request could not be satisfied.

\item \ID{destroyArray}
  \par \id{destroyArray(p)} frees the array \id{p} allocated by \id{newLintArray},
  \id{newIntArray}, or \id{newRealArray};

\item \ID{bandCopy}
  \par \id{bandCopy(a,b,n,a\_smu, b\_smu,copymu, copyml)} copies the \id{n} by \id{n} band 
  matrix \id{a} into the \id{n} by \id{n} band matrix \id{b};

\item \ID{bandScale}
  \par \id{bandScale(c,a,n,mu,ml,smu)} scales every element in the \id{n} by \id{n} band
  matrix \id{a} by \id{c};

\item \ID{bandAddIdentity}
  \par \id{bandAddIdentity(a,n,smu)} increments the \id{n} by \id{n} band matrix \id{a} by the
  identity matrix;

\item \ID{bandGETRF}
  \par \id{bandGETRF(a,n,mu,ml,smu,p)} factors the \id{n} by \id{n} band matrix \id{a},
  using Gaussian elimination with row pivoting. 
  It overwrites the elements of \id{a} with its LU factors and keeps track of the
  pivot rows chosen in the pivot array \id{p}.

\item \ID{bandGETRS}
  \par \id{bandGETRS(a,n,smu,ml,p,b)} solves the \id{n} by \id{n} linear system $ax = b$. 
  It assumes that \id{a} (of size \id{n} $\times$ \id{n}) has been LU-factored 
  and the pivot array \id{p} has been set by a successful call to 
  \id{bandGETRF(a,n,mu,ml,smu,p)}. The solution $x$ is written into the \id{b} array.

\end{itemize}
\index{BAND@{\band} generic linear solver!functions!small matrix|)}





% ====================================================================================
\section{The SPILS modules: SPGMR, SPBCG, and SPTFQMR}\label{s:spils}
% ====================================================================================

{\warn}A linear solver module from the {\em spils} family can only be used in conjunction 
with an actual {\nvector} implementation library, such as the {\nvecs} or {\nvecp} provided 
with {\sundials}.


% ------------------------------------------------------------------------------
\subsection{The SPGMR module}\label{ss:spgmr}
% ------------------------------------------------------------------------------

\index{generic linear solvers!SPGMR@{\spgmr}}
\index{SPGMR@{\spgmr} generic linear solver!description of|(}
The {\spgmr} package, in the files \id{sundials\_spgmr.h} and
\id{sundials\_spgmr.c}, includes an implementation of the scaled
preconditioned GMRES\index{GMRES method} method.  A separate code
module, implemented in \id{sundials\_iterative.(h,c)}, contains
auxiliary functions that support {\spgmr}, as well as the other Krylov
solvers in {\sundials} ({\spbcg} and {\sptfqmr}).
For full details, including usage instructions, see the header
files \id{sundials\_spgmr.h} and \id{sundials\_iterative.h}.
\index{SPGMR@{\spgmr} generic linear solver!description of|)}

The files comprising the {\spgmr} generic linear solver, and their locations
in the {\sundials} {\em srcdir}, are as follows:
\begin{itemize}
\item header files (located in {\em srcdir}\id{/include/sundials})\\
  \id{sundials\_spgmr.h} \id{sundials\_iterative.h} \id{sundials\_nvector.h} \\
  \id{sundials\_types.h} \id{sundials\_math.h}  \id{sundials\_config.h}
\item source files (located in {\em srcdir}\id{/src/sundials})\\
  \id{sundials\_spgmr.c} \id{sundials\_iterative.c} \id{sundials\_nvector.c}
\end{itemize}
Only two of the preprocessing directives in the header file \id{sundials\_config.h} 
are required to use the {\spgmr} package by itself (see \S\ref{ss:no_config} for details):
\begin{itemize}
\item (required) definition of the precision of the {\sundials} type \id{realtype}. 
  One of the following lines must be present:\\
  \id{\#define SUNDIALS\_DOUBLE\_PRECISION 1}\\
  \id{\#define SUNDIALS\_SINGLE\_PRECISION 1}\\
  \id{\#define SUNDIALS\_EXTENDED\_PRECISION 1}
\item (optional) use of generic math functions:\\
  \id{\#define SUNDIALS\_USE\_GENERIC\_MATH 1}
\end{itemize}
The \id{sundials\_types.h} header file defines the {\sundials}
\id{realtype} and \id{booleantype} types and the macro \id{RCONST}, while the 
\id{sundials\_math.h} header file is needed for the \id{MAX} and \id{ABS} macros
and \id{RAbs} and \id{RSqrt} functions.

The generic {\nvector} files, \id{sundials\_nvector.(h,c)} are needed for the
definition of the generic \id{N\_Vector} type and functions. 
The {\nvector} functions used by the {\spgmr} module are: 
\id{N\_VDotProd}, \id{N\_VLinearSum}, \id{N\_VScale}, \id{N\_VProd}, \id{N\_VDiv}, 
\id{N\_VConst}, \id{N\_VClone}, \id{N\_VCloneVectorArray}, \id{N\_VDestroy}, and
\id{N\_VDestroyVectorArray}.

The nine files listed above can be extracted from the {\sundials} {\em srcdir} and
compiled by themselves into an {\spgmr} library or into a larger user code.
 
\index{SPGMR@{\spgmr} generic linear solver!functions|(}
The following functions are available in the {\spgmr} package:  
\begin{itemize}
\item \id{SpgmrMalloc}: allocation of memory for \id{SpgmrSolve};
\item \id{SpgmrSolve}: solution of $Ax = b$ by the {\spgmr} method;
\item \id{SpgmrFree}: free memory allocated by \id{SpgmrMalloc}.
\end{itemize}
\index{SPGMR@{\spgmr} generic linear solver!functions|)}
%
\index{SPGMR@{\spgmr} generic linear solver!support functions|(}
The following functions are available in the support package 
\id{sundials\_iterative.(h,c)}:
\begin{itemize}
\item \id{ModifiedGS}: performs modified Gram-Schmidt procedure;
\item \id{ClassicalGS}: performs classical Gram-Schmidt procedure; 
\item \id{QRfact}: performs QR factorization of Hessenberg matrix;
\item \id{QRsol}: solves a least squares problem with a Hessenberg
       matrix factored by \id{QRfact}.
\end{itemize}
\index{SPGMR@{\spgmr} generic linear solver!support functions|)}


% ------------------------------------------------------------------------------
\subsection{The SPBCG module}\label{ss:spbcg}
% ------------------------------------------------------------------------------

\index{generic linear solvers!SPBCG@{\spbcg}}
\index{SPBCG@{\spbcg} generic linear solver!description of|(}
The {\spbcg} package, in the files \id{sundials\_spbcgs.h} and
\id{sundials\_spbcgs.c}, includes an implementation of the scaled
preconditioned Bi-CGStab\index{Bi-CGStab method} method.
For full details, including usage instructions, see the file \id{sundials\_spbcgs.h}.
\index{SPBCG@{\spbcg} generic linear solver!description of|)}

The files needed to use the {\spbcg} module by itself are the same as for the
{\spgmr} module, but with \id{sundials\_spbcgs.(h,c)} in place of
\id{sundials\_spgmr.(h,c)}.

\index{SPBCG@{\spbcg} generic linear solver!functions|(}
The following functions are available in the {\spbcg} package:  
\begin{itemize}
\item \id{SpbcgMalloc}: allocation of memory for \id{SpbcgSolve};
\item \id{SpbcgSolve}: solution of $Ax = b$ by the {\spbcg} method;
\item \id{SpbcgFree}: free memory allocated by \id{SpbcgMalloc}.
\end{itemize}
\index{SPBCG@{\spbcg} generic linear solver!functions|)}



% ------------------------------------------------------------------------------
\subsection{The SPTFQMR module}\label{ss:sptfqmr}
% ------------------------------------------------------------------------------

\index{generic linear solvers!SPTFQMR@{\sptfqmr}}
\index{SPTFQMR@{\sptfqmr} generic linear solver!description of|(}
The {\sptfqmr} package, in the files \id{sundials\_sptfqmr.h} and
\id{sundials\_sptfqmr.c}, includes an implementation of the scaled
preconditioned TFQMR\index{TFQMR method} method.  For full details,
including usage instructions, see the file \id{sundials\_sptfqmr.h}.
\index{SPTFQMR@{\sptfqmr} generic linear solver!description of|)}

The files needed to use the {\sptfqmr} module by itself are the same as for the
{\spgmr} module, but with \id{sundials\_sptfqmr.(h,c)} in place of
\id{sundials\_spgmr.(h,c)}.

\index{SPTFQMR@{\sptfqmr} generic linear solver!functions|(}
The following functions are available in the {\sptfqmr} package:  
\begin{itemize}
\item \id{SptfqmrMalloc}: allocation of memory for \id{SptfqmrSolve};
\item \id{SptfqmrSolve}: solution of $Ax = b$ by the {\sptfqmr} method;
\item \id{SptfqmrFree}: free memory allocated by \id{SptfqmrMalloc}.
\end{itemize}
\index{SPTFQMR@{\sptfqmr} generic linear solver!functions|)}