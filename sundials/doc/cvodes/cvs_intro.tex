%===================================================================================
\chapter{Introduction}\label{s:intro}
%===================================================================================

{\cvodes}~\cite{SeHi:05} is part of a software family called {\sundials}: 
SUite of Nonlinear and DIfferential/ALgebraic equation Solvers~\cite{HBGLSSW:05}.  
This suite consists of {\cvode}, {\kinsol} and {\ida}, and variants of these
with sensitivity analysis capabilities.
%
{\cvodes}\index{CVODES@{\cvodes}!brief description of} is a solver for
stiff and nonstiff initial value problems (IVPs) for systems of
ordinary differential equation (ODEs). In addition to solving stiff
and nonstiff ODE systems, {\cvodes} has sensitivity analysis
capabilities, using either the forward or the adjoint methods.

%---------------------------------
\section{Historical background}\label{ss:history}
%---------------------------------

\index{CVODES@{\cvodes}!relationship to {\vode}, {\vodpk}|(} {\F}
solvers for ODE initial value problems are widespread and heavily
used.  Two solvers that were previously written at LLNL are {\vode}
\cite{BBH:89} and {\vodpk}~\cite{Byr:92}.  {\vode}\index{VODE@{\vode}}
is a general-purpose solver that includes methods for both stiff and
nonstiff systems, and in the stiff case uses direct methods (full or
banded) for the solution of the linear systems that arise at each
implicit step. Externally, {\vode} is very similar to the well known
solver {\lsode}\index{LSODE@{\lsode}}~\cite{RaHi:94}.
{\vodpk}\index{VODPK@{\vodpk}} is a variant of {\vode} that uses a
preconditioned Krylov (iterative) method, namely GMRES, for the
solution of the linear systems. {\vodpk} is a powerful tool for large
stiff systems because it combines established methods for stiff
integration, nonlinear iteration, and Krylov (linear) iteration with a
problem-specific treatment of the dominant source of stiffness, in the
form of the user-supplied preconditioner matrix~\cite{BrHi:89}.  The
capabilities of both {\vode} and {\vodpk} were combined in the
{\C}-language package {\cvode}\index{CVODE@{\cvode}}~\cite{CoHi:94,
CoHi:96}.

At present, {\cvodes} contains three Krylov methods that can be used
in conjuction with Newton iteration:
the GMRES (Generalized Minimal RESidual)~\cite{SaSc:86},
Bi-CGStab (Bi-Conjugate Gradient Stabilized)~\cite{Van:92}, and
TFQMR (Transpose-Free Quasi-Minimal Residual) linear iterative methods
\cite{Fre:93}.  As Krylov methods, these require almost no matrix storage
for solving the Newton equations as compared to direct methods.
However, the algorithms allow for a user-supplied preconditioner
matrix, and for most problems preconditioning is essential for an
efficient solution.
For very large stiff ODE systems, the Krylov methods are preferable over
direct linear solver methods, and are often the only feasible choice.
Among the three Krylov methods in {\cvodes}, we recommend GMRES as the
best overall choice.  However, users are encouraged to compare all
three, especially if encountering convergence failures with GMRES.
Bi-CGFStab and TFQMR have an advantage in storage requirements, in
that the number of workspace vectors they require is fixed, while that
number for GMRES depends on the desired Krylov subspace size.

In the process of translating the {\vode} and {\vodpk} algorithms into
{\C}, the overall {\cvode} organization has changed considerably.  One
key feature of the {\cvode} organization is that the linear system
solvers comprise a layer of code modules that is separated from the
integration algorithm, thus allowing for easy modification and
expansion of the linear solver array.  A second key feature is a
separate module devoted to vector operations; this facilitated the
extension to multiprosessor environments with only a minimal impact on
the rest of the solver, resulting in {\pvode}\index{PVODE@{\pvode}}
\cite{ByHi:99}, the parallel variant of {\cvode}.
\index{CVODES@{\cvodes}!relationship to {\vode}, {\vodpk}|)}

\index{CVODES@{\cvodes}!relationship to {\cvode}, {\pvode}|(}
{\cvodes} is written with a functionality that is a superset of that
of the pair {\cvode}/{\pvode}. Sensitivity analysis capabilities, both
forward and adjoint, have been added to the main integrator. Enabling
forward sensititivity computations in {\cvodes} will result in the
code integrating the so-called {\em sensitivity equations}
simultaneously with the original IVP, yielding both the solution and
its sensitivity with respect to parameters in the model. Adjoint
sensitivity analysis, most useful when the gradients of relatively few
functionals of the solution with respect to many parameters are
sought, involves integration of the original IVP forward in time
followed by the integration of the so-called {\em adjoint equations}
backward in time. {\cvodes} provides the infrastructure needed to
integrate any final-condition ODE dependent on the solution of the
original IVP (in particular the adjoint system).

Development of {\cvodes} was concurrent with a redesign of the vector
operations module across the {\sundials} suite. The key feature of the
new {\nvector} module is that it is written in terms of abstract
vector operations with the actual vector functions attached by a
particular implementation (such as serial or parallel) of
{\nvector}. This allows writing the {\sundials} solvers in a manner
independent of the actual {\nvector} implementation (which can be
user-supplied), as well as allowing more than one {\nvector} module to
be linked into an executable file.
\index{CVODES@{\cvodes}!relationship to {\cvode}, {\pvode}|)}

\index{CVODES@{\cvodes}!motivation for writing in C|(} There were
several motivations for choosing the {\C} language for {\cvode}, and
later for {\cvodes}.  First, a general movement away from {\F} and
toward {\C} in scientific computing was and still is apparent.
Second, the pointer, structure, and dynamic memory allocation features
in {\C} are extremely useful in software of this complexity.  Finally,
we prefer {\C} over {\CPP} for {\cvodes} because of the wider
availability of {\C} compilers, the potentially greater efficiency of
{\C}, and the greater ease of interfacing the solver to applications
written in extended {\F}.  \index{CVODES@{\cvodes}!motivation for
writing in C|)}

\section{Changes from previous versions}

\subsection*{Changes in v2.7.0}

One significant design change was made with this release: The problem
size and its relatives, bandwidth parameters, related internal indices,
pivot arrays, and the optional output \id{lsflag} have all been
changed from type \id{int} to type \id{long int}, except for the
problem size and bandwidths in user calls to routines specifying
BLAS/LAPACK routines for the dense/band linear solvers.  The function
\id{NewIntArray} is replaced by a pair \id{NewIntArray}/\id{NewLintArray},
for \id{int} and \id{long int} arrays, respectively.  In a minor
change to the user interface, the type of the index \id{which} in
CVODES was changed from \id{long int} to \id{int}.

Errors in the logic for the integration of backward problems were
identified and fixed.

A large number of minor errors have been fixed.  Among these are the following:
In \id{CVSetTqBDF}, the logic was changed to avoid a divide by zero.
After the solver memory is created, it is set to zero before being filled.
In each linear solver interface function, the linear solver memory is
freed on an error return, and the \id{**Free} function now includes a
line setting to NULL the main memory pointer to the linear solver memory.
In the rootfinding functions \id{cvRcheck1}/\id{cvRcheck2}, when an exact
zero is found, the array \id{glo} of $g$ values at the left endpoint is
adjusted, instead of shifting the $t$ location \id{tlo} slightly.
In the installation files, we modified the treatment of the macro
SUNDIALS\_USE\_GENERIC\_MATH, so that the parameter GENERIC\_MATH\_LIB is
either defined (with no value) or not defined.


\subsection*{Changes in v2.6.0}

Two new features related to the integration of ODE IVP problems were added 
in this release: (a) a new linear solver module, based on Blas and Lapack 
for both dense and banded matrices, and (b) an option to specify which direction 
of zero-crossing is to be monitored while performing rootfinding. 

This version also includes several new features related to sensitivity analysis,
among which are: (a) support for integration of quadrature equations depending
on both the states and forward sensitivity (and thus support for forward sensitivity
analysis of quadrature equations), (b) support for simultaneous integration of 
multiple backward problems based on the same underlying ODE (e.g., for use
in an {\em forward-over-adjoint} method for computing second order derivative
information), (c) support for backward integration of ODEs and quadratures 
depending on both forward states and sensitivities (e.g., for use in computing 
second-order derivative information), and (d) support for reinitialization of 
the adjoint module.

The user interface has been further refined. Some of the API changes involve:
(a) a reorganization of all linear solver modules into two families (besides 
the existing family of scaled preconditioned iterative linear solvers,
the direct solvers, including the new Lapack-based ones, were also organized 
into a {\em direct} family); (b) maintaining a single pointer to user data,
optionally specified through a \id{Set}-type function; (c) a general 
streamlining of the preconditioner modules distributed with the solver.
Moreover, the prototypes of all functions related to integration of backward 
problems were modified to support the simultaneous integration of multiple 
problems. All backward problems defined by the user are internally managed
through a linked list and identified in the user interface through a
unique identifier.


\subsection*{Changes in v2.5.0}

The main changes in this release involve a rearrangement of the entire 
{\sundials} source tree (see \S\ref{ss:sun_org}). At the user interface 
level, the main impact is in the mechanism of including {\sundials} header
files which must now include the relative path (e.g. \id{\#include <cvode/cvode.h>}).
Additional changes were made to the build system: all exported header files are
now installed in separate subdirectories of the instaltion {\em include} directory.

In the adjoint solver module, the following two bugs were fixed: in \id{CVodeF}
the solver was sometimes incorrectly taking an additional step before
returning control to the user (in \id{CV\_NORMAL} mode) thus leading to 
a failure in the interpolated output function; in \id{CVodeB}, while searching
for the current check point, the solver was sometimes reaching outside the
integration interval resulting in a segmentation fault.

The functions in the generic dense linear solver (\id{sundials\_dense} and
\id{sundials\_smalldense}) were modified to work for rectangular $m \times n$
matrices ($m \le n$), while the factorization and solution functions were
renamed to \id{DenseGETRF}/\id{denGETRF} and \id{DenseGETRS}/\id{denGETRS}, 
respectively.
The factorization and solution functions in the generic band linear solver were 
renamed \id{BandGBTRF} and \id{BandGBTRS}, respectively.

\subsection*{Changes in v2.4.0}

{\cvspbcg} and {\cvsptfqmr} modules have been added to interface with the
Scaled Preconditioned Bi-CGstab ({\spbcg}) and Scaled Preconditioned
Transpose-Free Quasi-Minimal Residual ({\sptfqmr}) linear solver modules,
respectively (for details see Chapter \ref{s:simulation}).
At the same time, function type names for Scaled Preconditioned Iterative
Linear Solvers were added for the user-supplied Jacobian-times-vector and
preconditioner setup and solve functions.

A new interpolation method was added to the {\cvodes} adjoint module. The
function \id{CVadjMalloc} has an additional argument which can be used to select
the desired interpolation scheme.

The deallocation functions now take as arguments the address of the respective 
memory block pointer.

To reduce the possibility of conflicts, the names of all header files have
been changed by adding unique prefixes (\id{cvodes\_} and \id{sundials\_}).
When using the default installation procedure, the header files are exported
under various subdirectories of the target \id{include} directory. For more
details see Appendix \ref{c:install}.

\subsection*{Changes in v2.3.0}

A minor bug was fixed in the interpolation functions of the adjoint
{\cvodes} module.

\subsection*{Changes in v2.2.0}

The user interface has been further refined. Several functions used
for setting optional inputs were combined into a single one.  An
optional user-supplied routine for setting the error weight vector was
added.  Additionally, to resolve potential variable scope issues, all
SUNDIALS solvers release user data right after its use. The build
systems has been further improved to make it more robust.

\subsection*{Changes in v2.1.2}

A bug was fixed in the \id{CVode} function that was potentially
leading to erroneous behaviour of the rootfinding procedure on the 
integration first step.

\subsection*{Changes in v2.1.1}

This {\cvodes} release includes bug fixes related to forward sensitivity
computations (possible loss of accuray on a BDF order increase and incorrect
logic in testing user-supplied absolute tolerances). 
In addition, we have added the option of activating and deactivating
forward sensitivity calculations on successive {\cvodes} runs without memory
allocation/deallocation.

Other changes in this minor {\sundials} release affect the build system.

\subsection*{Changes in v2.1.0}

The major changes from the previous version involve a redesign of the
user interface across the entire {\sundials} suite. We have eliminated the
mechanism of providing optional inputs and extracting optional statistics 
from the solver through the \id{iopt} and \id{ropt} arrays. Instead,
{\cvodes} now provides a set of routines (with prefix \id{CVodeSet})
to change the default values for various quantities controlling the
solver and a set of extraction routines (with prefix \id{CVodeGet})
to extract statistics after return from the main solver routine.
Similarly, each linear solver module provides its own set of {\id{Set}-}
and {\id{Get}-type} routines. For more details see \S\ref{ss:optional_input}
and \S\ref{ss:optional_output}.

Additionally, the interfaces to several user-supplied routines
(such as those providing Jacobians, preconditioner information, and
sensitivity right hand sides) were simplified by reducing the number
of arguments. The same information that was previously accessible
through such arguments can now be obtained through {\id{Get}-type}
functions.

The rootfinding feature was added, whereby the roots of a set of given
functions may be computed during the integration of the ODE system.

Installation of {\cvodes} (and all of {\sundials}) has been completely 
redesigned and is now based on a configure script.


\section{Reading this user guide}\label{ss:reading}

This user guide is a combination of general usage instructions.
Specific example programs are provided as a separate document.
We expect that some readers will want to concentrate on the general 
instructions, while others will refer mostly to the examples.

There are different possible levels of usage of {\cvodes}. The most casual
user, with an IVP problem only, can get by with reading \S\ref{ss:ivp_sol}, 
then  Chapter \ref{s:simulation} through \S\ref{sss:cvode} only, and looking
at examples in~\cite{cvodes_ex}.
In addition, to solve a forward sensitivity problem the user should read 
\S\ref{ss:fwd_sensi}, followed by Chapter \ref{s:forward} through 
\S\ref{ss:sensi_get} only, and look at examples in~\cite{cvodes_ex}.

In a different direction, a more advanced user with an IVP problem may want to 
(a) use a package preconditioner (\S\ref{ss:preconds}), 
(b) supply his/her own Jacobian or preconditioner routines (\S\ref{ss:user_fct_sim}),
(c) do multiple runs of problems of the same size (\S\ref{sss:cvreinit}), 
(d) supply a new {\nvector} module (Chapter \ref{s:nvector}), or even 
(e) supply a different linear solver module (\S\ref{ss:cvodes_org}).
%
An advanced user with a forward sensitivity problem may also want to
(a) provide his/her own sensitivity equations right-hand side routine
(\S\ref{s:user_fct_fwd}), (b) perform multiple runs with the same number of
sensitivity parameters (\S\ref{ss:sensi_malloc}), or (c) extract additional
diagnostic information (\S\ref{ss:sensi_get}).
%
A user with an adjoint sensitivity problem needs to understand the IVP 
solution approach at the desired level and also go through 
\S\ref{ss:adj_sensi} for a short mathematical description of the adjoint
approach, Chapter \ref{s:adjoint} for the usage of the adjoint module in {\cvodes},
and the examples in~\cite{cvodes_ex}.

The structure of this document is as follows:
\begin{itemize}
\item
  In Chapter \ref{s:math}, we give short descriptions of the numerical 
  methods implemented by {\cvodes} for the solution of initial value problems
  for systems of ODEs, continue with short descriptions of preconditioning
  (\S\ref{s:preconditioning}), stability limit detection (\S\ref{s:bdf_stab}),
  and rootfinding (\S\ref{ss:rootfinding}), and conclude with an overview of the
  mathematical aspects of sensitivity analysis, both forward (\S\ref{ss:fwd_sensi})
  and adjoint (\S\ref{ss:adj_sensi}).
\item
  The following chapter describes the structure of the {\sundials} suite
  of solvers (\S\ref{ss:sun_org}) and the software organization of the {\cvodes}
  solver (\S\ref{ss:cvodes_org}). 
\item
  Chapter \ref{s:simulation} is the main usage document for {\cvodes}
  for simulation applications.  It includes a complete description of
  the user interface for the integration of ODE initial value problems.
  Readers that are not interested in using {\cvodes} for sensitivity
  analysis can then skip the next two chapters.
\item
  Chapter \ref{s:forward} describes the usage of {\cvodes} for forward
  sensitivity analysis as an extension of its IVP integration
  capabilities.  We begin with a skeleton of the user main program,
  with emphasis on the steps that are required in addition to those
  already described in Chapter \ref{s:simulation}.  Following that we
  provide detailed descriptions of the user-callable interface
  routines specific to forward sensitivity analysis and of the
  additonal optional user-defined routines.
\item
  Chapter \ref{s:adjoint} describes the usage of {\cvodes} for adjoint
  sensitivity analysis. We begin by describing the {\cvodes} checkpointing 
  implementation for interpolation of the original IVP solution during
  integration of the adjoint system backward in time, and with 
  an overview of a user's main program. Following that we provide complete
  descriptions of the user-callable interface routines for adjoint sensitivity
  analysis as well as descriptions of the required additional user-defined routines.
\item
  Chapter \ref{s:nvector} gives a brief overview of the generic
  {\nvector} module shared amongst the various components of
  {\sundials}, as well as details on the two {\nvector}
  implementations provided with {\sundials}: a serial implementation
  (\S\ref{ss:nvec_ser}) and a parallel implementation based on
  {\mpi}\index{MPI} (\S\ref{ss:nvec_par}).
\item
  Chapter \ref{s:new_linsolv} describes the specifications of linear
  solver modules as supplied by the user.
\item
  Chapter \ref{s:gen_linsolv} describes in detail the generic linear solvers shared 
  by all {\sundials} solvers.
\item
  Finally, in the appendices, we provide detailed instructions for the installation
  of {\cvodes}, within the structure of {\sundials} (Appendix \ref{c:install}), as well
  as a list of all the constants used for input to and output from {\cvodes} functions
  (Appendix \ref{c:constants}).
\end{itemize}

Finally, the reader should be aware of the following notational conventions
in this user guide:  Program listings and identifiers (such as \id{CVodeInit}) 
within textual explanations appear in typewriter type style; 
fields in {\C} structures (such as {\em content}) appear in italics;
and packages or modules, such as {\cvdense}, are written in all capitals.
Usage and installation instructions that constitute important warnings
are marked with a triangular symbol {\warn} in the margin.
