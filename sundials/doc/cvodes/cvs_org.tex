%===================================================================================
\chapter{Code Organization}\label{s:organization}
%===================================================================================

%----------------------------------
\section{SUNDIALS organization}\label{ss:sun_org}
%----------------------------------
% This is a shared SUNDIALS TEX file with description of
% the SUNDIALS organization
%
The family of solvers referred to as {\sundials} consists of the solvers
{\cvode} (for ODE systems), {\kinsol} (for nonlinear algebraic
systems), and {\ida} (for differential-algebraic systems).  In addition,
{\sundials} also includes variants of {\cvode} and {\ida} with sensitivity analysis 
capabilities (using either forward or adjoint methods): {\cvodes} and {\idas},
respectively.

The various solvers of this family share many subordinate modules.
For this reason, it is organized as a family, with a directory
structure that exploits that sharing (see Fig. \ref{f:sunorg}).
%%
%%
\begin{figure}
\subfigure[High-level diagram (note that none of the Lapack-based linear solver
 modules are represented.)]
{\centerline{\psfig{figure=sunorg1.eps,width=\textwidth}}}
\subfigure[Directory structure of the source tree]
{\centerline{\psfig{figure=sunorg2.eps,width=\textwidth}}}
\caption {Organization of the SUNDIALS suite}\label{f:sunorg}
\end{figure}
%%
%%
The following is a list of the solver packages presently available:
\begin{itemize}

\item {\cvode},  
  a solver for stiff and nonstiff ODEs $dy/dt = f(t,y)$;

\item {\cvodes},
  a solver for stiff and nonstiff ODEs
  with sensitivity analysis capabilities;

\item {\ida},
  a solver for differential-algebraic systems $F(t,y,\dot{y}) = 0$;

\item {\idas},
  a solver for differential-algebraic systems
  with sensitivity analysis capabilities;

\item {\kinsol}, 
  a solver for nonlinear algebraic systems $F(u) = 0$.

\end{itemize}


%----------------------------------
\section{CVODES organization}\label{ss:cvodes_org}
%----------------------------------

\index{CVODES@{\cvodes}!package structure}
The {\cvodes} package is written in ANSI {\C}. The following
summarizes the basic structure of the package, although knowledge
of this structure is not necessary for its use.

The overall organization of the {\cvodes} package is shown in Figure
\ref{f:cvsorg}.  The basic elements of the structure are a module for
the basic integration algorithm (including forward sensitivity analysis),
a module for adjoint sensitivity analysis, and a set of modules for the solution
of linear systems that arise in the case of a stiff system.  
\begin{figure}
{\centerline{\psfig{figure=cvsorg.eps,width=\textwidth}}}
\caption [Overall structure of the CVODES package]
{Overall structure of the {\cvodes} package.
  Modules specific to {\cvodes} are distinguished by rounded boxes, while 
  generic solver and auxiliary modules are in rectangular boxes. 
  Note that the direct linear solvers using Lapack implementations are not 
  explicitly represented.}
\label{f:cvsorg}
\end{figure}
The central integration module, implemented in the files \id{cvodes.h},
\id{cvodes\_impl.h}, and \id{cvodes.c}, deals 
with the evaluation of integration coefficients,
the functional or Newton iteration process, estimation of local error,
selection of step size and order, and interpolation to user output
points, among other issues.  Although this module contains logic for
the basic Newton iteration algorithm, it has no knowledge of the
method being used to solve the linear systems that arise.  For any
given user problem, one of the linear system modules is specified, and
is then invoked as needed during the integration. 

In addition, if forward sensitivity analysis is turned on, the main module 
will integrate the forward sensitivity equations simultaneously with the original
IVP. The sensitivity variables may be included in the local error control
mechanism of the main integrator.
\index{forward sensitivity analysis!correction strategies}
{\cvodes} provides three different strategies for dealing with the correction
stage for the sensitivity variables: \Id{CV\_SIMULTANEOUS}, \Id{CV\_STAGGERED} and
\Id{CV\_STAGGERED1} (see \S\ref{ss:fwd_sensi} and \S\ref{ss:sensi_malloc}).
The {\cvodes} package includes an algorithm for the approximation of the
sensitivity equations right-hand sides by difference quotients, but the user has
the option of supplying these right-hand sides directly.

\index{adjoint sensitivity analysis!implementation in {\cvodes}|(}
The adjoint sensitivity module (file \id{cvodea.c}) provides the infrastructure needed for the 
backward integration of any system of ODEs which depends on the solution 
of the original IVP, in particular the adjoint system and any quadratures required
in evaluating the gradient of the objective functional.  This module deals with
the setup of the checkpoints, the interpolation of the forward solution during
the backward integration, and the backward integration of the adjoint equations.
\index{adjoint sensitivity analysis!implementation in {\cvodes}|)}


\index{CVODES@{\cvodes} linear solvers!list of|(} 
At present, the package includes the following eight {\cvodes} linear algebra
modules, organized into two families. The {\em direct} familiy of linear
solvers provides solvers for the direct solution of linear systems with
dense or banded matrices and includes:
\begin{itemize} 
\item {\cvdense}: LU factorization and backsolving with dense matrices
  (using either an internal implementation or Blas/Lapack); 
\item {\cvband}: LU factorization and backsolving with banded matrices
  (using either an internal implementation or Blas/Lapack); 
\end{itemize}
The {\em spils} family of linear solvers provides scaled preconditioned
iterative linear solvers and includes:
\begin{itemize} 
\item {\cvspgmr}: scaled preconditioned GMRES method;
\item {\cvspbcg}: scaled preconditioned Bi-CGStab method;
\item {\cvsptfqmr}: scaled preconditioned TFQMR method.
\end{itemize}
Additionally, {\cvode} includes:
\begin{itemize}
\item {\cvdiag}: an internally generated diagonal approximation to the 
  Jacobian; 
\end{itemize}
The set of linear solver modules distributed with {\cvodes} is intended to be expanded in the
future as new algorithms are developed.
\index{CVODES@{\cvodes} linear solvers!list of|)} 

In the case of the direct methods {\cvdense} and {\cvband}
the package includes an algorithm for the approximation of the Jacobian by difference
quotients, but the user also has the option of supplying the Jacobian
(or an approximation to it) directly.  In the case of the Krylov iterative
methods {\cvspgmr}, {\cvspbcg}, and {\cvsptfqmr}, the package includes an algorithm
for the approximation by difference quotients of the product between the Jacobian
matrix and a vector of appropriate length. Again, the user has the option of
providing a routine for this operation.
For the Krylov methods, \index{preconditioning!setup and solve phases}
the preconditioner must be supplied by the user, in two phases: 
setup (preprocessing of Jacobian data) and solve.
While\index{preconditioning!advice on} there is no default choice of
preconditioner analogous to the difference-quotient approximation
in the direct case, the references \cite{BrHi:89, Byr:92},
together with the example programs included with {\cvodes}, offer
considerable assistance in building preconditioners.

\index{CVODES@{\cvodes} linear solvers!implementation details|(} 
Each {\cvodes} linear solver module consists of four routines devoted to (1)
memory allocation and initialization, (2) setup of the matrix data
involved, (3) solution of the system, and (4) freeing of memory.  
The setup and solution phases are separate because the evaluation of
Jacobians and preconditioners is done only periodically during the
integration, and only as required to achieve convergence. The call list within
the central {\cvodes} module for each of the five associated functions is
fixed, thus allowing the central module to be completely independent
of the linear system method.
\index{CVODES@{\cvodes} linear solvers!implementation details|)} 

These modules are also decomposed in another way.
\index{generic linear solvers!use in {\cvodes}|(} 
With the exception of {\cvdiag} and the modules interfacing to Lapack linear solvers,
each of the modules {\cvdense}, {\cvband}, {\cvspgmr}, {\cvspbcg}, and
{\cvsptfqmr} is a set of interface routines built on top of a generic solver
module, named {\dense}, {\band}, {\spgmr}, {\spbcg}, and {\sptfqmr}, respectively.
The interfaces deal with the use of these methods in the {\cvodes} context, 
whereas the generic solver is independent of the context.
While the generic solvers here were generated with {\sundials} in mind, our
intention is that they be usable in other applications as
general-purpose solvers.  This separation also allows for any generic
solver to be replaced by an improved version, with no necessity to
revise the {\cvodes} package elsewhere.
\index{generic linear solvers!use in {\cvodes}|)}

{\cvodes} also provides two preconditioner modules for use with any of
the Krylov iterative linear solvers.  The first one, {\cvbandpre}, is
intended to be used with {\nvecs} and provides banded difference-quotient
Jacobian-based preconditioner and solver routines.  The second
preconditioner module, {\cvbbdpre}, works in conjunction with {\nvecp}
and generates a preconditioner that is a block-diagonal matrix with
each block being a band matrix.

All state information used by {\cvodes} to solve a given problem is saved
in a structure, and a pointer to that structure is returned to the
user.  There is no global data in the {\cvodes} package, and so in this
respect it is reentrant. State information specific to the linear
solver is saved in a separate structure, a pointer to which resides in
the {\cvodes} memory structure. The reentrancy of {\cvodes} was motivated
by the anticipated multicomputer extension, but is also essential
in a uniprocessor setting where two or more problems are solved by
intermixed calls to the package from within a single user program.

