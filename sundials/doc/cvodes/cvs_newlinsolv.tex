%===================================================================================
\chapter{Providing Alternate Linear Solver Modules}\label{s:new_linsolv}
%===================================================================================
The central {\cvodes} module interfaces with the linear solver module to be
used by way of calls to four functions.  These are denoted here by 
\id{linit}, \id{lsetup}, \id{lsolve}, and \id{lfree}.
Briefly, their purposes are as follows:
%%
%%
\begin{itemize}
\item \id{linit}: initialize and allocate memory specific to the
  linear solver;
\item \id{lsetup}: preprocess and evaluate the Jacobian or preconditioner;
\item \id{lsolve}: solve the linear system;
\item \id{lfree}: free the linear solver memory.
\end{itemize}
%%
%%
A linear solver module must also provide a user-callable specification
function (like those described in \S\ref{sss:lin_solv_init}) which
will attach the above four functions to the main {\cvodes} memory block.
The {\cvodes} memory block is a structure defined in the header file
\id{cvodes\_impl.h}.
A pointer to such a structure is defined as the type \id{CVodeMem}. 
The four fields in a \id{CvodeMem} structure that must point to the
linear solver's functions are \id{cv\_linit}, \id{cv\_lsetup},
\id{cv\_lsolve}, and \id{cv\_lfree}, respectively.
Note that of the four interface functions, only the \id{lsolve}
function is required.  The \id{lfree} function must be provided only
if the solver specification function makes any memory allocation.
The linear solver specification function must also set the value of
the field \id{cv\_setupNonNull} in the {\cvodes} memory block --- to
\id{TRUE} if \id{lsetup} is used, or \id{FALSE} otherwise.

For consistency with the existing {\cvodes} linear solver modules, we
recommend that the return value of the specification function be 0 for
a successful return or a negative value if an error occurs (the
pointer to the main {\cvodes} memory block is \id{NULL}, an input is
illegal, the {\nvector} implementation is not compatible, a memory
allocation fails, etc.)

To facilitate data exchange between the four interface functions, the
field \id{cv\_lmem} in the {\cvodes} memory block can be used to
attach a linear solver-specific memory block.
That memory should be allocated in the linear solver specification function.

To be used during the backward integration with the {\cvodes} module,
a linear solver module must also provide an additional user-callable
specification function (like those described in
\S\ref{sss:lin_solv_b}) which will attach the four functions to the
{\cvodes} memory block for the backward integration. Note that this
block (of type \id{struct CVodeMemRec}) is not directly accessible to
the user, but rather is itself a field in the
{\cvodes} memory block.  The {\cvodes} memory block is a structure
defined in the header file \id{cvodes\_impl.h}.  A pointer to such a
structure is defined as the type \id{CVodeMem}.
%%
The specification function for backward integration should also return
a negative value if the adjoint {\cvodes} memory block is \id{NULL}.

An additional field (\id{ca\_lmemB}) in the {\cvodes} memory block
provides a hook-up for optionally attaching a linear solver-specific
memory block.

The four functions that interface between {\cvodes} and the linear solver module
necessarily have fixed call sequences.  Thus, a user wishing to implement another 
linear solver within the {\cvodes} package must adhere to this set of interfaces.
The following is a complete description of the argument list for each of
these functions.  Note that the argument list of each function includes a
pointer to the main {\cvodes} memory block, by which the function can access
various data related to the {\cvodes} solution.  The contents of this memory
block (of type \id{CVodeMem}) are given in the file \id{cvodes\_impl.h} 
(but not reproduced here, for the sake of space).

%------------------------------------------------------------------------------

\section{Initialization function}
The type definition of \ID{linit} is
\usfunction{linit}
{
  int (*linit)(CVodeMem cv\_mem);
}
{
  The purpose of \id{linit} is to complete linear solver-specific initializations,
  such as counters and statistics.        
}
{
  \begin{args}[cv\_mem]
  \item[cv\_mem]
    is the {\cvodes} memory pointer of type \id{CVodeMem}.
  \end{args}
}
{
  An \id{linit} function should return $0$ if it 
  has successfully initialized the {\cvodes} linear solver and 
  $-1$ otherwise. 
}
{}

%------------------------------------------------------------------------------

\section{Setup function} 
The type definition of \id{lsetup} is
\usfunction{lsetup}
{
   int (*lsetup)(&CVodeMem cv\_mem, int convfail, N\_Vector ypred, \\
                 &N\_Vector fpred, booleantype *jcurPtr,           \\
                 &N\_Vector vtemp1, N\_Vector vtemp2, N\_Vector vtemp3); 
}
{
  The job of \id{lsetup} is to prepare the linear solver for subsequent 
  calls to \id{lsolve}. It may recompute Jacobian-related data if it is 
  deemed necessary. 
}
{
   \begin{args}[convfail]
  
   \item[cv\_mem] 
     is the {\cvodes} memory pointer of type \id{CVodeMem}.
  
   \item[convfail]
     is an input flag used to indicate any problem that occurred during  
     the solution of the nonlinear equation on the current time step for which 
     the linear solver is being used. This flag can be used to help decide     
     whether the Jacobian data kept by a {\cvodes} linear solver needs to be
     updated or not. Its possible values are:
     \begin{itemize}
     \item \id{NO\_FAILURES}: this value is passed to \id{lsetup} if 
       either this is the first call for this step, or the local error test failed
       on the previous attempt at this step (but the Newton iteration converged).
     \item \id{FAIL\_BAD\_J}: this value is passed to \id{lsetup} if
       (a) the previous Newton corrector iteration did not converge and the linear
       solver's setup function indicated that its Jacobian-related data is not
       current, or                            
       (b) during the previous Newton corrector iteration, the linear solver's
       solve function failed in a recoverable manner and the linear solver's setup
       function indicated that its Jacobian-related data is not current.
     \item \id{FAIL\_OTHER}: this value is passed to \id{lsetup} if during the
       current internal step try, the previous Newton iteration failed to converge
       even though the linear solver was using current Jacobian-related data.
     \end{itemize}
  
   \item[ypred]
     is the predicted \id{y} vector for the current {\cvodes} internal step.
  
   \item[fpred]
     is the value of the right-hand side at \id{ypred}, i.e. $f(t_n, y_{pred})$.
  
   \item[jcurPtr]
     is a pointer to a boolean to be filled in by \id{lsetup}.  
     The function should set \id{*jcurPtr = TRUE} if its Jacobian 
     data is current after the call, and should set         
     \id{*jcurPtr = FALSE} if its Jacobian data is not current.   
     If \id{lsetup} calls for reevaluation of         
     Jacobian data (based on \id{convfail} and {\cvodes} state      
     data), it should return \id{*jcurPtr = TRUE} unconditionally;
     otherwise an infinite loop can result.                
    
   \item[vtemp1] 
   \item[vtemp2]
   \item[vtemp3] 
     are temporary variables of type \id{N\_Vector} provided for use by \id{lsetup}.      
  
   \end{args}
}
{
  An \id{lsetup} function should return $0$ if successful,            
  a positive value for a recoverable error, and a negative value  
  for an unrecoverable error.  
}
{}

%------------------------------------------------------------------------------

\section{Solve function}
The type definition of \id{lsolve} is
\usfunction{lsolve}
{
  int (*lsolve)(&CVodeMem cv\_mem, N\_Vector b, N\_Vector weight, \\
                &N\_Vector ycur, N\_Vector fcur);  
}
{
  The function \id{lsolve} must solve the linear equation $M x = b$, where         
  $M$ is some approximation to $I - \gamma J$, $J = (\dfdyI)(t_n, y_{cur})$
  (see Eq.(\ref{e:Newtonmat})), and the right-hand side vector $b$ is input. 
  Here $\gamma$ is available as \id{cv\_mem->cv\_gamma}.
}
{
  \begin{args}[cv\_mem]
  \item[cv\_mem]
    is the {\cvodes} memory pointer of type \id{CVodeMem}.
  \item[b]
    is the right-hand side vector $b$. The solution is to be    
    returned in the vector \id{b}.
  \item[weight]
    is a vector that contains the error weights.
    These are the $W_i$ of Eq.(\ref{e:errwt}).
  \item[ycur]
    is a vector that contains the solver's current approximation to $y(t_n)$.
  \item[fcur]
    is a vector that contains $f(t_n,y_{cur})$. 
  \end{args}
}
{
  An \id{lsolve} function should return a positive value    
  for a recoverable error and a negative value for an             
  unrecoverable error. Success is indicated by a $0$ return value.
}
{}

%------------------------------------------------------------------------------

\section{Memory deallocation function}
The type definition of \id{lfree} is
\usfunction{lfree}
{
  void (*lfree)(CVodeMem cv\_mem);
}
{
  The function \id{lfree} should free up any memory allocated by the linear
  solver.
}
{
  The argument \id{cv\_mem} is the {\cvodes} memory pointer of type \id{CVodeMem}.
}
{
  An \id{lfree} function has no return value.
}
{
  This function is called once a problem has been completed and the 
  linear solver is no longer needed.
}