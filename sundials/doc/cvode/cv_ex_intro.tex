%===============================================================================
\section{Introduction}\label{s:ex_intro}
%===============================================================================

This report is intended to serve as a companion document to the User
Documentation of {\cvode} \cite{cvode_ug}.  It provides details, with
listings, on the example programs supplied with the {\cvode} distribution
package.

The {\cvode} distribution contains examples of four types: serial
{\C} examples, parallel {\C} examples, and serial and parallel {\F}
examples.  
%%
With the exception of ''demo''-type example files, the names of all the examples 
distributed with {\sundials} are of the form \verb![slv][PbName]_[ls]_[prec]_[p]!, 
where
\begin{description}
\item [{[slv]}] identifies the solver (for {\cvode} examples this is \id{cv}, 
  while for {\fcvode} examples, this is \id{fcv});
\item [{[PbName]}] identifies the problem;
\item [{[ls]}] identifies the linear solver module used (for examples using
  functional iterations for the nonlinear system solver, \id{non} specifies
  that no linear solver was used);
\item [{[prec]}] indicates the {\cvode} preconditioner module used, \id{bp} for {\cvbandpre}
  or \id{bbd} for {\cvbbdpre} 
  (only if applicable, for examples using a Krylov linear solver);
\item [{[p]}] indicates an example using the parallel vector module {\nvecp}.
\end{description}

\vspace{0.2in}\noindent
The following lists summarize all examples distributed with {\cvode}.

\vspace{0.2in}\noindent
Supplied in the {\em srcdir}\id{/examples/cvode/serial} directory are the
following ten serial examples (using the {\nvecs} module):

\begin{itemize}
\item \id{cvRoberts\_dns}
  solves a chemical kinetics problem consisting of three rate equations.
  \newline
  This program solves the problem with the BDF method and Newton          
  iteration, with the {\cvdense} linear solver and a user-supplied    
  Jacobian routine.  It also uses the rootfinding feature of {\cvode}.
\item \id{cvRoberts\_dnsL}
  is the same as \id{cvRoberts\_dns} but uses the Lapack implementation of {\cvdense}.
\item \id{cvRoberts\_dns\_uw}
  is the same as \id{cvRoberts\_dns} but demonstrates the user-supplied error
  weight function feature of {\cvode}.
\item \id{cvAdvDiff\_bnd}
  solves the semi-discrete form of an advection-diffusion equation in 2-D. 
  \newline
  This program solves the problem with the BDF method and Newton          
  iteration, with the {\cvband} linear solver and a user-supplied     
  Jacobian routine.
\item \id{cvAdvDiff\_bndL}
  is the same as \id{cvAdvDiff\_bnd} but uses the Lapack implementation of {\cvband}.
\item \id{cvDiurnal\_kry}
  solves the semi-discrete form of a two-species diurnal kinetics
  advection-diffusion PDE system in 2-D.
  \newline
  The problem is solved with the BDF/GMRES method (i.e.    
  using the {\cvspgmr} linear solver) and the block-diagonal part of the  
  Newton matrix as a left preconditioner. A copy of the block-diagonal 
  part of the Jacobian is saved and conditionally reused within the    
  preconditioner setup routine.
\item \id{cvDiurnal\_kry\_bp}
  solves the same problem as \id{cvDiurnal\_kry}, with the BDF/GMRES method 
  and a banded preconditioner, generated by difference quotients, 
  using the module {\cvbandpre}.
  \newline
  The problem is solved twice: with preconditioning on the left,
  then on the right.
\item \id{cvDirectDemo\_ls}
  is a demonstration program for {\cvode} with direct linear solvers.
  \newline
  Two separate problems are solved using both the Adams and BDF linear
  multistep methods in combination with functional and Newton
  iterations. 
  \newline
  The first problem is the Van der Pol oscillator for which 
  the Newton iteration cases use the following types of Jacobian approximations:
  (1) dense, user-supplied, (2) dense, difference-quotient approximation, 
  (3) diagonal approximation. The second problem is a linear ODE with a
  banded lower triangular matrix derived from a 2-D advection PDE. In this
  case, the Newton iteration cases use the following types of Jacobian
  approximation: (1) banded, user-supplied, (2) banded, difference-quotient
  approximation, (3) diagonal approximation.
\item \id{cvKrylovDemo\_ls}
  solves the same problem as \id{cvDiurnal\_kry}, with the BDF method, but with
  three Krylov linear solvers: {\cvspgmr}, {\cvspbcg}, and {\cvsptfqmr}.
\item \id{cvKrylovDemo\_prec}
  is a demonstration program with the GMRES linear solver.
  \newline
  This program solves a stiff ODE system that arises from a system     
  of partial differential equations.  The PDE system is a six-species
  food web population model, with predator-prey interaction and diffusion 
  on the unit square in two dimensions.
  \newline
  The ODE system is solved using Newton iteration and the      
  {\cvspgmr} linear solver (scaled preconditioned GMRES).
  \newline
  The preconditioner matrix used is the product of two matrices:         
  (1) a matrix, only defined implicitly, based on a fixed number of     
  Gauss-Seidel iterations using the diffusion terms only; and               
  (2) a block-diagonal matrix based on the partial derivatives of the   
  interaction terms only, using block-grouping.                          
  \newline
  Four different runs are made for this problem.                        
  The product preconditoner is applied on the left and on the right.    
  In each case, both the modified and classical Gram-Schmidt options    
  are tested.
\end{itemize}

\vspace{0.2in}\noindent
Supplied in the {\em srcdir}\id{/examples/cvode/parallel} directory are
the following three parallel examples (using the {\nvecp} module):
\begin{itemize}
\item \id{cvAdvDiff\_non\_p}
  solves the semi-discrete form of a 1-D advection-diffusion equation.
  \newline
  This program solves the problem with the option for nonstiff systems,
  i.e. Adams method and functional iteration.
\item \id{cvDiurnal\_kry\_p}
  is a parallel implementation of \id{cvDiurnal\_kry}.
\item \id{cvDiurnal\_kry\_bbd\_p}
  solves the same problem as \id{cvDiurnal\_kry\_p}, with BDF and the GMRES linear
  solver, using a block-diagonal matrix with banded blocks as a preconditioner, 
  generated by difference quotients, using the module {\cvbbdpre}.
\end{itemize}

\vspace{0.2in}\noindent
Within the {\fcvode} module, in the directories 
{\em srcdir}\id{/examples/cvode/fcmix\_serial} and
{\em srcdir}\id{/examples/cvode/fcmix\_parallel}, are the following examples for
the {\F}-{\C} interface.  The first five of these are serial, while
the last three are parallel.
\begin{itemize}
\item \id{fcvRoberts\_dns} is a serial chemical kinetics example (\id{BDF}/{\cvdense})
  with rootfinding.
\item \id{fcvRoberts\_dnsL} is the same as \id{fcvRoberts\_dns} but uses the Lapack
  implementation of {\cvdense}.
\item \id{fcvAdvDiff\_bnd} is a serial advection-diffusion example (\id{BDF}/{\cvband}).
\item \id{fcvDiurnal\_kry} is a serial kinetics-transport example (\id{BDF}/{\cvspgmr}).
\item \id{fcvDiurnal\_kry\_bp} is the \id{fcvDiurnal\_kry} example with {\fcvbp}.
\item \id{fcvDiag\_non\_p} is a nonstiff parallel diagonal ODE example
  (\id{ADAMS}/\id{FUNCTIONAL}).
\item \id{fcvDiag\_kry\_p} is a stiff parallel diagonal ODE example (\id{BDF}/{\cvspgmr}).
\item \id{fcvDiag\_kry\_bbd\_p} is the same as the \id{fcvDiag\_kry\_p} example
       but using the {\fcvbbd} module.
\end{itemize}

\vspace{0.2in}\noindent
In the following sections, we give detailed descriptions of some (but
not all) of these examples.  We also give our output files for
each of these examples, but users should be cautioned that their
results may differ slightly from these.  Differences in solution
values may differ within the tolerances, and differences in cumulative
counters, such as numbers of steps or Newton iterations, may differ
from one machine environment to another by as much as 10\% to 20\%.

The final section of this report describes a set of tests done with the
parallel version of CVODE, using a problem based on the
\id{cvDiurnal\_kry}/\id{cvDiurnal\_kry\_p} example.

In the descriptions below, we make frequent references to the {\cvode}
User Document \cite{cvode_ug}.  All citations to specific sections
(e.g. \S\ref{s:types}) are references to parts of that User Document, unless
explicitly stated otherwise.

\vspace{0.2in}\noindent {\bf Note}.
The examples in the {\cvode} distribution are written in such a way as
to compile and run for any combination of configuration options during
the installation of {\sundials} (see Appendix \ref{c:install} in the User Guide).
As a consequence, they contain portions of code that will not be typically present in a
user program. For example, all {\C} example programs make use of the
variables \id{SUNDIALS\_EXTENDED\_PRECISION} and \id{SUNDIALS\_DOUBLE\_PRECISION}
to test if the solver libraries were built in extended or double precision,
and use the appropriate conversion specifiers in \id{printf} functions.
Similarly, the {\F} examples in {\fcvode} are automatically pre-processed to
generate source code that corresponds to the manner in which the {\cvode}
libraries were built (see \ref{s:ex_fortran} in this document for more details).
