%===================================================================================
\chapter{Mathematical Considerations}\label{s:math}
%===================================================================================

% What does CVODE do?
%--------------------
{\cvode} solves ODE initial value problems (IVPs) in real $N$-space, which we
write in the abstract form
\begin{equation}\label{e:ivp} 
  \dot{y} = f(t,y) \, ,\quad y(t_0) = y_0 \, ,
\end{equation}
where $y \in \mbox{\bf R}^N$.
Here we use $\dot{y}$ to denote $dy/dt$.  While we use $t$ to denote
the independent variable, and usually this is time, it certainly need
not be.  {\cvode} solves both stiff and nonstiff systems.  Roughly
speaking, stiffness is characterized by the presence of at least one
rapidly damped mode, whose time constant is small compared to the time
scale of the solution itself.

%------------------------
\section{IVP solution}\label{ss:ivp_sol}
%------------------------

% Linear multistep methods
%-------------------------
The methods used in {\cvode} are variable-order, variable-step multistep
methods, based on formulas of the form
\begin{equation}\label{e:lmm}
 \sum_{i = 0}^{K_1} \alpha_{n,i} y^{n-i} + 
     h_n \sum_{i = 0}^{K_2} \beta_{n,i} {\dot{y}}^{n-i} = 0 \, .
\end{equation}
Here the $y^n$ are computed approximations to $y(t_n)$, and
$h_n = t_n - t_{n-1}$ is the step size.  The user of {\cvode} must choose
appropriately one of two multistep methods.  For nonstiff problems,
{\cvode} includes the Adams-Moulton formulas \index{Adams method},
characterized by $K_1 = 1$
and $K_2 = q$ above, where the order $q$ varies between $1$ and $12$.
For stiff problems, {\cvode} includes the Backward Differentiation
Formulas (BDFs)  \index{BDF method} 
in so-called fixed-leading coefficient form, given by
$K_1 = q$ and $K_2 = 0$, with order $q$ varying between $1$ and $5$.
The coefficients are uniquely determined by the method type, its
order, the recent history of the step sizes, and the normalization
$\alpha_{n,0} = -1$.  See \cite{ByHi:75} and \cite{JaSD:80}.

% Nonlinear system
%-----------------
\index{nonlinear system!definition|(}
For either choice of formula, the nonlinear system
\begin{equation}\label{e:nonlinear}
  G(y^n) \equiv y^n - h_n \beta_{n,0} f(t_n,y^n) - a_n = 0 \, ,
\end{equation}
where $a_n\equiv\sum_{i>0}(\alpha_{n,i}y^{n-i}+h_n\beta_{n,i} {\dot{y}}^{n-i})$, 
must be solved (approximately) at each integration step.  For this, {\cvode}
offers the choice of either {\em functional iteration}, suitable only
for nonstiff systems, and various versions of {\em Newton iteration}.
Functional iteration, given by
\[ y^{n(m+1)} = h_n \beta_{n,0} f(t_n,y^{n(m)}) + a_n \, , \]
involves evaluations of $f$ only.  In contrast, Newton iteration requires
the solution of linear systems
\begin{equation}\label{e:Newton}
  M [y^{n(m+1)} - y^{n(m)}] = -G(y^{n(m)}) \, ,
\end{equation}
in which
\begin{equation}\label{e:Newtonmat} 
  M \approx I - \gamma J \, ,
  \quad J = \partial f / \partial y \, ,
  \quad \mbox{and} \quad
  \gamma = h_n \beta_{n,0} \, . 
\end{equation}
The initial guess for the iteration is a predicted value $y^{n(0)}$
computed explicitly from the available history data.
\index{nonlinear system!definition|)}

For the solution of the linear systems within the Newton corrections, 
{\cvode} provides several choices, including the option of an user-supplied
linear solver module. The linear solver modules distributed with {\sundials}
are organized in two families, a {\em direct} family comprising direct linear 
solvers for dense or banded matrices and a {\em spils} family comprising
scaled preconditioned iterative (Krylov) linear solvers. 
%%
In addition, {\cvode} also provides a linear solver module which only uses
a diagonal approximation of the Jacobian matrix. 
%%
The methods offered through these modules are as follows:
\begin{itemize}
\item dense direct solvers, using either an internal implementation or 
  a Blas/Lapack implementation (serial version only),
\item band direct solvers, using either an internal implementation or 
  a Blas/Lapack implementation (serial version only),
\item a diagonal approximate Jacobian solver,
\item {\spgmr}, a scaled preconditioned GMRES (Generalized Minimal Residual method)
  solver without restarts,
\item {\spbcg}, a scaled preconditioned Bi-CGStab (Bi-Conjugate Gradient Stable
  method) solver, or
\item {\sptfqmr}, a scaled preconditioned TFQMR (Transpose-Free Quasi-Minimal
  Residual method) solver.
\end{itemize}
For large stiff systems, where direct methods are not feasible, the
combination of a BDF integrator and any of the preconditioned Krylov
methods ({\spgmr}, {\spbcg}, or {\sptfqmr}) yields a powerful tool
because it combines established methods for stiff integration,
nonlinear iteration, and Krylov (linear) iteration with a
problem-specific treatment of the dominant source of stiffness, in the
form of the user-supplied preconditioner matrix \cite{BrHi:89}.
%%
Note that the direct linear solvers (dense and band) can only be 
used with serial vector representations.

% WRMS Norm
%----------
\index{weighted root-mean-square norm|(}
In the process of controlling errors at various levels, {\cvode} uses a
weighted root-mean-square norm, denoted $\|\cdot\|_{\mbox{\scriptsize WRMS}}$,
for all error-like quantities.  The multiplicative weights used are
based on the current solution and on the relative and absolute
tolerances input by the user, namely
\index{tolerances}
\begin{equation}\label{e:errwt}
 W_i = 1 / [\mbox{\sc rtol} \cdot |y_i| + \mbox{\sc atol}_i ] \, .
\end{equation}
Because $1/W_i$ represents a tolerance in the component $y_i$, a vector
whose norm is 1 is regarded as ``small.''  For brevity, we will
usually drop the subscript WRMS on norms in what follows.
\index{weighted root-mean-square norm|)}

% Newton iteration
%-----------------
\index{nonlinear system!Newton iteration|(}
In the cases of a direct solver (dense, band, or diagonal), the
iteration is a Modified Newton iteration, in that the iteration matrix
$M$ is fixed throughout the nonlinear iterations.  However, for any of
the Krylov methods, it is an Inexact Newton iteration, in which $M$ is
applied in a matrix-free manner, with matrix-vector products $Jv$
obtained by either difference quotients or a user-supplied routine.
The matrix $M$ (direct cases) or preconditioner matrix $P$ (Krylov
cases) is updated as infrequently as possible to balance the high costs
of matrix operations against other costs.  Specifically, this matrix
update occurs when:
\begin{itemize}
\item starting the problem,
\item more than 20 steps have been taken since the last update,
\item the value $\bar{\gamma}$ of $\gamma$ at the last update
satisfies $|\gamma/\bar{\gamma} - 1| > 0.3$,
\item a non-fatal convergence failure just occurred, or
\item an error test failure just occurred.
\end{itemize}
When forced by a convergence failure, an update of $M$ or $P$ may or
may not involve a reevaluation of $J$ (in $M$) or of Jacobian data
(in $P$), depending on whether Jacobian error was the likely cause of
the failure.  More generally, the decision is made to reevaluate $J$
(or instruct the user to reevaluate Jacobian data in $P$) when:
\begin{itemize}
\item starting the problem,
\item more than 50 steps have been taken since the last evaluation,
\item a convergence failure occurred with an outdated matrix, and
the value $\bar{\gamma}$ of $\gamma$ at the last update
satisfies $|\gamma/\bar{\gamma} - 1| < 0.2$, or
\item a convergence failure occurred that forced a step size reduction.
\end{itemize}
\index{nonlinear system!Newton iteration|)}

% Newton convergence test
%------------------------
\index{nonlinear system!Newton convergence test|(}
The stopping test for the Newton iteration is related to the
subsequent local error test, with the goal of keeping the nonlinear
iteration errors from interfering with local error control.  As
described below, the final computed value $y^{n(m)}$ will have to
satisfy a local error test $\|y^{n(m)} - y^{n(0)}\| \leq \epsilon$.
Letting $y^n$ denote the exact solution of (\ref{e:nonlinear}), we want
to ensure that the iteration error $y^n - y^{n(m)}$ is small relative
to $\epsilon$, specifically that it is less than $0.1 \epsilon$.
(The safety factor $0.1$ can be changed by the user.)  For this, we
also estimate the linear convergence rate constant $R$ as follows.
We initialize $R$ to 1, and reset $R = 1$ when $M$ or $P$ is updated.
After computing a correction $\delta_m = y^{n(m)}-y^{n(m-1)}$, we
update $R$ if $m > 1$ as
\begin{equation*}
  R \leftarrow \max\{0.3R , \|\delta_m\| / \|\delta_{m-1}\| \} \, . 
\end{equation*}
Now we use the estimate
\begin{equation*}
  \| y^n - y^{n(m)} \| \approx \| y^{n(m+1)} - y^{n(m)} \| 
  \approx R \| y^{n(m)} - y^{n(m-1)} \|  =  R \|\delta_m \| \, . 
\end{equation*}
Therefore the convergence (stopping) test is 
\begin{equation*}
  R \|\delta_m \| < 0.1 \epsilon \, .
\end{equation*}
We allow at most 3 iterations (but this limit can be changed by the
user).  We also declare the iteration diverged if any $\|\delta_m\| /
\|\delta_{m-1}\| > 2$ with $m > 1$. If convergence fails with $J$ or
$P$ current, we are forced to reduce the step size, and we replace
$h_n$ by $h_n/4$.  The integration is halted after a preset number
of convergence failures; the default value of this limit is 10, 
but this can be changed by the user.
\index{nonlinear system!Newton convergence test|)}

When a Krylov method is used to solve the linear system, its errors must also be
controlled, and this also involves the local error test constant.  The
linear iteration error in the solution vector $\delta_m$ is
approximated by the preconditioned residual vector.  Thus to ensure
(or attempt to ensure) that the linear iteration errors do not
interfere with the nonlinear error and local integration error
controls, we require that the norm of the preconditioned residual
be less than $0.05 \cdot (0.1 \epsilon)$.

% Jacobian DQ approximations
%---------------------------
With the direct dense and band methods, the Jacobian may be supplied
by a user routine, or approximated by difference quotients,
at the user's option.  In the latter case, we use the usual
approximation
\[ J_{ij} = [f_i(t,y+\sigma_j e_j) - f_i(t,y)]/\sigma_j \, . \]
The increments $\sigma_j$ are given by
\[ \sigma_j = \max\left\{\sqrt{U} \; |y_j| , \sigma_0 / W_j \right\} \, , \]
where $U$ is the unit roundoff, $\sigma_0$ is a dimensionless value,
and $W_j$ is the error weight defined in (\ref{e:errwt}).  In the dense
case, this scheme requires $N$ evaluations of $f$, one for each column
of $J$.  In the band case, the columns of $J$ are computed in groups,
by the Curtis-Powell-Reid algorithm, with the number of $f$ evaluations
equal to the bandwidth.

In the case of a Krylov method, preconditioning may be used on the left, on the
right, or both, with user-supplied routines for the preconditioning
setup and solve operations, and optionally also for the required
matrix-vector products $Jv$.  If a routine for $Jv$ is not supplied,
these products are computed as
\begin{equation}\label{jacobv}
Jv = [f(t,y+\sigma v) - f(t,y)]/\sigma \, . 
\end{equation}
The increment $\sigma$ is $1/\|v\|$, so that $\sigma v$ has norm 1.

% Error test
%-----------
\index{error control!step size selection|(}
A critical part of {\cvode} --- making it an ODE ``solver'' rather than
just an ODE method, is its control of local error.  At every step, the
local error is estimated and required to satisfy tolerance conditions,
and the step is redone with reduced step size whenever that error test
fails.  As with any linear multistep method, the local truncation
error LTE, at order $q$ and step size $h$, satisfies an asymptotic
relation
\[ \mbox{LTE} = C h^{q+1} y^{(q+1)} + O(h^{q+2}) \]
for some constant $C$, under mild assumptions on the step sizes.
A similar relation holds for the error in the predictor $y^{n(0)}$.
These are combined to get a relation
\[ \mbox{LTE} = C' [y^n - y^{n(0)}] + O(h^{q+2}) \, . \]
The local error test is simply $\|\mbox{LTE}\| \leq 1$.  Using the above,
it is performed on the predictor-corrector difference 
$\Delta_n \equiv y^{n(m)} - y^{n(0)}$ (with $y^{n(m)}$ the final
iterate computed), and takes the form
\[ \|\Delta_n\| \leq \epsilon \equiv 1/|C'| \, . \]
If this test passes, the step is considered successful.  If it fails,
the step is rejected and a new step size $h'$ is computed based on the
asymptotic behavior of the local error, namely by the equation
\[ (h'/h)^{q+1} \|\Delta_n\| = \epsilon/6 \, . \]
Here 1/6 is a safety factor.  A new attempt at the step is made,
and the error test repeated.  If it fails three times, the order $q$
is reset to 1 (if $q > 1$), or the step is restarted from scratch (if
$q = 1$).  The ratio $h'/h$ is limited above to 0.2 after two error test
failures, and limited below to 0.1 after three.  After seven failures,
{\cvode} returns to the user with a give-up message.
\index{error control!step size selection|)}

% Step/order control
%-------------------
\index{error control!order selection|(}
In addition to adjusting the step size to meet the local error test,
{\cvode} periodically adjusts the order, with the goal of maximizing the
step size.  The integration starts out at order 1 and varies the order
dynamically after that.  The basic idea is to pick the order $q$ for
which a polynomial of order $q$ best fits the discrete data involved
in the multistep method.  However, if either a convergence failure or
an error test failure occurred on the step just completed, no change
in step size or order is done.  At the current order $q$, selecting a
new step size is done exactly as when the error test fails, giving a
tentative step size ratio
\[ h'/h = (\epsilon / 6 \|\Delta_n\| )^{1/(q+1)} \equiv \eta_q \, . \]
We consider changing order only after taking $q+1$ steps at order $q$,
and then we consider only orders $q' = q - 1$ (if $q > 1$) or
$q' = q + 1$ (if $q < 5$).  The local truncation error at order $q'$
is estimated using the history data.  Then a tentative step size ratio
is computed on the basis that this error, LTE$(q')$, behaves
asymptotically as $h^{q'+1}$.  With safety factors of 1/6 and
1/10 respectively, these ratios are:
\[ h'/h = [1 / 6 \|\mbox{LTE}(q-1)\| ]^{1/q} \equiv \eta_{q-1} \]
and
\[ h'/h = [1 / 10 \|\mbox{LTE}(q+1)\| ]^{1/(q+2)} \equiv \eta_{q+1} \, . \]
The new order and step size are then set according to
\[ \eta = \max\{\eta_{q-1},\eta_q,\eta_{q+1}\} ~,~~ h' = \eta h \, , \]
with $q'$ set to the index achieving the above maximum.
However, if we find that $\eta < 1.5$, we do not bother with the
change.  Also, $h'/h$ is always limited to 10, except on the first
step, when it is limited to $10^4$.
\index{error control!order selection|)}

The various algorithmic features of {\cvode} described above, as
inherited from the solvers {\vode} and {\vodpk}, are documented in 
\cite{BBH:89,Byr:92,Hin:00}.  They are also summarized in
\cite{HBGLSSW:05}.

% Output modes
%-------------
\index{output mode}
Normally, {\cvode} takes steps until a user-defined output value 
$t = t_{\mbox{\scriptsize out}}$ is overtaken, and then it computes
$y(t_{\mbox{\scriptsize out}})$ by interpolation.  However, a
``one step'' mode option is available, where control returns to the
calling program after each step.  There are also options to force
{\cvode} not to integrate past a given stopping point 
$t = t_{\mbox{\scriptsize stop}}$.

%------------------------
\section{Preconditioning}\label{s:preconditioning}
%------------------------
\index{preconditioning!advice on|(}
When using a Newton method to solve the nonlinear system (\ref{e:nonlinear}),
{\cvode} makes repeated use of a linear solver to solve linear systems of the form
$M x = - r$, where $x$ is a correction vector and $r$ is a residual vector.
If this linear system solve is done with one of the scaled preconditioned iterative 
linear solvers, these solvers are rarely successful if used without preconditioning;
it is generally necessary to precondition the system in order to obtain acceptable efficiency.  
A system $A x = b$ can be preconditioned on the left, as $(P^{-1}A) x = P^{-1} b$;
on the right, as $(A P^{-1}) P x = b$; or on both sides, as
$(P_L^{-1} A P_R^{-1}) P_R x = P_L^{-1}b$.  The Krylov method is then
applied to a system with the matrix $P^{-1}A$, or $AP^{-1}$, or
$P_L^{-1} A P_R^{-1}$, instead of $A$.  In order to improve the
convergence of the Krylov iteration, the preconditioner matrix $P$, or
the product $P_L P_R$ in the last case, should in some sense
approximate the system matrix $A$.  Yet at the same time, in order to
be cost-effective, the matrix $P$, or matrices $P_L$ and $P_R$, should
be reasonably efficient to evaluate and solve.  Finding a good point
in this tradeoff between rapid convergence and low cost can be very
difficult.  Good choices are often problem-dependent (for example, see
\cite{BrHi:89} for an extensive study of preconditioners for
reaction-transport systems).

The {\cvode} solver allow for preconditioning either
side, or on both sides, although we know of no situation where
preconditioning on both sides is clearly superior to
preconditioning on one side only (with the product $P_L P_R$).
Moreover, for a given preconditioner matrix, the merits of left vs. right
preconditioning are unclear in general, and the user should experiment
with both choices.  Performance will differ because the inverse of the
left preconditioner is included in the linear system residual whose
norm is being tested in the Krylov algorithm.  As a rule, however, if
the preconditioner is the product of two matrices, we recommend that
preconditioning be done either on the left only or the right only,
rather than using one factor on each side. 

Typical preconditioners used with {\cvode} are based on approximations
to the system Jacobian, $J = \partial f / \partial y$.  Since the
Newton iteration matrix involved is $M = I - \gamma J$, any
approximation $\bar{J}$ to $J$ yields a matrix that is of potential
use as a preconditioner, namely $P = I - \gamma \bar{J}$.  Because the
Krylov iteration occurs within a Newton iteration and further also
within a time integration, and since each of these iterations has its
own test for convergence, the preconditioner may use a very crude
approximation, as long as it captures the dominant numerical
feature(s) of the system.  We have found that the combination of a
preconditioner with the Newton-Krylov iteration, using even a fairly
poor approximation to the Jacobian, can be surprisingly superior to
using the same matrix without Krylov acceleration (i.e., a modified
Newton iteration), as well as to using the Newton-Krylov method with
no preconditioning.  \index{preconditioning!advice on|)}

%------------------------
\section{BDF stability limit detection}\label{s:bdf_stab}
%------------------------

\index{Stability limit detection}
{\cvode} includes an algorithm, {\stald} (STAbility Limit Detection),
which provides protection against potentially unstable behavior of the 
BDF multistep integration methods is certain situations, as described below.

When the BDF option is selected, {\cvode} uses Backward Differentiation 
Formula methods of orders 1 to 5.  At order 1 or 2, the BDF
method is A-stable, meaning that for any complex constant $\lambda$ in
the open left half-plane, the method is unconditionally stable (for
any step size) for the standard scalar model problem $\dot{y} = \lambda y$.
For an ODE system, this means that, roughly speaking, as long as all
modes in the system are stable, the method is also stable for any
choice of step size, at least in the sense of a local linear stability
analysis.

At orders 3 to 5, the BDF methods are not A-stable, although they are
{\em stiffly stable}. In each case, in order for the method to be stable
at step size $h$ on the scalar model problem, the product $h\lambda$ must
lie in a {\em region of absolute stability}. 
That region excludes a portion of the left half-plane that is concentrated 
near the imaginary axis.  The size of that region of instability grows as the order
increases from 3 to 5.  What this means is that, when running BDF at
any of these orders, if an eigenvalue $\lambda$ of the system lies close
enough to the imaginary axis, the step sizes $h$ for which the method is
stable are limited (at least according to the linear stability theory)
to a set that prevents $h\lambda$ from leaving the stability region.
The meaning of {\em close enough} depends on the order.  
At order 3, the unstable region is much narrower than at order 5, 
so the potential for unstable behavior grows with order.

System eigenvalues that are likely to run into this instability are
ones that correspond to weakly damped oscillations.  A pure undamped
oscillation corresponds to an eigenvalue on the imaginary axis.
Problems with modes of that kind call for different considerations,
since the oscillation generally must be followed by the solver, and
this requires step sizes ($h \sim 1/\nu$, where $\nu$ is the frequency) 
that are stable for BDF anyway.  But for a weakly damped oscillatory mode,
the oscillation in the solution is eventually damped to the noise level, 
and at that time it is important that the solver not be restricted to step 
sizes on the order of $1/\nu$.  It is in this situation that the new option may
be of great value.

In terms of partial differential equations, the typical problems for
which the stability limit detection option is appropriate are
ODE systems resulting from semi-discretized PDEs (i.e., PDEs discretized 
in space) with advection and diffusion, but with advection dominating 
over diffusion.
Diffusion alone produces pure decay modes, while advection tends to
produce undamped oscillatory modes.  A mix of the two with advection
dominant will have weakly damped oscillatory modes.

The {\stald} algorithm attempts to detect, in a direct
manner, the presence of a stability region boundary that is limiting
the step sizes in the presence of a weakly damped oscillation \cite{Hin:92}.
The algorithm supplements (but differs greatly from) the existing
algorithms in {\cvode} for choosing step size and order based on
estimated local truncation errors.  It works directly
with history data that is readily available in {\cvode}.  If it concludes
that the step size is in fact stability-limited, it dictates a
reduction in the method order, regardless of the outcome of the
error-based algorithm.  The {\stald} algorithm has been tested in
combination with the {\vode} solver on linear advection-dominated
advection-diffusion problems \cite{Hin:95}, where it works well.  The
implementation in {\cvode} has been successfully tested on linear 
and nonlinear advection-diffusion problems, among others.

This stability limit detection option adds some overhead computational
cost to the {\cvode} solution.  (In timing tests, these overhead costs
have ranged from 2\% to 7\% of the total, depending on the size and
complexity of the problem, with lower relative costs for larger
problems.)  Therefore, it should be activated only when there is
reasonable expectation of modes in the user's system for which it is
appropriate.  In particular, if a {\cvode} solution with this option
turned off appears to take an inordinately large number of steps at
orders 3-5 for no apparent reason in terms of the solution time scale,
then there is a good chance that step sizes are being limited by
stability, and that turning on the option will improve the efficiency
of the solution. 


%------------------------
\section{Rootfinding}\label{ss:rootfinding}
%------------------------

\index{Rootfinding}
The {\cvode} solver has been augmented to include a rootfinding
feature.  This means that, while integrating the Initial Value Problem
(\ref{e:ivp}), {\cvode} can also find the roots of a set of user-defined
functions $g_i(t,y)$ that depend on $t$ and the solution vector 
$y = y(t)$.  The number of these root functions is arbitrary, and if
more than one $g_i$ is found to have a root in any given interval, the
various root locations are found and reported in the order that they
occur on the $t$ axis, in the direction of integration.

Generally, this rootfinding feature finds only roots of odd
multiplicity, corresponding to changes in sign of $g_i(t,y(t))$,
denoted $g_i(t)$ for short.  If a user root function has a root of
even multiplicity (no sign change), it will probably be missed by
{\cvode}.  If such a root is desired, the user should reformulate the
root function so that it changes sign at the desired root.

The basic scheme used is to check for sign changes of any $g_i(t)$ over
each time step taken, and then (when a sign change is found) to home
in on the root (or roots) with a modified secant method \cite{HeSh:80}.  
In addition, each time $g$ is computed, {\cvode} checks to see if 
$g_i(t) = 0$ exactly, and if so it reports this as a root.  However,
if an exact zero of any $g_i$ is found at a point $t$, {\cvode}
computes $g$ at $t + \delta$ for a small increment $\delta$, slightly
further in the direction of integration, and if any $g_i(t + \delta)=0$ 
also, {\cvode} stops and reports an error.  This way, each time
{\cvode} takes a time step, it is guaranteed that the values of all
$g_i$ are nonzero at some past value of $t$, beyond which a search for
roots is to be done.

At any given time in the course of the time-stepping, after suitable
checking and adjusting has been done, {\cvode} has an interval
$(t_{lo},t_{hi}]$ in which roots of the $g_i(t)$ are to be sought, such
that $t_{hi}$ is further ahead in the direction of integration, and
all $g_i(t_{lo}) \neq 0$.  The endpoint $t_{hi}$ is either $t_n$,
the end of the time step last taken, or the next requested output time
$t_{\mbox{\scriptsize out}}$ if this comes sooner.  The endpoint
$t_{lo}$ is either $t_{n-1}$, or the last output time
$t_{\mbox{\scriptsize out}}$ (if this occurred within the last
step), or the last root location (if a root was just located within
this step), possibly adjusted slightly toward $t_n$ if an exact zero
was found.  The algorithm checks $g$ at $t_{hi}$ for zeros and for
sign changes in $(t_{lo},t_{hi})$.  If no sign changes are found, then
either a root is reported (if some $g_i(t_{hi}) = 0$) or we proceed to
the next time interval (starting at $t_{hi}$).  If one or more sign
changes were found, then a loop is entered to locate the root to
within a rather tight tolerance, given by
\[ \tau = 100 * U * (|t_n| + |h|)~~~ (U = \mbox{unit roundoff}) ~. \]
Whenever sign changes are seen in two or more root functions, the one
deemed most likely to have its root occur first is the one with the
largest value of $|g_i(t_{hi})|/|g_i(t_{hi}) - g_i(t_{lo})|$,
corresponding to the closest to $t_{lo}$ of the secant method values.
At each pass through the loop, a new value $t_{mid}$ is set, strictly
within the search interval, and the values of $g_i(t_{mid})$ are
checked.  Then either $t_{lo}$ or $t_{hi}$ is reset to $t_{mid}$
according to which subinterval is found to have the sign change.  If
there is none in $(t_{lo},t_{mid})$ but some $g_i(t_{mid}) = 0$, then
that root is reported.  The loop continues until 
$|t_{hi}-t_{lo}| < \tau$, and then the reported root location is
$t_{hi}$.

In the loop to locate the root of $g_i(t)$, the formula for $t_{mid}$
is
\[ t_{mid} = t_{hi} - (t_{hi} - t_{lo})
             g_i(t_{hi}) / [g_i(t_{hi}) - \alpha g_i(t_{lo})] ~, \] 
where $\alpha$ a weight parameter.  On the first two passes through
the loop, $\alpha$ is set to $1$, making $t_{mid}$ the secant method
value.  Thereafter, $\alpha$ is reset according to the side of the
subinterval (low vs high, i.e. toward $t_{lo}$ vs toward $t_{hi}$)
in which the sign change was found in the previous two passes.  If the
two sides were opposite, $\alpha$ is set to 1.  If the two sides were
the same, $\alpha$ is halved (if on the low side) or doubled (if on
the high side).  The value of $t_{mid}$ is closer to $t_{lo}$ when
$\alpha < 1$ and closer to $t_{hi}$ when $\alpha > 1$.  If the above
value of $t_{mid}$ is within $\tau/2$ of $t_{lo}$ or $t_{hi}$, it is
adjusted inward, such that its fractional distance from the endpoint
(relative to the interval size) is between .1 and .5 (.5 being the
midpoint), and the actual distance from the endpoint is at least
$\tau/2$.
