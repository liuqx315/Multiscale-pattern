%===================================================================================
\chapter{Introduction}\label{s:intro}
%===================================================================================

{\cvode} is part of a software family called {\sundials}: 
SUite of Nonlinear and DIfferential/ALgebraic equation Solvers~\cite{HBGLSSW:05}.  
This suite consists of {\cvode}, {\arkode}, {\kinsol}, and {\ida}, and variants
of these with sensitivity analysis capabilities.
%
%---------------------------------
\section{Historical Background}\label{ss:history}
%---------------------------------

\index{CVODE@{\cvode}!relationship to {\vode}, {\vodpk}|(}
{\F} solvers for ODE initial value problems are widespread and heavily used. 
Two solvers that have been written at LLNL in the past are {\vode}~\cite{BBH:89} 
and {\vodpk}~\cite{Byr:92}.
{\vode}\index{VODE@{\vode}} is a general purpose solver that includes methods for
stiff and nonstiff systems, and in the stiff case uses direct methods (full or
banded) for the solution of the linear systems that arise at each implicit
step. Externally, {\vode} is very similar to the well known solver
{\lsode}\index{LSODE@{\lsode}}~\cite{RaHi:94}. {\vodpk}\index{VODPK@{\vodpk}}
is a variant of {\vode} that uses a preconditioned Krylov (iterative)
method, namely GMRES, for the solution of the linear systems. {\vodpk}
is a powerful tool for large stiff systems because it combines
established methods for stiff integration, nonlinear iteration, and
Krylov (linear) iteration with a problem-specific treatment of the
dominant source of stiffness, in the form of the user-supplied
preconditioner matrix~\cite{BrHi:89}.  The capabilities of both
{\vode} and {\vodpk} have been combined in the {\C}-language package
{\cvode}\index{CVODE@{\cvode}}~\cite{CoHi:96}.

At present, {\cvode} contains three Krylov methods that can be used
in conjuction with Newton iteration:
the GMRES (Generalized Minimal RESidual)~\cite{SaSc:86},
Bi-CGStab (Bi-Conjugate Gradient Stabilized)~\cite{Van:92}, and
TFQMR (Transpose-Free Quasi-Minimal Residual) linear iterative 
methods~\cite{Fre:93}.  As Krylov methods, these require almost no 
matrix storage for solving the Newton equations as compared to direct 
methods. However, the algorithms allow for a user-supplied preconditioner
matrix, and for most problems preconditioning is essential for an
efficient solution.
For very large stiff ODE systems, the Krylov methods are preferable over
direct linear solver methods, and are often the only feasible choice.
Among the three Krylov methods in {\cvode}, we recommend GMRES as the
best overall choice.  However, users are encouraged to compare all
three, especially if encountering convergence failures with GMRES.
Bi-CGFStab and TFQMR have an advantage in storage requirements, in
that the number of workspace vectors they require is fixed, while that
number for GMRES depends on the desired Krylov subspace size.

In the process of translating the {\vode} and {\vodpk} algorithms into
{\C}, the overall {\cvode} organization has been changed considerably.
One key feature of the {\cvode} organization is that the linear system
solvers comprise a layer of code modules that is separated from the
integration algorithm, allowing for easy modification and expansion of
the linear solver array.  A second key feature is a separate module
devoted to vector operations; this facilitated the extension to
multiprosessor environments with minimal impacts on the rest of the
solver, resulting in {\pvode}\index{PVODE@{\pvode}}~\cite{ByHi:99},
the parallel variant of {\cvode}.  \index{CVODE@{\cvode}!relationship
to {\vode}, {\vodpk}|)}

\index{CVODE@{\cvode}!relationship to {\cvode}, {\pvode}|(} Recently,
the functionality of {\cvode} and {\pvode} has been combined into one
single code, simply called {\cvode}. Development of the new version of
{\cvode} was concurrent with a redesign of the vector operations
module across the {\sundials} suite. The key feature of the new
{\nvector} module is that it is written in terms of abstract vector
operations with the actual vector kernels attached by a particular
implementation (such as serial or parallel) of {\nvector}. This allows
writing the {\sundials} solvers in a manner independent of the actual
{\nvector} implementation (which can be user-supplied), as well as
allowing more than one {\nvector} module linked into an executable file.
\index{CVODE@{\cvode}!relationship to {\cvode}, {\pvode}|)}

\index{CVODE@{\cvode}!motivation for writing in C|(}
There are several motivations for choosing the {\C} language for {\cvode}.
First, a general movement away from {\F} and toward {\C} in scientific
computing is apparent.  Second, the pointer, structure, and dynamic
memory allocation features in C are extremely useful in software of
this complexity, with the great variety of method options offered.
Finally, we prefer {\C} over {\CPP} for {\cvode} because of the wider
availability of {\C} compilers, the potentially greater efficiency of {\C},
and the greater ease of interfacing the solver to applications written
in extended {\F}.
\index{CVODE@{\cvode}!motivation for writing in C|)}

\section{Changes from previous versions}

\subsection*{Changes in v2.8.0}

Two major additions were made to the linear system solvers that are
available for use with the {\cvode} solver.  First, in the serial case,
an interface to the sparse direct solver KLU was added.
Second, an interface to SuperLU\_MT, the multi-threaded version of
SuperLU, was added as a thread-parallel sparse direct solver option,
to be used with the serial version of the NVECTOR module.
As part of these additions, a sparse matrix (CSC format) structure 
was added to {\cvode}.

Otherwise, only relatively minor modifications were made to the
{\cvode} solver:

In \id{cvRootfind}, a minor bug was corrected, where the input
array \id{rootdir} was ignored, and a line was added to break out of
root-search loop if the initial interval size is below the tolerance
\id{ttol}.

In \id{CVLapackBand}, the line \id{smu = MIN(N-1,mu+ml)} was changed to
\id{smu = mu + ml} to correct an illegal input error for \id{DGBTRF/DGBTRS}.

In order to eliminate or minimize the differences between the sources
for private functions in {\cvode} and {\cvodes}, the names of 48
private functions were changed from \id{CV**} to \id{cv**}, and a few
other names were also changed.

In the FCVODE optional input routines \id{FCVSETIIN} and \id{FCVSETRIN},
the optional fourth argument \id{key\_length} was removed, with
hardcoded key string lengths passed to all \id{strncmp} tests.

In all FCVODE examples, integer declarations were revised so that
those which must match a C type \id{long int} are declared \id{INTEGER*8},
and a comment was added about the type match.  All other integer
declarations are just \id{INTEGER}.  Corresponding minor corrections were
made to the user guide.

Two new {\nvector} modules have been added for parallel computing
environments --- one for openMP, denoted \id{NVECTOR\_OPENMP},
and one for Pthreads, denoted \id{NVECTOR\_PTHREADS}.

With this version of {\sundials}, support and documentation of the
Autotools mode of installation is being dropped, in favor of the
CMake mode, which is considered more widely portable.

\subsection*{Changes in v2.7.0}

One significant design change was made with this release: The problem
size and its relatives, bandwidth parameters, related internal indices,
pivot arrays, and the optional output \id{lsflag} have all been
changed from type \id{int} to type \id{long int}, except for the
problem size and bandwidths in user calls to routines specifying
BLAS/LAPACK routines for the dense/band linear solvers.  The function
\id{NewIntArray} is replaced by a pair \id{NewIntArray}/\id{NewLintArray},
for \id{int} and \id{long int} arrays, respectively.

A large number of minor errors have been fixed.  Among these are the following:
In \id{CVSetTqBDF}, the logic was changed to avoid a divide by zero.
After the solver memory is created, it is set to zero before being filled.
In each linear solver interface function, the linear solver memory is
freed on an error return, and the \id{**Free} function now includes a
line setting to NULL the main memory pointer to the linear solver memory.
In the rootfinding functions \id{CVRcheck1}/\id{CVRcheck2}, when an exact
zero is found, the array \id{glo} of $g$ values at the left endpoint is
adjusted, instead of shifting the $t$ location \id{tlo} slightly.
In the installation files, we modified the treatment of the macro
SUNDIALS\_USE\_GENERIC\_MATH, so that the parameter GENERIC\_MATH\_LIB is
either defined (with no value) or not defined.

\subsection*{Changes in v2.6.0}

Two new features were added in this release: (a) a new linear solver module,
based on Blas and Lapack for both dense and banded matrices, and (b) an option
to specify which direction of zero-crossing is to be monitored while performing
rootfinding. 

The user interface has been further refined. Some of the API changes involve:
(a) a reorganization of all linear solver modules into two families (besides 
the existing family of scaled preconditioned iterative linear solvers,
the direct solvers, including the new Lapack-based ones, were also organized 
into a {\em direct} family); (b) maintaining a single pointer to user data,
optionally specified through a \id{Set}-type function; and (c) a general 
streamlining of the preconditioner modules distributed with the solver.

\subsection*{Changes in v2.5.0}

The main changes in this release involve a rearrangement of the entire 
{\sundials} source tree (see \S\ref{ss:sun_org}). At the user interface 
level, the main impact is in the mechanism of including {\sundials} header
files which must now include the relative path (e.g. \id{\#include <cvode/cvode.h>}).
Additional changes were made to the build system: all exported header files are
now installed in separate subdirectories of the instaltion {\em include} directory.

The functions in the generic dense linear solver (\id{sundials\_dense} and
\id{sundials\_smalldense}) were modified to work for rectangular $m \times n$
matrices ($m \le n$), while the factorization and solution functions were
renamed to \id{DenseGETRF}/\id{denGETRF} and \id{DenseGETRS}/\id{denGETRS}, 
respectively.
The factorization and solution functions in the generic band linear solver were 
renamed \id{BandGBTRF} and \id{BandGBTRS}, respectively.

\subsection*{Changes in v2.4.0}

{\cvspbcg} and {\cvsptfqmr} modules have been added to interface with the
Scaled Preconditioned Bi-CGstab ({\spbcg}) and Scaled Preconditioned
Transpose-Free Quasi-Minimal Residual ({\sptfqmr}) linear solver modules,
respectively (for details see Chapter \ref{s:simulation}). Corresponding
additions were made to the {\F} interface module {\fcvode}.
At the same time, function type names for Scaled Preconditioned Iterative
Linear Solvers were added for the user-supplied Jacobian-times-vector and
preconditioner setup and solve functions.

The deallocation functions now take as arguments the address of the respective 
memory block pointer.

To reduce the possibility of conflicts, the names of all header files have
been changed by adding unique prefixes (\id{cvode\_} and \id{sundials\_}).
When using the default installation procedure, the header files are exported
under various subdirectories of the target \id{include} directory. For more
details see Appendix \ref{c:install}.

\subsection*{Changes in v2.3.0}

The user interface has been further refined. Several functions used
for setting optional inputs were combined into a single one.  An optional
user-supplied routine for setting the error weight vector was added.
Additionally, to resolve potential variable scope issues, all
SUNDIALS solvers release user data right after its use. The build
systems has been further improved to make it more robust.

\subsection*{Changes in v2.2.1}

The changes in this minor {\sundials} release affect only the build system.

\subsection*{Changes in v2.2.0}

The major changes from the previous version involve a redesign of the
user interface across the entire {\sundials} suite. We have eliminated the
mechanism of providing optional inputs and extracting optional statistics 
from the solver through the \id{iopt} and \id{ropt} arrays. Instead,
{\cvode} now provides a set of routines (with prefix \id{CVodeSet})
to change the default values for various quantities controlling the
solver and a set of extraction routines (with prefix \id{CVodeGet})
to extract statistics after return from the main solver routine.
Similarly, each linear solver module provides its own set of {\id{Set}-}
and {\id{Get}-type} routines. For more details see \S\ref{ss:optional_input}
and \S\ref{ss:optional_output}.

Additionally, the interfaces to several user-supplied routines
(such as those providing Jacobians and preconditioner information) 
were simplified by reducing the number
of arguments. The same information that was previously accessible
through such arguments can now be obtained through {\id{Get}-type}
functions.

The rootfinding feature was added, whereby the roots of a set of given
functions may be computed during the integration of the ODE system.

Installation of {\cvode} (and all of {\sundials}) has been completely 
redesigned and is now based on configure scripts.


\section{Reading this User Guide}\label{ss:reading}

This user guide is a combination of general usage instructions and
specific example programs.  We expect that some readers will want to
concentrate on the general instructions, while others will refer
mostly to the examples, and the organization is intended to
accommodate both styles.

There are different possible levels of usage of {\cvode}. The most
casual user, with a small IVP problem only, can get by with reading
\S\ref{ss:ivp_sol}, then Chapter \ref{s:simulation} through
\S\ref{sss:cvode} only, and looking at examples in~\cite{cvode_ex}.
In a different direction, a more expert user with an IVP problem may want
to (a) use a package preconditioner (\S\ref{ss:preconds}), (b) supply
his/her own Jacobian or preconditioner routines (\S\ref{ss:user_fct_sim}),
(c) do multiple runs of problems of the same size (\S\ref{sss:cvreinit}), 
(d) supply a new {\nvector} module (Chapter \ref{s:nvector}), or even 
(e) supply a different linear solver module
(\S\ref{ss:cvode_org} and Chapter \ref{s:new_linsolv}).

The structure of this document is as follows:
\begin{itemize}
\item
  In Chapter \ref{s:math}, we give short descriptions of the numerical
  methods implemented by {\cvode} for the solution of initial value
  problems for systems of ODEs, and continue with short descriptions of
  preconditioning (\S\ref{s:preconditioning}), stability limit detection
  (\S\ref{s:bdf_stab}), and rootfinding (\S\ref{ss:rootfinding}).
\item
  The following chapter describes the structure of the {\sundials} suite
  of solvers (\S\ref{ss:sun_org}) and the software organization of the {\cvode}
  solver (\S\ref{ss:cvode_org}). 
\item
  Chapter \ref{s:simulation} is the main usage document for {\cvode} for
  {\C} applications.  It includes a complete description of the user interface
  for the integration of ODE initial value problems.
\item
  In Chapter \ref{s:fcmix}, we describe {\fcvode}, an interface module
  for the use of {\cvode} with {\F} applications.
\item
  Chapter \ref{s:nvector} gives a brief overview of the generic
  {\nvector} module shared among the various components of
  {\sundials}, and details on the two {\nvector} implementations
  provided with {\sundials}: a serial implementation
  (\S\ref{ss:nvec_ser}) and a parallel implementation based on
  MPI\index{MPI} (\S\ref{ss:nvec_par}).
\item
  Chapter \ref{s:new_linsolv} describes the interfaces to the linear
  solver modules, so that a user can provide his/her own such module.
\item
  Chapter \ref{s:gen_linsolv} describes in detail the generic linear
  solvers shared by all {\sundials} solvers.
\item
  Finally, in the appendices, we provide detailed instructions for the installation
  of {\cvode}, within the structure of {\sundials} (Appendix \ref{c:install}), as well
  as a list of all the constants used for input to and output from {\cvode} functions
  (Appendix \ref{c:constants}).
\end{itemize}

Finally, the reader should be aware of the following notational conventions
in this user guide:  program listings and identifiers (such as \id{CVodeInit}) 
within textual explanations appear in typewriter type style; 
fields in {\C} structures (such as {\em content}) appear in italics;
and packages or modules, such as {\cvdense}, are written in all capitals. 
Usage and installation instructions that constitute important warnings
are marked with a triangular symbol {\warn} in the margin.

\paragraph{Acknowledgments.}
We wish to acknowledge the contributions to previous versions of the
{\cvode} and {\pvode} codes and their user guides by Scott D. Cohen~\cite{CoHi:94}
and George D. Byrne~\cite{ByHi:98}.

