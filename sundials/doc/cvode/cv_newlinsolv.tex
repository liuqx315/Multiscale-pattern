%===================================================================================
\chapter{Providing Alternate Linear Solver Modules}\label{s:new_linsolv}
%===================================================================================
The central {\cvode} module interfaces with the linear solver module to be
used by way of calls to four functions.  These are denoted here by 
\id{linit}, \id{lsetup}, \id{lsolve}, and \id{lfree}.
Briefly, their purposes are as follows:
%%
%%
\begin{itemize}
\item \id{linit}: initialize memory specific to the linear solver;
\item \id{lsetup}: evaluate and preprocess the Jacobian or preconditioner;
\item \id{lsolve}: solve the linear system;
\item \id{lfree}: free the linear solver memory.
\end{itemize}
%%
%%
A linear solver module must also provide a user-callable {\bf specification function}
(like those described in \S\ref{sss:lin_solv_init}) which will
attach the above four functions to the main {\cvode} memory block.
The {\cvode} memory block is a structure defined in the header file
\id{cvode\_impl.h}.  A pointer to such a structure is defined as the
type \id{CVodeMem}.  The four fields in a \id{CvodeMem} structure that
must point to the linear solver's functions are \id{cv\_linit},
\id{cv\_lsetup}, \id{cv\_lsolve}, and \id{cv\_lfree}, respectively.
Note that of the four interface functions, only the \id{lsolve} function
is required.  The \id{lfree} function must be provided only if the solver
specification function makes any memory allocation.   For any of the functions
that are {\it not} provided, the corresponding field should be set to \id{NULL}.
The linear solver specification function must also set the value of
the field \id{cv\_setupNonNull} in the {\cvode} memory block --- to
\id{TRUE} if \id{lsetup} is used, or \id{FALSE} otherwise.

Typically, the linear solver will require a block of memory specific
to the solver, and a principal function of the specification function
is to allocate that memory block, and initialize it.  Then the field
\id{cv\_lmem} in the {\cvode} memory block is available to attach a
pointer to that linear solver memory.  This block can then be used
to facilitate the exchange of data between the four interface
functions.

If the linear solver involves adjustable parameters, the specification
function should set the default values of those.  User-callable
functions may be defined that could, optionally, override the default
parameter values.

We encourage the use of performance counters in connection with the various
operations involved with the linear solver.  Such counters would be
members of the linear solver memory block, would be initialized in the
\id{linit} function, and would be incremented by the \id{lsetup} and
\id{lsolve} functions.  Then user-callable functions would be needed to
obtain the values of these counters.

For consistency with the existing {\cvode} linear solver modules, we
recommend that the return value of the specification function be 0 for
a successful return, and a negative value if an error occurs.
Possible error conditions include: the pointer to the main {\cvode}
memory block is \id{NULL}, an input is illegal, the {\nvector}
implementation is not compatible, or a memory allocation fails.

\vspace{0.1in}
These four functions, which interface between {\cvode} and the linear solver module,
necessarily have fixed call sequences.  Thus, a user wishing to implement another 
linear solver within the {\cvode} package must adhere to this set of interfaces.
The following is a complete description of the call list for each of
these functions.  Note that the call list of each function includes a
pointer to the main {\cvode} memory block, by which the function can access
various data related to the {\cvode} solution.  The contents of this memory
block are given in the file \id{cvode\_impl.h} (but not reproduced here, for
the sake of space).

%------------------------------------------------------------------------------

\section{Initialization function}
The type definition of \ID{linit} is
\usfunction{linit}
{
  int (*linit)(CVodeMem cv\_mem);
}
{
  The purpose of \id{linit} is to complete initializations for
  the specific linear solver, such as counters and statistics.
  It should also set pointers to data blocks that will later be
  passed to functions associated with the linear solver.
  The \id{linit} function is called once only, at the start of
  the problem, during the first call to \id{CVode}.
}
{
  \begin{args}[cv\_mem]
  \item[cv\_mem]
    is the {\cvode} memory pointer of type \id{CVodeMem}.
  \end{args}
}
{
  An \id{linit} function should return 0 if it 
  has successfully initialized the {\cvode} linear solver, and 
  a negative value otherwise. 
}
{}

%------------------------------------------------------------------------------

\section{Setup function} 
The type definition of \id{lsetup} is
\usfunction{lsetup}
{
   int (*lsetup)(&CVodeMem cv\_mem, int convfail, N\_Vector ypred, \\
                 &N\_Vector fpred, booleantype *jcurPtr,           \\
                 &N\_Vector vtemp1, N\_Vector vtemp2, N\_Vector vtemp3); 
}
{
  The job of \id{lsetup} is to prepare the linear solver for subsequent 
  calls to \id{lsolve}, in the solution of systems $M x = b$, where         
  $M$ is some approximation to the Newton matrix,
  $I - \gamma ~ \partial f / \partial y$.  (See Eq.(\ref{e:Newtonmat})).
  Here $\gamma$ is available as \id{cv\_mem->cv\_gamma}.

  The \id{lsetup} function may call a user-supplied function, or a function
  within the linear solver module, to compute needed data related to the
  Jacobian matrix $\partial f / \partial y$.  Alterntively, it may
  choose to retrieve and use stored values of this data.

  In either case, \id{lsetup} may also preprocess that
  data as needed for \id{lsolve}, which may involve calling a generic
  function (such as for LU factorization).  This data may be intended
  either for direct use (in a direct linear solver) or for use in a
  preconditioner (in a preconditioned iterative linear solver).

  The \id{lsetup} function is not called at every time step, but only
  as frequently as the solver determines that it is appropriate to perform
  the setup task.  In this way, Jacobian-related data generated by \id{lsetup}
  is expected to be used over a number of time steps.
}
{
   \begin{args}[convfail]
  
   \item[cv\_mem] 
     is the {\cvode} memory pointer of type \id{CVodeMem}.
  
   \item[convfail]
     is an input flag used to indicate any problem that occurred during  
     the solution of the nonlinear equation on the current time step for which 
     the linear solver is being used. This flag can be used to help decide     
     whether the Jacobian data kept by a {\cvode} linear solver needs to be updated 
     or not. Its possible values are:
     \begin{itemize}
     \item \id{CV\_NO\_FAILURES}: this value is passed to \id{lsetup} if 
       either this is the first call for this step, or the local error test failed
       on the previous attempt at this step (but the Newton iteration converged).
     \item \id{CV\_FAIL\_BAD\_J}: this value is passed to \id{lsetup} if
       (a) the previous Newton corrector iteration did not converge and the linear
       solver's setup function indicated that its Jacobian-related data is not
       current, or                            
       (b) during the previous Newton corrector iteration, the linear solver's
       solve function failed in a recoverable manner and the linear solver's setup
       function indicated that its Jacobian-related data is not current.
     \item \id{CV\_FAIL\_OTHER}: this value is passed to \id{lsetup} if 
       during the current internal step try, the previous Newton iteration
       failed to converge even though the linear solver was using current
       Jacobian-related data.
     \end{itemize}
  
   \item[ypred]
     is the predicted \id{y} vector for the current {\cvode} internal step.
  
   \item[fpred]
     is the value of the right-hand side at \id{ypred}, $f(t_n,$ \id{ypred}$)$.
  
   \item[jcurPtr]
     is a pointer to a boolean to be filled in by \id{lsetup}.  
     The function should set \id{*jcurPtr = TRUE} if its Jacobian 
     data is current after the call, and should set         
     \id{*jcurPtr = FALSE} if its Jacobian data is not current.   
     If \id{lsetup} calls for re-evaluation of         
     Jacobian data (based on \id{convfail} and {\cvode} state      
     data), it should return \id{*jcurPtr = TRUE} unconditionally;
     otherwise an infinite loop can result.                
    
   \item[vtemp1] 
   \item[vtemp2]
   \item[vtemp3] 
     are temporary variables of type \id{N\_Vector} provided for use by \id{lsetup}.
   \end{args}
}
{
  The \id{lsetup} function should return 0 if successful,            
  a positive value for a recoverable error, and a negative value  
  for an unrecoverable error.  On a recoverable error return, the solver
  will attempt to recover by reducing the step size.
}
{}

%------------------------------------------------------------------------------

\section{Solve function}
The type definition of \id{lsolve} is
\usfunction{lsolve}
{
  int (*lsolve)(&CVodeMem cv\_mem, N\_Vector b, N\_Vector weight, \\
                &N\_Vector ycur, N\_Vector fcur);  
}
{
  The function \id{lsolve} must solve the linear system $M x = b$, where         
  $M$ is some approximation to the Newton matrix $I - \gamma J$, where
  $J = (\dfdyI)(t_n, y_{cur})$ (see Eq.(\ref{e:Newtonmat})),
  and the right-hand side vector $b$ is input.
  Here $\gamma$ is available as \id{cv\_mem->cv\_gamma}.

  \id{lsolve} is called once per Newton iteration, hence possibly
  several times per time step.

  If there is an \id{lsetup} function, this \id{lsolve} function should
  make use of any Jacobian data that was computed and preprocessed
  by \id{lsetup}, either for direct use, or for use in a preconditioner.
}
{
  \begin{args}[cv\_mem]
  \item[cv\_mem]
    is the {\cvode} memory pointer of type \id{CVodeMem}.
  \item[b]
    is the right-hand side vector $b$. The solution is to be    
    returned in the vector \id{b}.
  \item[weight]
    is a vector that contains the error weights.
    These are the $W_i$ of Eq.(\ref{e:errwt}).
    This weight vector is included here to enable the computation of
    weighted norms needed to test for the convergence of iterative methods
    (if any) within the linear solver.
   \item[ycur]
    is a vector that contains the solver's current approximation to $y(t_n)$.
  \item[fcur]
    is a vector that contains the current right-hand side, $f(t_n,$ \id{ycur}$)$. 
  \end{args}
}
{
  The \id{lsolve} function should return a positive value    
  for a recoverable error and a negative value for an             
  unrecoverable error. Success is indicated by a 0 return value.
  On a recoverable error return, the solver will attempt to recover, such
  as by calling the \id{lsetup} function with current arguments.
}
{}

%------------------------------------------------------------------------------

\section{Memory deallocation function}
The type definition of \id{lfree} is
\usfunction{lfree}
{
  void (*lfree)(CVodeMem cv\_mem);
}
{
  The function \id{lfree} should free up any memory allocated by the linear
  solver.
}
{
  The argument \id{cv\_mem} is the {\cvode} memory pointer of type \id{CVodeMem}.
}
{
  The \id{lfree} function has no return value.
}
{
  This function is called once a problem has been completed and the 
  linear solver is no longer needed.
}