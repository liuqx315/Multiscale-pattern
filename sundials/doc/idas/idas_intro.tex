%===================================================================================
\chapter{Introduction}\label{s:intro}
%===================================================================================

{\idas} is part of a software family called {\sundials}: 
SUite of Nonlinear and DIfferential/ALgebraic equation Solvers~\cite{HBGLSSW:05}. 
This suite consists of {\cvode}, {\arkode}, {\kinsol}, and {\ida}, and variants
of these with sensitivity analysis capabilities, {\cvodes} and {\idas}.

{\idas} is a general purpose solver for the initial value problem (IVP) for
systems of differential-algebraic equations (DAEs).  The name IDAS
stands for Implicit Differential-Algebraic solver with Sensitivity capabilities.
{\idas} is an extension of the {\ida} solver within {\sundials}, itself 
based on {\daspk}~\cite{BHP:94,BHP:98}; however, like all {\sundials} solvers,
{\idas} is written in ANSI-standard C rather than {\F}77.  
Its most notable features are that, 
(1) in the solution of the underlying nonlinear system at each time step, it 
offers a choice of Newton/direct methods and a choice of Inexact Newton/Krylov
(iterative) methods;
(2) it is written in a {\em data-independent} manner in that it acts
on generic vectors without any assumptions on the underlying organization
of the data; and
(3) it provides a flexible, extensible framework for sensitivity analysis,
using either {\em forward} or {\em adjoint} methods.
Thus {\idas} shares significant modules previously written within CASC
at LLNL to support the ordinary differential equation (ODE) solvers
{\cvode}~\cite{cvode_ug,CoHi:96} and {\pvode}~\cite{ByHi:98,ByHi:99},
the DAE solver {\ida}~\cite{ida_ug} on which {\idas} is based, the
sensitivity-enabled ODE solver {\cvodes}~\cite{cvodes_ug,SeHi:05}, and
also the nonlinear system solver {\kinsol}~\cite{kinsol_ug}.

The Newton/Krylov methods in {\idas} are:
the GMRES (Generalized Minimal RESidual)~\cite{SaSc:86},
Bi-CGStab (Bi-Conjugate Gradient Stabilized)~\cite{Van:92}, and
TFQMR (Transpose-Free Quasi-Minimal Residual) linear iterative 
methods~\cite{Fre:93}.  As Krylov methods, these require almost no 
matrix storage for solving the Newton equations as compared to direct
methods. However, the algorithms allow for a user-supplied preconditioner
matrix, and for most problems preconditioning is essential for an
efficient solution.

For very large DAE systems, the Krylov methods are preferable over
direct linear solver methods, and are often the only feasible choice.
Among the three Krylov methods in {\idas}, we recommend GMRES as the
best overall choice.  However, users are encouraged to compare all
three, especially if encountering convergence failures with GMRES.
Bi-CGFStab and TFQMR have an advantage in storage requirements, in
that the number of workspace vectors they require is fixed, while that
number for GMRES depends on the desired Krylov subspace size.

\index{IDAS@{\idas}!relationship to {\ida}|(}
{\idas} is written with a functionality that is a superset of that
of {\ida}. Sensitivity analysis capabilities, both
forward and adjoint, have been added to the main integrator. Enabling
forward sensitivity computations in {\idas} will result in the
code integrating the so-called {\em sensitivity equations}
simultaneously with the original IVP, yielding both the solution and
its sensitivity with respect to parameters in the model. Adjoint
sensitivity analysis, most useful when the gradients of relatively few
functionals of the solution with respect to many parameters are
sought, involves integration of the original IVP forward in time
followed by the integration of the so-called {\em adjoint equations}
backward in time. {\idas} provides the infrastructure needed to
integrate any final-condition ODE dependent on the solution of the
original IVP (in particular the adjoint system).
\index{IDAS@{\idas}!relationship to {\ida}|)}

\index{IDAS@{\idas}!motivation for writing in C|(}
There are several motivations for choosing the {\C} language for {\idas}.
First, a general movement away from {\F} and toward {\C} in scientific
computing is apparent.  Second, the pointer, structure, and dynamic
memory allocation features in {\C} are extremely useful in software of
this complexity, with the great variety of method options offered.
Finally, we prefer {\C} over {\CPP} for {\idas} because of the wider
availability of {\C} compilers, the potentially greater efficiency of {\C},
and the greater ease of interfacing the solver to applications written
in extended {\F}.
\index{IDAS@{\idas}!motivation for writing in C|)}


\section{Changes from previous versions}

\subsection*{Changes in v1.2.0}

Two major additions were made to the linear system solvers that are
available for use with the {\idas} solver.  First, in the serial case,
an interface to the sparse direct solver KLU was added.
Second, an interface to SuperLU\_MT, the multi-threaded version of
SuperLU, was added as a thread-parallel sparse direct solver option,
to be used with the serial version of the NVECTOR module.
As part of these additions, a sparse matrix (CSC format) structure 
was added to {\idas}.

Otherwise, only relatively minor modifications were made to {\idas}:

In \id{IDARootfind}, a minor bug was corrected, where the input
array \id{rootdir} was ignored, and a line was added to break out of
root-search loop if the initial interval size is below the tolerance
\id{ttol}.

In \id{IDALapackBand}, the line \id{smu = MIN(N-1,mu+ml)} was changed to
\id{smu = mu + ml} to correct an illegal input error for \id{DGBTRF/DGBTRS}.

An option was added in the case of Adjoint Sensitivity Analysis with
dense or banded Jacobian:  With a call to \id{IDADlsSetDenseJacFnBS} or
\id{IDADlsSetBandJacFnBS}, the user can specify a user-supplied Jacobian
function of type \id{IDADls***JacFnBS}, for the case where the backward
problem depends on the forward sensitivities.

In the User Guide, a paragraph was added in Section 6.2.1 on
\id{IDAAdjReInit}, and a paragraph was added in Section 6.2.9 on
\id{IDAGetAdjY}.

Two new {\nvector} modules have been added for parallel computing
environments --- one for openMP, denoted \id{NVECTOR\_OPENMP},
and one for Pthreads, denoted \id{NVECTOR\_PTHREADS}.

With this version of {\sundials}, support and documentation of the
Autotools mode of installation is being dropped, in favor of the
CMake mode, which is considered more widely portable.

\subsection*{Changes in v1.1.0}

One significant design change was made with this release: The problem
size and its relatives, bandwidth parameters, related internal indices,
pivot arrays, and the optional output \id{lsflag} have all been
changed from type \id{int} to type \id{long int}, except for the
problem size and bandwidths in user calls to routines specifying
BLAS/LAPACK routines for the dense/band linear solvers.  The function
\id{NewIntArray} is replaced by a pair \id{NewIntArray}/\id{NewLintArray},
for \id{int} and \id{long int} arrays, respectively.  In a minor
change to the user interface, the type of the index \id{which} in
IDAS was changed from \id{long int} to \id{int}.

Errors in the logic for the integration of backward problems were
identified and fixed.

A large number of minor errors have been fixed.  Among these are the following:
A missing vector pointer setting was added in \id{IDASensLineSrch}.
In \id{IDACompleteStep}, conditionals around lines loading a new column of three
auxiliary divided difference arrays, for a possible order increase, were fixed.
After the solver memory is created, it is set to zero before being filled.
In each linear solver interface function, the linear solver memory is
freed on an error return, and the \id{**Free} function now includes a
line setting to NULL the main memory pointer to the linear solver memory.
A memory leak was fixed in two of the \id{IDASp***Free} functions.
In the rootfinding functions \id{IDARcheck1}/\id{IDARcheck2}, when an exact
zero is found, the array \id{glo} of $g$ values at the left endpoint is
adjusted, instead of shifting the $t$ location \id{tlo} slightly.
In the installation files, we modified the treatment of the macro
SUNDIALS\_USE\_GENERIC\_MATH, so that the parameter GENERIC\_MATH\_LIB is
either defined (with no value) or not defined.


\section{Reading this User Guide}\label{ss:reading}

The structure of this document is as follows:
\begin{itemize}
\item
  In Chapter \ref{s:math}, we give short descriptions of the numerical 
  methods implemented by {\idas} for the solution of initial value problems
  for systems of DAEs, continue with short descriptions of preconditioning
  (\S\ref{s:preconditioning}) and rootfinding (\S\ref{s:rootfinding}), and
  then give an overview of the mathematical aspects of sensitivity analysis,
  both forward (\S\ref{ss:fwd_sensi}) and adjoint (\S\ref{ss:adj_sensi}).
\item
  The following chapter describes the structure of the {\sundials} suite
  of solvers (\S\ref{ss:sun_org}) and the software organization of the {\idas}
  solver (\S\ref{ss:idas_org}). 
\item
  Chapter \ref{s:simulation} is the main usage document for {\idas}
  for simulation applications.  It includes a complete description of
  the user interface for the integration of DAE initial value problems.
  Readers that are not interested in using {\idas} for sensitivity
  analysis can then skip the next two chapters.
\item
  Chapter \ref{s:forward} describes the usage of {\idas} for forward
  sensitivity analysis as an extension of its IVP integration
  capabilities.  We begin with a skeleton of the user main program,
  with emphasis on the steps that are required in addition to those
  already described in Chapter \ref{s:simulation}.  Following that we
  provide detailed descriptions of the user-callable interface
  routines specific to forward sensitivity analysis and of the
  additonal optional user-defined routines.
\item
  Chapter \ref{s:adjoint} describes the usage of {\idas} for adjoint
  sensitivity analysis. We begin by describing the {\idas} checkpointing 
  implementation for interpolation of the original IVP solution during
  integration of the adjoint system backward in time, and with 
  an overview of a user's main program. Following that we provide complete
  descriptions of the user-callable interface routines for adjoint sensitivity
  analysis as well as descriptions of the required additional user-defined routines.
\item
  Chapter \ref{s:nvector} gives a brief overview of the generic
  {\nvector} module shared amongst the various components of
  {\sundials}, as well as details on the two {\nvector}
  implementations provided with {\sundials}: a serial implementation
  (\S\ref{ss:nvec_ser}) and a parallel implementation based on
  {\mpi}\index{MPI} (\S\ref{ss:nvec_par}).
\item
  Chapter \ref{s:new_linsolv} describes the specifications of linear
  solver modules as supplied by the user.
\item
  Chapter \ref{s:gen_linsolv} describes in detail the generic linear solvers shared 
  by all {\sundials} solvers.
\item
  Finally, in the appendices, we provide detailed instructions for the installation
  of {\idas}, within the structure of {\sundials} (Appendix \ref{c:install}), as
  well as a list of all the constants used for input to and output from {\idas}
  functions
  (Appendix \ref{c:constants}).
\end{itemize}

The reader should be aware of the following notational conventions
in this user guide:  program listings and identifiers (such as \id{IDAInit}) 
within textual explanations appear in typewriter type style; 
fields in {\C} structures (such as {\em content}) appear in italics;
and packages or modules, such as {\idadense}, are written in all capitals. 
Usage and installation instructions that constitute important warnings
are marked with a triangular symbol {\warn} in the margin.
