%===================================================================================
\chapter{Providing Alternate Linear Solver Modules}\label{s:new_linsolv}
%===================================================================================
The central {\idas} module interfaces with the linear solver module to be
used by way of calls to five functions.  These are denoted here by 
\id{linit}, \id{lsetup}, \id{lsolve}, \id{lperf}, and \id{lfree}.
Briefly, their purposes are as follows:
%%
%%
\begin{itemize}
\item \id{linit}: initialize memory specific to the linear solver;
\item \id{lsetup}: evaluate and preprocess the Jacobian or preconditioner;
\item \id{lsolve}: solve the linear system;
\item \id{lperf}: monitor performance and issue warnings;
\item \id{lfree}: free the linear solver memory.
\end{itemize}
%%
%%
A linear solver module must also provide a user-callable {\bf specification function}
(like those described in \S\ref{sss:lin_solv_init}) which will attach
the above five functions to the main {\idas} memory block. 
The {\idas} memory block is a structure defined in the header file \id{idas\_impl.h}. 
A pointer to such a structure is defined as the type \id{IDAMem}. 
The five fields in an \id{IDAMem} structure that must point to the linear solver's 
functions are \id{ida\_linit}, \id{ida\_lsetup}, \id{ida\_lsolve}, \id{ida\_lperf}, 
and \id{ida\_lfree}, respectively.
Note that of the five interface functions, only \id{lsolve} is required. 
The \id{lfree} function must be provided only if the solver specification function
makes any memory allocation.   For any of the functions that are
{\it not} provided, the corresponding field should be set to \id{NULL}.
The linear solver specification function must also set the value of
the field \id{ida\_setupNonNull} in the {\idas} memory block --- to
\id{TRUE} if \id{lsetup} is used, or \id{FALSE} otherwise.

Typically, the linear solver will require a block of memory specific
to the solver, and a principal function of the specification function
is to allocate that memory block, and initialize it.  Then the field
\id{ida\_lmem} in the {\idas} memory block is available to attach a
pointer to that linear solver memory.  This block can then be used
to facilitate the exchange of data between the five interface
functions.

If the linear solver involves adjustable parameters, the specification
function should set the default values of those.  User-callable
functions may be defined that could, optionally, override the default
parameter values.

We encourage the use of performance counters in connection with the various
operations involved with the linear solver.  Such counters would be
members of the linear solver memory block, would be initialized in the
\id{linit} function, and would be incremented by the \id{lsetup} and
\id{lsolve} functions.  Then user-callable functions would be needed to
obtain the values of these counters.

For consistency with the existing {\idas} linear solver modules, we
recommend that the return value of the specification function be 0 for
a successful return, and a negative value if an error occurs.
Possible error conditions include: the pointer to the main {\idas}
memory block is \id{NULL}, an input is illegal, the {\nvector}
implementation is not compatible, or a memory allocation fails.

To be used during the backward integration with the {\idas} module,
a linear solver module must also provide an additional user-callable
specification function (like those described in \S\ref{sss:lin_solv_b})
which will attach the five functions to the {\idas} memory block for
each backward integration.
Note that this block, of type \id{IDAMem}, is not directly accessible
to the specification function, but rather is itself a field in the
{\idas} memory block.  For a given backward problem identifier \id{which},
the corresponding memory block must be located in the linked list starting
at \id{ida\_mem->ida\_adj\_mem->IDAB\_mem}; see for example the function
\id{IDADenseB} for specific details.
This specification function must also allocate the linear solver memory for
the backward problem, and attach that, as well as a corresponding memory free
function, to the above block \id{IDAB\_mem}, of type \id{struct IDABMemRec}.
The specification function for backward integration should return a negative
value if it encounters an illegal input, if backward integration has
not been initialized, or if its memory allocation failed.

\vspace{0.1in}
These five functions, which interface between {\idas} and the linear solver module,
necessarily have fixed call sequences.  Thus a user wishing to implement another 
linear solver within the {\idas} package must adhere to this set of interfaces.
The following is a complete description of the call list for each of
these functions.  Note that the call list of each function includes a
pointer to the main {\idas} memory block, by which the function can access
various data related to the {\idas} solution.  The contents of this memory
block are given in the file \id{idas\_impl.h} (but not reproduced here, for
the sake of space).

%------------------------------------------------------------------------------

\section{Initialization function}
The type definition of \id{linit} is
\usfunction{linit}
{
  int (*linit)(IDAMem IDA\_mem);
}
{
  The purpose of \id{linit} is to complete initializations for
  the specific linear solver, such as counters and statistics.
  It should also set pointers to data blocks that will later be
  passed to functions associated with the linear solver.
  The \id{linit} function is called once only, at the start of
  the problem, during the first call to \id{IDASolve}.
}
{
  \begin{args}[IDA\_mem]
  \item[IDA\_mem]
    is the {\idas} memory pointer of type \id{IDAMem}.
  \end{args}
}
{
  An \id{linit} function should return 0 if it 
  has successfully initialized the {\idas} linear solver and 
  a negative value otherwise. 
}
{}

%------------------------------------------------------------------------------

\section{Setup function} 
The type definition of \id{lsetup} is
\usfunction{lsetup}
{
   int (*lsetup)(&IDAMem IDA\_mem, N\_Vector yyp, N\_Vector ypp, 
                 N\_Vector resp,\\
                 &N\_Vector vtemp1, N\_Vector vtemp2, N\_Vector vtemp3); 
}
{
  The job of \id{lsetup} is to prepare the linear solver for subsequent 
  calls to \id{lsolve}, in the solution of systems $M x = b$, where         
  $M$ is some approximation to the system Jacobian,
  $J = \partial F / \partial y + cj ~ \partial F / \partial \dot{y}$. 
  (See Eqn. (\ref{e:DAE_Jacobian}), in which $\alpha = cj$).
  Here $cj$ is available as \id{IDA\_mem->ida\_cj}.

  The \id{lsetup} function may call a user-supplied function, or a function
  within the linear solver module, to compute Jacobian-related data
  that is required by the linear solver.  It may also preprocess that
  data as needed for \id{lsolve}, which may involve calling a generic
  function (such as for LU factorization).  This data may be intended
  either for direct use (in a direct linear solver) or for use in a
  preconditioner (in a preconditioned iterative linear solver).

  The \id{lsetup} function is not called at every time step, but only
  as frequently as the solver determines that it is appropriate to perform
  the setup task.  In this way, Jacobian-related data generated by \id{lsetup}
  is expected to be used over a number of time steps.
}
{
   \begin{args}[IDA\_mem]
   \item[IDA\_mem] 
     is the {\idas} memory pointer of type \id{IDAMem}.
  
   \item[yyp]
     is the predicted $y$ vector for the current {\idas} internal step.
  
   \item[ypp]
     is the predicted $\dot{y}$ vector for the current {\idas} internal step.
  
   \item[resp]
     is the value of the residual function at \id{yyp} and \id{ypp}, i.e.
     $F(t_n, y_{pred}, \dot{y}_{pred})$.
  
   \item[vtemp1] 
   \item[vtemp2]
   \item[vtemp3] 
     are temporary variables of type \id{N\_Vector} provided for use by \id{lsetup}.      
  
   \end{args}
}
{
  The \id{lsetup} function should return 0 if successful,            
  a positive value for a recoverable error, and a negative value  
  for an unrecoverable error.  On a recoverable error return, the
  solver will attempt to recover by reducing the step size.
}
{}

%------------------------------------------------------------------------------

\section{Solve function}
The type definition of \id{lsolve} is
\usfunction{lsolve}
{
  int (*lsolve)(&IDAMem IDA\_mem, N\_Vector b, N\_Vector weight, \\
                &N\_Vector ycur, N\_Vector ypcur, N\_Vector rescur);  
}
{
  The \id{lsolve} function must solve the linear system $M x = b$, where         
  $M$ is some approximation to the system Jacobian
  $J = \partial F / \partial y + cj ~ \partial F / \partial \dot{y}$  
  (see Eqn. (\ref{e:DAE_Jacobian}), in which $\alpha = cj$),
  and the right-hand side vector $b$ is input. 
  Here $cj$ is available as \id{IDA\_mem->ida\_cj}.

  \id{lsolve} is called once per Newton iteration, hence possibly
  several times per time step.

  If there is an \id{lsetup} function, this \id{lsolve} function should
  make use of any Jacobian data that was computed and preprocessed
  by \id{lsetup}, either for direct use, or for use in a preconditioner.
}
{
  \begin{args}[IDA\_mem]
  \item[IDA\_mem]
    is the {\idas} memory pointer of type \id{IDAMem}.
  \item[b]
    is the right-hand side vector $b$. The solution is to be    
    returned in the vector \id{b}.
  \item[weight]
    is a vector that contains the error weights.
    These are the $W_i$ of (\ref{e:errwt}).
    This weight vector is included here to enable the computation of
    weighted norms needed to test for the convergence of iterative methods
    (if any) within the linear solver.
  \item[ycur]
    is a vector that contains the solver's current approximation to $y(t_n)$.
  \item[ypcur]
    is a vector that contains the solver's current approximation to $\dot{y}(t_n)$.
  \item[rescur]
    is a vector that contains the current residual, $F(t_n,$ \id{ycur}, \id{ypcur}$)$.
  \end{args}
}
{
  The \id{lsolve} function should return a positive value    
  for a recoverable error and a negative value for an             
  unrecoverable error. Success is indicated by a 0 return value.
  On a recoverable error return, the solver will attempt to recover, such
  as by calling the \id{lsetup} function with current arguments.
}
{}

%------------------------------------------------------------------------------

\section{Performance monitoring function}
The type definition of \id{lperf} is
\usfunction{lperf}
{
  int (*lperf)(IDAMem IDA\_mem, int perftask);
}
{
  The \id{lperf} function is to monitor the performance of the linear solver.
  It can be used to compute performance metrics related to the linear solver
  and issue warnings if these indicate poor performance of the linear solver.
  The \id{lperf} function is called with \id{perftask} = 0 at the start of
  each call to \id{IDASolve}, and then is called with \id{perftask} = 1 just
  before each internal time step.
}
{
  \begin{args}[IDA\_mem]
  \item[IDA\_mem]
    is the {\idas} memory pointer of type \id{IDAMem}.
  \item[perftask]
    is a task flag.  \id{perftask} = 0 means initialize needed counters.
    \id{perftask} = 1 means evaluate performance and issue warnings if needed.
    Counters that are used to compute performance metrics (e.g. counts of
    iterations within the \id{lsolve} function) should be initialized here when
    \id{perftask} = 0, and used for the calculation of metrics when
    \id{perftask} = 1.
  \end{args}
}
{
  The \id{lperf} return value is ignored.
}
{}

%------------------------------------------------------------------------------

\section{Memory deallocation function}
The type definition of \id{lfree} is
\usfunction{lfree}
{
  void (*lfree)(IDAMem IDA\_mem);
}
{
  The \id{lfree} function should free up any memory allocated by the linear
  solver.
}
{
  The argument \id{IDA\_mem} is the {\idas} memory pointer of type \id{IDAMem}.
}
{
  The \id{lfree} function has no return value.
}
{
  This function is called once a problem has been completed and the 
  linear solver is no longer needed.
}
