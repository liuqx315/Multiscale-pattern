%===================================================================================
\chapter{Providing Alternate Linear Solver Modules}\label{s:new_linsolv}
%===================================================================================
The central {\kinsol} module interfaces with the linear solver module 
to be used by way of calls to four functions.  These are denoted here by 
\id{linit}, \id{lsetup}, \id{lsolve}, and \id{lfree}.
Briefly, their purposes are as follows:
%%
%%
\begin{itemize}
\item \id{linit}: initialize memory specific to the linear solver;
\item \id{lsetup}: evaluate and preprocess the Jacobian or preconditioner;
\item \id{lsolve}: solve the linear system;
\item \id{lfree}: free the linear solver memory.
\end{itemize}
%%
%%
A linear solver module must also provide a user-callable {\bf specification function}
(like that described in \S\ref{sss:lin_solv_init}) which will
attach the above four functions to the main {\kinsol} memory block.
The {\kinsol} memory block is a structure defined in the header file
\id{kinsol\_impl.h}. A pointer to such a structure is defined as the
type \id{KINMem}. The four fields in a \id{KINMem} structure that
must point to the linear solver's functions are \id{kin\_linit},
\id{kin\_lsetup}, \id{kin\_lsolve}, and \id{kin\_lfree}, respectively.
Note that of the four interface functions, only the \id{lsolve} function
is required. The \id{lfree} function must be provided only if the solver
specification function makes any memory allocation.
For any of the functions that are {\it not} provided, the
corresponding field should be set to \id{NULL}.
The linear solver specification function must also set the value of
the field \id{kin\_setupNonNull} in the {\kinsol} memory block --- to
\id{TRUE} if \id{lsetup} is used, or \id{FALSE} otherwise.

Typically, the linear solver will require a block of memory specific
to the solver, and a principal function of the specification function
is to allocate that memory block, and initialize it.  Then the field
\id{kin\_lmem} in the {\kinsol} memory block is available to attach a
pointer to that linear solver memory.  This block can then be used
to facilitate the exchange of data between the four interface
functions.

If the linear solver involves adjustable parameters, the specification
function should set the default values of those.  User-callable
functions may be defined that could, optionally, override the default
parameter values.

We encourage the use of performance counters in connection with the various
operations involved with the linear solver.  Such counters would be
members of the linear solver memory block, would be initialized in the
\id{linit} function, and would be incremented by the \id{lsetup} and
\id{lsolve} functions.  Then user-callable functions would be needed to
obtain the values of these counters.

For consistency with the existing {\kinsol} linear solver modules, we
recommend that the return value of the specification function be 0 for
a successful return, and a negative value if an error occurs.
Possible error conditions include: the pointer to the main {\kinsol}
memory block is \id{NULL}, an input is illegal, the {\nvector}
implementation is not compatible, or a memory allocation fails.

\vspace{0.1in}
These four functions, which interface between {\kinsol} and the linear solver module,
necessarily have fixed call sequences.  Thus a user wishing to implement another 
linear solver within the {\kinsol} package must adhere to this set of interfaces.
The following is a complete description of the call list for each of
these functions.  Note that the call list of each function includes a
pointer to the main {\kinsol} memory block, by which the function can access
various data related to the {\kinsol} solution.  The contents of this memory
block are given in the file \id{kinsol\_impl.h} (but not reproduced here, for
the sake of space).

%------------------------------------------------------------------------------

\section{Initialization function}
The type definition of \ID{linit} is
\usfunction{linit}
{
  int (*linit)(KINMem kin\_mem);
}
{
  The purpose of \id{linit} is to complete initializations for      
  a specific linear solver, such as counters and statistics.        
  It should also set pointers to data blocks that will later be
  passed to functions associated with the linear solver.
  The \id{linit} function is called once only, at the start of
  the problem, by \id{KINSol}.
}
{
  \begin{args}[kin\_mem]
  \item[kin\_mem]
    is the {\kinsol} memory pointer of type \id{KINMem}.
  \end{args}
}
{
  An \id{linit} function should return 0 if it 
  has successfully initialized the {\kinsol} linear solver, and 
  a negative value otherwise. 
}
{}

%------------------------------------------------------------------------------

\section{Setup function} 
The type definition of \id{lsetup} is
\usfunction{lsetup}
{
  int (*lsetup)(&KINMem kin\_mem);
}
{
  The job of \id{lsetup} is to prepare the linear solver for subsequent 
  calls to \id{lsolve}, in the solution of linear systems $A x = b$.
  The exact nature of this system depends on the input \id{strategy} to
  \id{KINSol}.  In the cases \id{strategy} = \id{KIN\_NONE} or
  \id{KIN\_LINESEARCH}, $A$ is the Jacobian $J = \partial F / \partial u$.
  If \id{strategy} = \id{KIN\_PICARD}, $A$ is the approximate Jacobian matrix $L$.
  If \id{strategy} = \id{KIN\_FP}, linear systems do not arise.

  The \id{lsetup} function may call a user-supplied function, or a function
  within the linear solver module, to compute Jacobian-related data
  that is required by the linear solver.  It may also preprocess that
  data as needed for \id{lsolve}, which may involve calling a generic
  function (such as for LU factorization).  This data may be intended
  either for direct use (in a direct linear solver) or for use in a
  preconditioner (in a preconditioned iterative linear solver).

  The \id{lsetup} function is not called at every Newton iteration, but only
  as frequently as the solver determines that it is appropriate to perform
  the setup task.  In this way, Jacobian-related data generated by \id{lsetup}
  is expected to be used over a number of Newton iterations.
}
{
  \begin{args}[kin\_mem]
  \item[kin\_mem]
    is the {\kinsol} memory pointer of type \id{KINMem}.
  \end{args}
}
{
  The \id{lsetup} function should return 0 if successful and
  a non-zero value otherwise.
}
{
  The current values of $u$ and $F(u)$ can be accessed by \id{lsetup} through
  the fields \id{kin\_uu} and \id{kin\_fval} (respectively) in \id{kin\_mem}.
  If needed, the scaling vectors \id{u\_scale} and \id{f\_scale} can be
  accessed by \id{lsetup} through the fields \id{kin\_uscale} and \id{kin\_fscale}
  (respectively) in \id{kin\_mem}.
}

%------------------------------------------------------------------------------

\section{Solve function}
The type definition of \id{lsolve} is
\usfunction{lsolve}
{
  int (*lsolve)(&KINMem kin\_mem, N\_Vector x, N\_Vector b, \\
                &realtype *sJpnorm, realtype *sFdotJp);
}
{
  The \id{lsolve} function must solve the linear system $A x = b$, where $A$ is
  either the Jacobian $J = \partial F / \partial u$ (evaluated at the current
  iterate), or the approximate Jacobian $L$ in the case of Picard iteration.
  The right-hand side vector $b$ is input. 
}
{
  \begin{args}[N\_Vector]
  \item[kin\_mem]
    is the {\kinsol} memory pointer of type \id{KINMem}.
  \item[x]
    is a vector set to an initial guess prior to calling \id{lsolve}. 
    On return it should contain the solution to $J x = b$.
  \item[b]
    is the right-hand side vector $b$, set to $-F(u)$, evaluated at
    the current iterate.
  \item[sJpnorm]
    is a pointer to a real scalar to be computed by \id{lsolve}.  The returned
    value \id{sJpnorm} should be equal to $\|D_F Jp\|_2$, the scaled $L_2$
    norm of the product $Jp$, where $p~(= x)$ is the computed solution of the
    linear system $J p = b$, and the scaling is that given by $D_F$.
  \item[sFdotJp]
    is a pointer to a real scalar to be computed by \id{lsolve}.  The returned
    value  \id{sFdotJp} should be equal to $(D_F F)\cdot(D_F Jp)$, the
    dot product of the scaled $F$ vector and the scaled vector $J p$, where
    $p~(= x)$ is the computed solution of the linear system $J p = b$,
    and the scaling is that given by $D_F$.
  \end{args}
}
{
  \id{lsolve} should return $0$ if successful.
  If an error occurs and recovery could be possible by calling the
  \id{lsetup} function again, then it should return a positive value.
  Otherwise, \id{lsolve} should return a negative value.
}
{
  The current values of $u$ and $F(u)$ can be accessed by \id{lsolve} through
  the fields \id{kin\_uu} and \id{kin\_fval} (respectively) in \id{kin\_mem},
  and the scaling vectors \id{u\_scale} and \id{f\_scale} can be accessed
  through the fields \id{kin\_uscale} and \id{kin\_fscale} (respectively)
   in \id{kin\_mem}.
}

In the case of a direct solver, \id{sJpnorm} can be ignored, and
\id{sFdotJp} can be computed with lines of the form
\vspace{-.1in}
  \begin{verbatim}
  N_VProd(b, f_scale, b);
  N_VProd(b, f_scale, b);
  *sFdotJp = N_VDotProd(fval, b);
  \end{verbatim}
\vspace{-.2in}
in which $Jp$ is taken to be equal to the input right-hand side $b$,
and \id{f\_scale} and \id{fval} $(= F(u))$ are taken from \id{kin\_mem}.

In the case of an iterative solver, the two terms \id{sJpnorm} and
\id{sFdotJp} can be computed with lines of the form
\vspace{-.1in}
  \begin{verbatim}
  ret = KINSpilsAtimes(kin_mem, x, b);
  *sJpnorm = N_VWL2Norm(b, f_scale);
  N_VProd(b, f_scale, b);
  N_VProd(b, f_scale, b);
  *sFdotJp = N_VDotProd(fval, b);
  \end{verbatim}
\vspace{-.2in}
following the computation of the solution vector \id{x}, in which
\id{f\_scale} and \id{fval} $(= F(u))$ are taken from \id{kin\_mem}.

The values \id{sFdotJp} and \id{sFdotJp} need not be set in all cases,
and so for maximum efficiency, the \id{lsolve} function could do these
calculations conditionally, depending on the value of the input \id{strategy}
to \id{KINSol}, and the choice (given by \id{etachoice}) of Forcing Term
in the Krylov iteration stopping test (see \id{KINSetEtaForm}).
The precise conditions are as follows:  First, if \id{strategy} is
\id{KIN\_FP}, neither of these quantities need to be computed.
In the other cases, if the linear solver is iterative
and \id{etachoice} = \id{KIN\_ETACHOICE1} (the default) then both
\id{sFdotJp} and \id{sFdotJp} must be set.  If \id{strategy} is
\id{KIN\_LINESEARCH}, then \id{sFdotJp} must be set, regardless of
the linear solver type.

The values of \id{strategy} and \id{etachoice} are available from the fields
\id{kin\_global\_strategy} and \id{kin\_etaflag} (respectively) in \id{kin\_mem}.

%------------------------------------------------------------------------------

\section{Memory deallocation function}
The type definition of \id{lfree} is
\usfunction{lfree}
{
  void (*lfree)(KINMem kin\_mem);
}
{
  The \id{lfree} function should free any memory allocated by the linear solver.
}
{
  \begin{args}[kin\_mem]
  \item[kin\_mem]
    is the {\kinsol} memory pointer of type \id{KINMem}.
  \end{args}
}
{
  The \id{lfree} function has no return value.
}
{
  This function is called once a problem has been completed and the 
  linear solver is no longer needed.
}