%===================================================================================
\chapter{Providing Alternate Linear Solver Modules}\label{s:new_linsolv}
%===================================================================================
The central {\kinsol} module interfaces with the linear solver module 
to be used by way of calls to four routines.  These are denoted here by 
\id{linit}, \id{lsetup}, \id{lsolve}, and \id{lfree}.
Briefly, their purposes are as follows:
%%
%%
\begin{itemize}
\item \id{linit}: initialize and allocate memory specific to the
  linear solver;
\item \id{lsetup}: evaluate and preprocess the Jacobian or preconditioner;
\item \id{lsolve}: solve the linear system;
\item \id{lfree}: free the linear solver memory.
\end{itemize}
%%
%%
A linear solver module must also provide a user-callable specification
routine (like that described in \S\ref{sss:lin_solv_init}) which will
attach the above four routines to the main {\kinsol} memory block.
The {\kinsol} memory block is a structure defined in the header file
\id{kinsol\_impl.h}. A pointer to such a structure is defined as the
type \id{KINMem}. The four fields in a \id{KINMem} structure that
must point to the linear solver's functions are \id{kin\_linit},
\id{kin\_lsetup}, \id{kin\_lsolve}, and \id{kin\_lfree}, respectively.
Note that of the four interface routines, only the \id{lsolve} routine
is required. The \id{lfree} routine must be provided only if the solver
specification routine makes any memory allocation.
The linear solver specification function must also set the value of
the field \id{kin\_setupNonNull} in the {\kinsol} memory block --- to
\id{TRUE} if \id{lsetup} is used, or \id{FALSE} otherwise.

For consistency with the existing {\kinsol} linear solver modules, we
recommend that the return value of the specification function be $0$ for
a successful return or a negative value if an error occurs (the
pointer to the main {\kinsol} memory block is \id{NULL}, an input is
illegal, the {\nvector} implementation is not compatible, a memory
allocation fails, etc.)

To facilitate data exchange between the four interface functions, the
field \id{kin\_lmem} in the {\kinsol} memory block can be used to attach
a linear solver-specific memory block.
That memory should be allocated in the linear solver specification function.

These four routines, which interface between {\kinsol} and the linear solver module,
necessarily have fixed call sequences.  Thus, a user wishing to implement another 
linear solver within the {\kinsol} package must adhere to this set of interfaces.
The following is a complete description of the call list for each of
these routines.  Note that the call list of each routine includes a
pointer to the main {\kinsol} memory block, by which the routine can access
various data related to the {\kinsol} solution.  The contents of this memory
block are given in the file \id{kinsol\_impl.h} (but not reproduced here, for
the sake of space).

%------------------------------------------------------------------------------

\section{Initialization function}
The type definition of \ID{linit} is
\usfunction{linit}
{
  int (*linit)(KINMem kin\_mem);
}
{
  The purpose of \id{linit} is to complete initializations for      
  a specific linear solver, such as counters and statistics.        
}
{
  \begin{args}[kin\_mem]
  \item[kin\_mem]
    is the {\kinsol} memory pointer of type \id{KINMem}.
  \end{args}
}
{
  An \id{linit} function should return $0$ if it 
  has successfully initialized the {\kinsol} linear solver and 
  $-1$ otherwise. 
}
{}

%------------------------------------------------------------------------------

\section{Setup function} 
The type definition of \id{lsetup} is
\usfunction{lsetup}
{
  int (*lsetup)(&KINMem kin\_mem);
}
{
  The job of \id{lsetup} is to prepare the linear solver for subsequent 
  calls to \id{lsolve}. It may recompute Jacobian-related data if it 
  deems necessary. 
}
{
  \begin{args}[kin\_mem]
  \item[kin\_mem]
    is the {\kinsol} memory pointer of type \id{KINMem}.
  \end{args}
}
{
  The \id{lsetup} routine should return $0$ if successful and
  a non-zero value otherwise.
}
{}

%------------------------------------------------------------------------------

\section{Solve function}
The type definition of \id{lsolve} is
\usfunction{lsolve}
{
  int (*lsolve)(&KINMem kin\_mem, N\_Vector x, \\
                &N\_Vector b, realtype *res\_norm);
}
{
  The routine \id{lsolve} must solve the linear equation $J x = b$, where         
  $J = \partial F / \partial u$ is evaluated at the current iterate
  and the right-hand side vector $b$ is input. 
}
{
  \begin{args}[res\_norm]
  \item[kin\_mem]
    is the {\kinsol} memory pointer of type \id{KINMem}.
  \item[x]
    is a vector set to an initial guess prior to calling \id{lsolve}. 
    On return it should contain the solution to $J x = b$.
  \item[b]
    is the right-hand side vector $b$, set to $-F(u)$, evaluated at
    the current iterate.
  \item[res\_norm]
    holds the value of the $L_2$ norm of the residual vector upon return.
  \end{args}
}
{
  \id{lsolve} should return $0$ if successful.
  If an error occurs and recovery could be possible by calling again
  the \id{lsetup} function, then it should return a positive value.
  Otherwise, \id{lsolve} should return a negative value.
}
{}

%------------------------------------------------------------------------------

\section{Memory deallocation function}
The type definition of \id{lfree} is
\usfunction{lfree}
{
  void (*lfree)(KINMem kin\_mem);
}
{
  The routine \id{lfree} should free any linear solver memory
  allocated by the \id{linit} routine.
}
{
  \begin{args}[kin\_mem]
  \item[kin\_mem]
    is the {\kinsol} memory pointer of type \id{KINMem}.
  \end{args}
}
{
  This routine has no return value.
}
{
  This routine is called once a problem has been completed and the 
  linear solver is no longer needed.
}