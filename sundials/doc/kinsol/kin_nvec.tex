%===================================================================================
\chapter{Description of the NVECTOR module}\label{s:nvector}
%===================================================================================
\index{NVECTOR@\texttt{NVECTOR} module}
% This is a shared SUNDIALS TEX file with description of
% the generic nvector abstraction
%
The {\sundials} solvers are written in a data-independent manner. 
They all operate on generic vectors (of type \Id{N\_Vector}) through a set of
operations defined by the particular {\nvector} implementation.
Users can provide their own specific implementation of the {\nvector} module,
or use one of four provided within {\sundials} -- a serial implementation and
three parallel implementations.  The generic operations are described below.
In the sections following, the implementations provided with {\sundials}
are described.

The generic \ID{N\_Vector} type is a pointer to a structure that has an 
implementation-dependent {\em content} field containing the 
description and actual data of the vector, and an {\em ops} field 
pointing to a structure with generic vector operations.
The type \id{N\_Vector} is defined as
%%
%%
\begin{verbatim}
typedef struct _generic_N_Vector *N_Vector;

struct _generic_N_Vector {
    void *content;
    struct _generic_N_Vector_Ops *ops;
};
\end{verbatim}
%%
%%
The \id{\_generic\_N\_Vector\_Ops} structure is essentially a list of pointers to
the various actual vector operations, and is defined as
%%
\begin{verbatim}
struct _generic_N_Vector_Ops {
  N_Vector    (*nvclone)(N_Vector);
  N_Vector    (*nvcloneempty)(N_Vector);
  void        (*nvdestroy)(N_Vector);
  void        (*nvspace)(N_Vector, long int *, long int *);
  realtype*   (*nvgetarraypointer)(N_Vector);
  void        (*nvsetarraypointer)(realtype *, N_Vector);
  void        (*nvlinearsum)(realtype, N_Vector, realtype, N_Vector, N_Vector); 
  void        (*nvconst)(realtype, N_Vector);
  void        (*nvprod)(N_Vector, N_Vector, N_Vector);
  void        (*nvdiv)(N_Vector, N_Vector, N_Vector);
  void        (*nvscale)(realtype, N_Vector, N_Vector);
  void        (*nvabs)(N_Vector, N_Vector);
  void        (*nvinv)(N_Vector, N_Vector);
  void        (*nvaddconst)(N_Vector, realtype, N_Vector);
  realtype    (*nvdotprod)(N_Vector, N_Vector);
  realtype    (*nvmaxnorm)(N_Vector);
  realtype    (*nvwrmsnorm)(N_Vector, N_Vector);
  realtype    (*nvwrmsnormmask)(N_Vector, N_Vector, N_Vector);
  realtype    (*nvmin)(N_Vector);
  realtype    (*nvwl2norm)(N_Vector, N_Vector);
  realtype    (*nvl1norm)(N_Vector);
  void        (*nvcompare)(realtype, N_Vector, N_Vector);
  booleantype (*nvinvtest)(N_Vector, N_Vector);
  booleantype (*nvconstrmask)(N_Vector, N_Vector, N_Vector);
  realtype    (*nvminquotient)(N_Vector, N_Vector);
};
\end{verbatim}




The generic {\nvector} module defines and implements the vector operations 
acting on \id{N\_Vector}.
These routines are nothing but wrappers for the vector operations defined by
a particular {\nvector} implementation, which are accessed through the {\em ops}
field of the \id{N\_Vector} structure. To illustrate this point we
show below the implementation of a typical vector operation from the
generic {\nvector} module, namely \id{N\_VScale}, which performs the scaling of a
vector \id{x} by a scalar \id{c}:
%%
%%
\begin{verbatim}
void N_VScale(realtype c, N_Vector x, N_Vector z) 
{
   z->ops->nvscale(c, x, z);
}
\end{verbatim}
%%
%%
Table \ref{t:nvecops} contains a complete list of all vector operations defined
by the generic {\nvector} module.

Finally, note that the generic {\nvector} module defines the functions
\ID{N\_VCloneVectorArray} and \ID{N\_VCloneEmptyVectorArray}.  Both functions
create (by cloning) an array of \id{count} variables of type \id{N\_Vector}, each
of the same type as an existing \id{N\_Vector}. Their prototypes are
\begin{verbatim}
N_Vector *N_VCloneVectorArray(int count, N_Vector w);
N_Vector *N_VCloneEmptyVectorArray(int count, N_Vector w);
\end{verbatim}
and their definitions are based on the implementation-specific \id{N\_VClone} and
\id{N\_VCloneEmpty} operations, respectively.

An array of variables of type \id{N\_Vector} can be destroyed by
calling \ID{N\_VDestroyVectorArray}, whose prototype is
\begin{verbatim}
void N_VDestroyVectorArray(N_Vector *vs, int count);
\end{verbatim}
and whose definition is based on the implementation-specific \id{N\_VDestroy} operation.


A particular implementation of the {\nvector} module must:
\begin{itemize}
\item Specify the {\em content} field of \id{N\_Vector}.
\item Define and implement the vector operations. 
  Note that the names of these routines should be unique to that implementation in order 
  to permit using more than one {\nvector} module (each with different \id{N\_Vector} 
  internal data representations) in the same code.
\item Define and implement user-callable constructor and destructor
  routines to create and free an \id{N\_Vector} with
  the new {\em content} field and with {\em ops} pointing to the
  new vector operations.
\item Optionally, define and implement additional user-callable routines
  acting on the newly defined \id{N\_Vector} (e.g., a routine to print
  the content for debugging purposes).
\item Optionally, provide accessor macros as needed for that particular implementation to 
  be used to access different parts in the {\em content} field of the newly defined \id{N\_Vector}.
\end{itemize}



%---------------------------------------------------------------------------
% Table of vector kernels
%---------------------------------------------------------------------------
\newpage

\newlength{\colone}
\settowidth{\colone}{\id{N\_VGetArrayPointer}}
\newlength{\coltwo}
\setlength{\coltwo}{\textwidth}
\addtolength{\coltwo}{-0.5in}
\addtolength{\coltwo}{-\colone}

\tablecaption{Description of the NVECTOR operations}\label{t:nvecops}
\tablefirsthead{\hline {\rule{0mm}{5mm}}{\bf Name} & {\bf Usage and Description} \\[3mm] \hline\hline}
\tablehead{\hline \multicolumn{2}{|l|}{\small\slshape continued from last page} \\
           \hline {\rule{0mm}{5mm}}{\bf Name} & {\bf Usage and  Description} \\[3mm] \hline\hline}
\tabletail{\hline \multicolumn{2}{|r|}{\small\slshape continued on next page} \\ \hline}
\tablelasttail{\hline}
\begin{supertabular}{|p{\colone}|p{\coltwo}|}
%%
\id{N\_VClone} & \id{v = N\_VClone(w);} \\ 
& Creates a new \id{N\_Vector} of the same type as an existing vector \id{w} and sets the
{\em ops} field.
It does not copy the vector, but rather allocates storage for the new vector.
\\[2mm]
%%
\id{N\_VCloneEmpty} & \id{v = N\_VCloneEmpty(w);} \\ 
& Creates a new \id{N\_Vector} of the same type as an existing vector \id{w} and sets the
{\em ops} field.
It does not allocate storage for the data array.
\\[2mm]
%%
\id{N\_VDestroy} & \id{N\_VDestroy(v);} \\
& Destroys the \id{N\_Vector} \id{v} and frees memory allocated for its
internal data.
\\[2mm]
%%
\id{N\_VSpace} & \id{N\_VSpace(nvSpec, \&lrw, \&liw);} \\
& Returns storage requirements for one \id{N\_Vector}.
\id{lrw} contains the number of realtype words and \id{liw}
contains the number of integer words.
This function is advisory only, for use in determining a user's total
space requirements; it could be a dummy function in a user-supplied
{\nvector} module if that information is not of interest.
\\[2mm]
%%
\id{N\_VGetArrayPointer} & \id{vdata = N\_VGetArrayPointer(v);} \\
& Returns a pointer to a \id{realtype} array from the \id{N\_Vector} \id{v}.
Note that this assumes that the internal data in \id{N\_Vector} is
a contiguous array of \id{realtype}.
This routine is only used in the solver-specific interfaces to the dense and
banded (serial) linear solvers, the sparse linear solvers (serial and
threaded), and in the interfaces to the banded (serial)
and band-block-diagonal (parallel) preconditioner modules provided with {\sundials}.
\\[2mm]
%%
\id{N\_VSetArrayPointer} & \id{N\_VSetArrayPointer(vdata, v);} \\
& Overwrites the data in an \id{N\_Vector} with a given array of \id{realtype}.
Note that this assumes that the internal data in \id{N\_Vector} is
a contiguous array of \id{realtype}.
This routine is only used in the interfaces to the dense (serial) linear
solver, hence need not exist in a user-supplied {\nvector} module for a
parallel environment.
\\[2mm]
%%
\id{N\_VLinearSum} & \id{N\_VLinearSum(a, x, b, y, z);} \\
& Performs the operation $z = a x + b y$, where $a$ and $b$ are scalars
and $x$ and $y$ are of type \id{N\_Vector}:
$z_i = a x_i + b y_i, \: i=0,\ldots,n-1$.
\\[2mm]
%%
\id{N\_VConst} & \id{N\_VConst(c, z);} \\
& Sets all components of the \id{N\_Vector} \id{z} to \id{c}:
$z_i = c,\: i=0,\ldots,n-1$.
\\[2mm]
%%
\id{N\_VProd} & \id{N\_VProd(x, y, z);} \\
& Sets the \id{N\_Vector} \id{z} to be the component-wise product of the
\id{N\_Vector} inputs \id{x} and \id{y}:
$z_i = x_i y_i,\: i=0,\ldots,n-1$.
\\[2mm]
%%
\id{N\_VDiv} & \id{N\_VDiv(x, y, z);} \\
& Sets the \id{N\_Vector} \id{z} to be the component-wise ratio of the
\id{N\_Vector} inputs \id{x} and \id{y}:
$z_i = x_i / y_i,\: i=0,\ldots,n-1$. The $y_i$ may not be tested 
for $0$ values. It should only be called with a \id{y} that is
guaranteed to have all nonzero components.
\\[2mm]
%%
\id{N\_VScale} & \id{N\_VScale(c, x, z);} \\
& Scales the \id{N\_Vector} \id{x} by the scalar \id{c} and returns
the result in \id{z}:
$z_i = c x_i , \: i=0,\ldots,n-1$.
\\[2mm]
%%
\id{N\_VAbs} & \id{N\_VAbs(x, z);} \\
& Sets the components of the \id{N\_Vector} \id{z} to be the absolute
values of the components of the \id{N\_Vector} \id{x}:
$y_i = | x_i | , \: i=0,\ldots,n-1$.
\\[2mm]
%%
\id{N\_VInv} & \id{N\_VInv(x, z);} \\
& Sets the components of the \id{N\_Vector} \id{z} to be the inverses
of the components of the \id{N\_Vector} \id{x}:
$z_i = 1.0 /  x_i  , \: i=0,\ldots,n-1$. This routine
may not check for division by $0$. It should be called only with an 
\id{x} which is guaranteed to have all nonzero components.
\\[2mm]
%%
\id{N\_VAddConst} & \id{N\_VAddConst(x, b, z);} \\
& Adds the scalar \id{b} to all components of \id{x} and returns the
result in the \id{N\_Vector} \id{z}:
$z_i = x_i + b , \: i=0,\ldots,n-1$.
\\[2mm]
%%
\id{N\_VDotProd} & \id{d = N\_VDotProd(x, y);} \\
& Returns the value of the ordinary dot product of \id{x} and \id{y}:
$d=\sum_{i=0}^{n-1} x_i y_i$.
\\[2mm]
%%
\id{N\_VMaxNorm} & \id{m = N\_VMaxNorm(x);} \\
& Returns the maximum norm of the \id{N\_Vector} \id{x}:
$m = \max_{i} | x_i |$.
\\[2mm]
%%
\id{N\_VWrmsNorm} & \id{m = N\_VWrmsNorm(x, w)} \\
& Returns the weighted root-mean-square norm of the \id{N\_Vector} \id{x} with
weight vector \id{w}:
$m = \sqrt{\left( \sum_{i=0}^{n-1} (x_i w_i)^2 \right) / n}$.
\\[2mm]
%%
\id{N\_VWrmsNormMask} & \id{m = N\_VWrmsNormMask(x, w, id);} \\
& Returns the weighted root mean square norm of the \id{N\_Vector} \id{x} with
weight vector \id{w} built using only the elements of \id{x} corresponding to
nonzero elements of the \id{N\_Vector} \id{id}:\\
&$m = \sqrt{\left( \sum_{i=0}^{n-1} (x_i w_i \text{sign}(id_i))^2 \right) / n}$.
\\[2mm]
%%
\id{N\_VMin} & \id{m = N\_VMin(x);} \\
& Returns the smallest element of the \id{N\_Vector} \id{x}:
$m = \min_i x_i $.
\\[2mm]
%%
\id{N\_VWL2Norm} & \id{m = N\_VWL2Norm(x, w);} \\
& Returns the weighted Euclidean $\ell_2$ norm of the \id{N\_Vector} \id{x}
with weight vector \id{w}: 
$m = \sqrt{\sum_{i=0}^{n-1} (x_i w_i)^2}$.
\\[2mm]
%%
\id{N\_VL1Norm} & \id{m = N\_VL1Norm(x);} \\
& Returns the $\ell_1$ norm of the \id{N\_Vector} \id{x}:
$m = \sum_{i=0}^{n-1} | x_i |$.
\\[2mm]
%%
\id{N\_VCompare} & \id{N\_VCompare(c, x, z);} \\
& Compares the components of the \id{N\_Vector} \id{x} to the scalar
\id{c} and returns an \id{N\_Vector} \id{z} such that:
$z_i = 1.0$ if $| x_i | \ge c$ and $z_i = 0.0$ otherwise.
\\[2mm]
%%
\id{N\_VInvTest} & \id{t = N\_VInvTest(x, z);} \\
& Sets the components of the \id{N\_Vector} \id{z} to be the inverses
of the components of the \id{N\_Vector} \id{x}, with prior testing
for zero values:
$z_i = 1.0 /  x_i  , \: i=0,\ldots,n-1$.
This routine returns \id{TRUE} if all components of \id{x} are
nonzero (successful inversion) and returns \id{FALSE} otherwise.  
\\[2mm]
%%
\id{N\_VConstrMask} & \id{t = N\_VConstrMask(c, x, m);} \\
& Performs the following constraint tests:
$x_i > 0$ if $c_i=2$,
$x_i \ge 0$ if $c_i=1$,
$x_i \le 0$ if $c_i=-1$,
$x_i < 0$ if $c_i=-2$.
There is no constraint on $x_i$ if $c_i=0$.
This routine returns \id{FALSE} if any element failed
the constraint test, \id{TRUE} if all passed.  It also sets a
mask vector \id{m}, with elements equal to $1.0$ where the constraint 
test failed, and $0.0$ where the test passed.
This routine is used only for constraint checking.
\\[2mm]
%%
\id{N\_VMinQuotient} & \id{minq = N\_VMinQuotient(num, denom);} \\
& This routine returns the minimum of the quotients obtained   
by term-wise dividing \id{num}$_i$ by \id{denom}$_i$. 
A zero element in \id{denom} will be skipped. 
If no such quotients are found, then the large value 
\Id{BIG\_REAL} (defined in the header file \id{sundials\_types.h})
is returned. 
\\
%%
\end{supertabular}
\bigskip


%---------------------------------------------------------------------------
\section{The NVECTOR\_SERIAL implementation}\label{ss:nvec_ser}
%% This is a shared SUNDIALS TEX file with a description of the
%% serial nvector implementation
%%

The serial implementation of the {\nvector} module provided with {\sundials},
{\nvecs}, defines the {\em content} field of \id{N\_Vector} to be a structure 
containing the length of the vector, a pointer to the beginning of a contiguous 
data array, and a boolean flag {\em own\_data} which specifies the ownership 
of {\em data}.
%%
\begin{verbatim} 
struct _N_VectorContent_Serial {
  long int length;
  booleantype own_data;
  realtype *data;
};
\end{verbatim}
%%
%%--------------------------------------------
%%
The following five macros are provided to access the content of an {\nvecs}
vector. The suffix \id{\_S} in the names denotes serial version.
%%
\begin{itemize}

\item \ID{NV\_CONTENT\_S}                             
    
  This routine gives access to the contents of the serial
  vector \id{N\_Vector}.
  
  The assignment \id{v\_cont} $=$ \id{NV\_CONTENT\_S(v)} sets           
  \id{v\_cont} to be a pointer to the serial \id{N\_Vector} content  
  structure.                                             
                                                            
  Implementation: 
  
  \verb|#define NV_CONTENT_S(v) ( (N_VectorContent_Serial)(v->content) )|
  
\item \ID{NV\_OWN\_DATA\_S}, \ID{NV\_DATA\_S}, \ID{NV\_LENGTH\_S}


  These macros give individual access to the parts of    
  the content of a serial \id{N\_Vector}.                        
                                                               
  The assignment \id{v\_data = NV\_DATA\_S(v)} sets \id{v\_data} to be     
  a pointer to the first component of the data for the \id{N\_Vector} \id{v}. 
  The assignment \id{NV\_DATA\_S(v) = v\_data} sets the component array of \id{v} to     
  be \id{v\_data} by storing the pointer \id{v\_data}.                   
  
  The assignment \id{v\_len = NV\_LENGTH\_S(v)} sets \id{v\_len} to be     
  the length of \id{v}. On the other hand, the call \id{NV\_LENGTH\_S(v) = len\_v} 
  sets the length of \id{v} to be \id{len\_v}.
                                                            
  Implementation: 
  
  \verb|#define NV_OWN_DATA_S(v) ( NV_CONTENT_S(v)->own_data )|

  \verb|#define NV_DATA_S(v) ( NV_CONTENT_S(v)->data )|
  
  \verb|#define NV_LENGTH_S(v) ( NV_CONTENT_S(v)->length )|

\item \ID{NV\_Ith\_S}                                               
                                                            
  This macro gives access to the individual components of the data
  array of an \id{N\_Vector}.

  The assignment \id{r = NV\_Ith\_S(v,i)} sets \id{r} to be the value of 
  the \id{i}-th component of \id{v}. The assignment \id{NV\_Ith\_S(v,i) = r}   
  sets the value of the \id{i}-th component of \id{v} to be \id{r}.        
  
  Here $i$ ranges from $0$ to $n-1$ for a vector of length $n$.

  Implementation:

  \verb|#define NV_Ith_S(v,i) ( NV_DATA_S(v)[i] )|

\end{itemize}
%%
%%----------------------------------------------
%%
The {\nvecs} module defines serial implementations of all vector operations listed 
in Table \ref{t:nvecops}. Their names are obtained from those in Table \ref{t:nvecops} by
appending the suffix \id{\_Serial}. The module {\nvecs} provides the following additional
user-callable routines:
%%
\begin{itemize}

%%--------------------------------------

\item \ID{N\_VNew\_Serial}

  This function creates and allocates memory for a serial \id{N\_Vector}.
  Its only argument is the vector length.

  

  \verb|N_Vector N_VNew_Serial(long int vec_length);|

%%--------------------------------------

\item \ID{N\_VNewEmpty\_Serial}

  This function creates a new serial \id{N\_Vector} with an empty (\id{NULL}) data array.

  

  \verb|N_Vector N_VNewEmpty_Serial(long int vec_length);|

%%--------------------------------------

\item \ID{N\_VMake\_Serial}

 This function creates and allocates memory for a serial vector
 with user-provided data array.

 

 \verb|N_Vector N_VMake_Serial(long int vec_length, realtype *v_data);|

%%--------------------------------------

\item \ID{N\_VCloneVectorArray\_Serial}

 This function creates (by cloning) an array of \id{count} serial vectors.

 

 \verb|N_Vector *N_VCloneVectorArray_Serial(int count, N_Vector w);|

%%--------------------------------------

\item \ID{N\_VCloneEmptyVectorArray\_Serial}

 This function creates (by cloning) an array of \id{count} serial vectors, each with an
 empty (\id{NULL}) data array.

 

 \verb|N_Vector *N_VCloneEmptyVectorArray_Serial(int count, N_Vector w);|

%%--------------------------------------

\item \ID{N\_VDestroyVectorArray\_Serial}

 This function frees memory allocated for the array of \id{count} variables of type
 \id{N\_Vector} created with \id{N\_VCloneVectorArray\_Serial} or with
 \id{N\_VCloneEmptyVectorArray\_Serial}.

 

 \verb|void N_VDestroyVectorArray_Serial(N_Vector *vs, int count);|

%%--------------------------------------

\item \ID{N\_VPrint\_Serial}

 This function prints the content of a serial vector to \id{stdout}.

 
 
 \verb|void N_VPrint_Serial(N_Vector v);|

\end{itemize}
%%
%%------------------------------------
%%
\paragraph{\bf Notes}                                                      
           
\begin{itemize}
                                        
\item
  When looping over the components of an \id{N\_Vector} \id{v}, it is     
  more efficient to first obtain the component array via       
  \id{v\_data = NV\_DATA\_S(v)} and then access \id{v\_data[i]} within the     
  loop than it is to use \id{NV\_Ith\_S(v,i)} within the loop.        

\item
  {\warn}\id{N\_VNewEmpty\_Serial}, \id{N\_VMake\_Serial}, 
  and \id{N\_VCloneEmptyVectorArray\_Serial} set the field 
  {\em own\_data} $=$ \id{FALSE}. 
  \id{N\_VDestroy\_Serial} and \id{N\_VDestroyVectorArray\_Serial}
  will not attempt to free the pointer {\em data} for any \id{N\_Vector} with
  {\em own\_data} set to \id{FALSE}. In such a case, it is the user's responsibility to
  deallocate the {\em data} pointer.
                                     
\item
  {\warn}To maximize efficiency, vector operations in the {\nvecs} implementation
  that have more than one \id{N\_Vector} argument do not check for
  consistent internal representation of these vectors. It is the user's 
  responsibility to ensure that such routines are called with \id{N\_Vector}
  arguments that were all created with the same internal representations.

\end{itemize}


%---------------------------------------------------------------------------
\section{The NVECTOR\_PARALLEL implementation}\label{ss:nvec_par}
% This is a shared SUNDIALS TEX file with description of
% the MPI parallel nvector implementation
%
The {\nvecp} implementation of the {\nvector} module provided with
{\sundials} is based on {\mpi}.  It defines the {\em content}
field of \id{N\_Vector} to be a structure containing the global and local lengths 
of the vector, a pointer to the beginning of a contiguous local data array,
an {\mpi} communicator, an a boolean flag {\em own\_data} indicating ownership of 
the data array {\em data}.
%%
\begin{verbatim} 
struct _N_VectorContent_Parallel {
  long int local_length;
  long int global_length;
  booleantype own_data;
  realtype *data;
  MPI_Comm comm;
};
\end{verbatim}
%%
%%--------------------------------------------
%%
The following seven macros are provided to access the content of a {\nvecp}
vector. The suffix \id{\_P} in the names denotes parallel version.
\begin{itemize}

\item 
  \ID{NV\_CONTENT\_P}

  This macro gives access to the contents of the parallel
  vector \id{N\_Vector}.
  
  The assignment \id{v\_cont = NV\_CONTENT\_P(v)} sets       
  \id{v\_cont} to be a pointer to the \id{N\_Vector} content    
  structure of type \id{struct \_N\_VectorParallelContent}.
  
  Implementation:
  
  \verb|#define NV_CONTENT_P(v) ( (N_VectorContent_Parallel)(v->content) )|
  
\item 
  \ID{NV\_OWN\_DATA\_P}, \ID{NV\_DATA\_P}, 
  \ID{NV\_LOCLENGTH\_P}, \ID{NV\_GLOBLENGTH\_P}
  
  These macros give individual access to the parts of    
  the content of a parallel \id{N\_Vector}.                        
  
  The assignment \id{v\_data = NV\_DATA\_P(v)} sets \id{v\_data} to be     
  a pointer to the first component of the local data for the \id{N\_Vector} \id{v}. 
  The assignment \id{NV\_DATA\_P(v) = v\_data} sets the component array of 
  \id{v} to be \id{v\_data} by storing the pointer \id{v\_data}.                   
  
  The assignment \id{v\_llen = NV\_LOCLENGTH\_P(v)} sets \id{v\_llen} to be     
  the length of the local part of \id{v}. 
  The call \id{NV\_LENGTH\_P(v) = llen\_v} sets      
  the local length of \id{v} to be \id{llen\_v}.
  
  The assignment \id{v\_glen = NV\_GLOBLENGTH\_P(v)} sets \id{v\_glen} to  
  be the global length of the vector \id{v}.                    
  The call \id{NV\_GLOBLENGTH\_P(v) = glen\_v} sets the global       
  length of \id{v} to be \id{glen\_v}.
  
  Implementation:
  
  \verb|#define NV_OWN_DATA_P(v)   ( NV_CONTENT_P(v)->own_data )|

  \verb|#define NV_DATA_P(v)       ( NV_CONTENT_P(v)->data )|

  \verb|#define NV_LOCLENGTH_P(v)  ( NV_CONTENT_P(v)->local_length )|

  \verb|#define NV_GLOBLENGTH_P(v) ( NV_CONTENT_P(v)->global_length )|
  
\item \ID{NV\_COMM\_P}

  This macro provides access to the {\mpi} communicator used by the {\nvecp}
  vectors.

  Implementation:

  \verb|#define NV_COMM_P(v) ( NV_CONTENT_P(v)->comm )|

\item \ID{NV\_Ith\_P}

  This macro gives access to the individual components of the local data
  array of an \id{N\_Vector}.

  The assignment \id{r = NV\_Ith\_P(v,i)} sets \id{r} to be the value of 
  the \id{i}-th component of the local part of \id{v}. 
  The assignment \id{NV\_Ith\_P(v,i) = r}   
  sets the value of the \id{i}-th component of the local part of \id{v} 
  to be \id{r}.        
  
  Here $i$ ranges from $0$ to $n-1$, where $n$ is the local length.
      
  Implementation:

  \verb|#define NV_Ith_P(v,i) ( NV_DATA_P(v)[i] )|

\end{itemize}
%%
%%--------------------------------------------
%%
The {\nvecp} module defines parallel implementations of all vector operations listed 
in Table \ref{t:nvecops}  Their names are obtained from those in Table \ref{t:nvecops} by
appending the suffix \id{\_Parallel}. The module {\nvecp} provides the following additional
user-callable routines:
%%
%%
\begin{itemize}

%%--------------------------------------

\item  \ID{N\_VNew\_Parallel}
  
  This function creates and allocates memory for a parallel vector.
 
  

\begin{verbatim}
N_Vector N_VNew_Parallel(MPI_Comm comm, 
                         long int local_length, 
                         long int global_length);
\end{verbatim}
  
%%--------------------------------------

\item \ID{N\_VNewEmpty\_Parallel}
 
  This function creates a new parallel \id{N\_Vector} with an empty (\id{NULL}) data array.
 
  

\begin{verbatim}
N_Vector N_VNewEmpty_Parallel(MPI_Comm comm, 
                              long int local_length, 
                              long int global_length);
\end{verbatim}

  
%%--------------------------------------

\item \ID{N\_VMake\_Parallel}
  
  This function creates and allocates memory for a parallel vector
  with user-provided data array.
 
  

\begin{verbatim}
N_Vector N_VMake_Parallel(MPI_Comm comm, 
                          long int local_length,
                          long int global_length,
                          realtype *v_data);
\end{verbatim}

%%--------------------------------------

\item \ID{N\_VCloneVectorArray\_Parallel}
 
  This function creates (by cloning) an array of \id{count} parallel vectors.
 
\begin{verbatim}
N_Vector *N_VCloneVectorArray_Parallel(int count, N_Vector w);
\end{verbatim}

%%--------------------------------------

\item \ID{N\_VCloneEmptyVectorArray\_Parallel}
 
  This function creates (by cloning) an array of \id{count} parallel vectors,
  each with an empty (\id{NULL}) data array.
 
\begin{verbatim}
N_Vector *N_VCloneEmptyVectorArray_Parallel(int count, N_Vector w);
\end{verbatim}

%%--------------------------------------

\item \ID{N\_VDestroyVectorArray\_Parallel}
 
 This function frees memory allocated for the array of \id{count}  variables of
 type \id{N\_Vector} created with \id{N\_VCloneVectorArray\_Parallel} or with
 \id{N\_VCloneEmptyVectorArray\_Parallel}.
 

 \verb|void N_VDestroyVectorArray_Parallel(N_Vector *vs, int count);|


%%--------------------------------------

\item \ID{N\_VPrint\_Parallel}
  
  This function prints the content of a parallel vector to stdout.
 
    
  \verb|void N_VPrint_Parallel(N_Vector v);|


\end{itemize}
%%
%%------------------------------------
%%
\paragraph{\bf Notes} 
           
\begin{itemize}
                                        
\item
  When looping over the components of an \id{N\_Vector} \id{v}, it is     
  more efficient to first obtain the local component array via       
  \id{v\_data = NV\_DATA\_P(v)} and then access \id{v\_data[i]} within the     
  loop than it is to use \id{NV\_Ith\_P(v,i)} within the loop.        
                                                               
\item
  {\warn}\id{N\_VNewEmpty\_Parallel}, \id{N\_VMake\_Parallel}, 
  and \id{N\_VCloneEmptyVectorArray\_Parallel} set the field 
  {\em own\_data} $=$ \id{FALSE}. 
  \id{N\_VDestroy\_Parallel} and \id{N\_VDestroyVectorArray\_Parallel}
  will not attempt to free the pointer {\em data} for any \id{N\_Vector} with
  {\em own\_data} set to \id{FALSE}. In such a case, it is the user's responsibility to
  deallocate the {\em data} pointer.

\item
  {\warn}To maximize efficiency, vector operations in the {\nvecp} implementation
  that have more than one \id{N\_Vector} argument do not check for
  consistent internal representation of these vectors. It is the user's 
  responsibility to ensure that such routines are called with \id{N\_Vector}
  arguments that were all created with the same internal representations.

\end{itemize}



%---------------------------------------------------------------------------
\section{The NVECTOR\_OPENMP implementation}\label{ss:nvec_openmp}
%% This is a shared SUNDIALS TEX file with a description of the
%% OpenMP nvector implementation
%%

The OpenMP implementation of the {\nvector} module provided with {\sundials},
{\nvecopenmp}, defines the {\em content} field of \id{N\_Vector} to be a structure 
containing the length of the vector, a pointer to the beginning of a contiguous 
data array, and a boolean flag {\em own\_data} which specifies the ownership 
of {\em data}.  Operations on the vector are threaded using OpenMP, 
the number of threads used is based on the supplied argument in 
the vector constructor.
%%
\begin{verbatim} 
struct _N_VectorContent_OpenMP {
  long int length;
  booleantype own_data;
  realtype *data;
  int num_threads;
};
\end{verbatim}
%%
%%--------------------------------------------
%%
The following six macros are provided to access the content of an {\nvecopenmp}
vector. The suffix \id{\_OMP} in the names denotes OpenMP version.
%%
\begin{itemize}

\item \ID{NV\_CONTENT\_OMP}                             
    
  This routine gives access to the contents of the OpenMP
  vector \id{N\_Vector}.
  
  The assignment \id{v\_cont} $=$ \id{NV\_CONTENT\_OMP(v)} sets           
  \id{v\_cont} to be a pointer to the OpenMP \id{N\_Vector} content  
  structure.                                             
                                                            
  Implementation: 
  
  \verb|#define NV_CONTENT_OMP(v) ( (N_VectorContent_OpenMP)(v->content) )|
  
\item \ID{NV\_OWN\_DATA\_OMP}, \ID{NV\_DATA\_OMP}, \ID{NV\_LENGTH\_OMP}, \ID{NV\_NUM\_THREADS\_OMP}


  These macros give individual access to the parts of    
  the content of a OpenMP \id{N\_Vector}.                        
                                                               
  The assignment \id{v\_data = NV\_DATA\_OMP(v)} sets \id{v\_data} to be     
  a pointer to the first component of the data for the \id{N\_Vector} \id{v}. 
  The assignment \id{NV\_DATA\_OMP(v) = v\_data} sets the component array of \id{v} to     
  be \id{v\_data} by storing the pointer \id{v\_data}.                   
  
  The assignment \id{v\_len = NV\_LENGTH\_OMP(v)} sets \id{v\_len} to be     
  the length of \id{v}. On the other hand, the call \id{NV\_LENGTH\_OMP(v) = len\_v} 
  sets the length of \id{v} to be \id{len\_v}.
                                                            
  The assignment \id{v\_num\_threads = NV\_NUM\_THREADS\_OMP(v)} sets \id{v\_num\_threads} to be     
  the number of threads from \id{v}. On the other hand, the call \id{NV\_NUM\_THREADS\_OMP(v) = num\_threads\_v} 
  sets the number of threads for \id{v} to be \id{num\_threads\_v}.
                                                            
  Implementation: 
  
  \verb|#define NV_OWN_DATA_OMP(v) ( NV_CONTENT_OMP(v)->own_data )|

  \verb|#define NV_DATA_OMP(v) ( NV_CONTENT_OMP(v)->data )|
  
  \verb|#define NV_LENGTH_OMP(v) ( NV_CONTENT_OMP(v)->length )|

  \verb|#define NV_NUM_THREADS_OMP(v) ( NV_CONTENT_OMP(v)->num_threads )|

\item \ID{NV\_Ith\_OMP}                                               
                                                            
  This macro gives access to the individual components of the data
  array of an \id{N\_Vector}.

  The assignment \id{r = NV\_Ith\_OMP(v,i)} sets \id{r} to be the value of 
  the \id{i}-th component of \id{v}. The assignment \id{NV\_Ith\_OMP(v,i) = r}   
  sets the value of the \id{i}-th component of \id{v} to be \id{r}.        
  
  Here $i$ ranges from $0$ to $n-1$ for a vector of length $n$.

  Implementation:

  \verb|#define NV_Ith_OMP(v,i) ( NV_DATA_OMP(v)[i] )|

\end{itemize}
%%
%%----------------------------------------------
%%
The {\nvecopenmp} module defines OpenMP implementations of all vector operations listed 
in Table \ref{t:nvecops}. Their names are obtained from those in Table \ref{t:nvecops} by
appending the suffix \id{\_OpenMP}. The module {\nvecopenmp} provides the following additional
user-callable routines:
%%
\begin{itemize}

%%--------------------------------------

\item \ID{N\_VNew\_OpenMP}

  This function creates and allocates memory for a OpenMP \id{N\_Vector}.
  Arguments are the vector length and number of threads.

  \verb|N_Vector N_VNew_OpenMP(long int vec_length, int num_threads);|

%%--------------------------------------

\item \ID{N\_VNewEmpty\_OpenMP}

  This function creates a new OpenMP \id{N\_Vector} with an empty (\id{NULL}) data array.

  

  \verb|N_Vector N_VNewEmpty_OpenMP(long int vec_length, int num_threads);|

%%--------------------------------------

\item \ID{N\_VMake\_OpenMP}

 This function creates and allocates memory for a OpenMP vector
 with user-provided data array.

 

 \verb|N_Vector N_VMake_OpenMP(long int vec_length, realtype *v_data, int num_threads);|

%%--------------------------------------

\item \ID{N\_VCloneVectorArray\_OpenMP}

 This function creates (by cloning) an array of \id{count} OpenMP vectors.

 

 \verb|N_Vector *N_VCloneVectorArray_OpenMP(int count, N_Vector w);|

%%--------------------------------------

\item \ID{N\_VCloneEmptyVectorArray\_OpenMP}

 This function creates (by cloning) an array of \id{count} OpenMP vectors, each with an
 empty (\id{NULL}) data array.

 

 \verb|N_Vector *N_VCloneEmptyVectorArray_OpenMP(int count, N_Vector w);|

%%--------------------------------------

\item \ID{N\_VDestroyVectorArray\_OpenMP}

 This function frees memory allocated for the array of \id{count} variables of type
 \id{N\_Vector} created with \id{N\_VCloneVectorArray\_OpenMP} or with
 \id{N\_VCloneEmptyVectorArray\_OpenMP}.

 

 \verb|void N_VDestroyVectorArray_OpenMP(N_Vector *vs, int count);|

%%--------------------------------------

\item \ID{N\_VPrint\_OpenMP}

 This function prints the content of a OpenMP vector to \id{stdout}.

 
 
 \verb|void N_VPrint_OpenMP(N_Vector v);|

\end{itemize}
%%
%%------------------------------------
%%
\paragraph{\bf Notes}                                                      
           
\begin{itemize}
                                        
\item
  When looping over the components of an \id{N\_Vector} \id{v}, it is     
  more efficient to first obtain the component array via       
  \id{v\_data = NV\_DATA\_OMP(v)} and then access \id{v\_data[i]} within the     
  loop than it is to use \id{NV\_Ith\_OMP(v,i)} within the loop.        

\item
  {\warn}\id{N\_VNewEmpty\_OpenMP}, \id{N\_VMake\_OpenMP}, 
  and \id{N\_VCloneEmptyVectorArray\_OpenMP} set the field 
  {\em own\_data} $=$ \id{FALSE}. 
  \id{N\_VDestroy\_OpenMP} and \id{N\_VDestroyVectorArray\_OpenMP}
  will not attempt to free the pointer {\em data} for any \id{N\_Vector} with
  {\em own\_data} set to \id{FALSE}. In such a case, it is the user's responsibility to
  deallocate the {\em data} pointer.
                                     
\item
  {\warn}To maximize efficiency, vector operations in the {\nvecopenmp} implementation
  that have more than one \id{N\_Vector} argument do not check for
  consistent internal representation of these vectors. It is the user's 
  responsibility to ensure that such routines are called with \id{N\_Vector}
  arguments that were all created with the same internal representations.

\end{itemize}


%---------------------------------------------------------------------------
\section{The NVECTOR\_PTHREADS implementation}\label{ss:nvec_pthreads}
%% This is a shared SUNDIALS TEX file with a description of the
%% Pthreads nvector implementation
%%

The Pthreads implementation of the {\nvector} module provided with {\sundials},
{\nvecpthreads}, defines the {\em content} field of \id{N\_Vector} to be a structure 
containing the length of the vector, a pointer to the beginning of a contiguous 
data array, and a boolean flag {\em own\_data} which specifies the ownership 
of {\em data}.  Operations on the vector are threaded using POSIX threads 
(Pthreads), the number of threads used is based on the supplied argument in 
the vector constructor.
%%
\begin{verbatim} 
struct _N_VectorContent_Pthreads {
  long int length;
  booleantype own_data;
  realtype *data;
  int num_threads;
};
\end{verbatim}
%%
%%--------------------------------------------
%%
The following five macros are provided to access the content of an {\nvecpthreads}
vector. The suffix \id{\_PT} in the names denotes Pthreads version.
%%
\begin{itemize}

\item \ID{NV\_CONTENT\_PT}                             
    
  This routine gives access to the contents of the Pthreads
  vector \id{N\_Vector}.
  
  The assignment \id{v\_cont} $=$ \id{NV\_CONTENT\_PT(v)} sets           
  \id{v\_cont} to be a pointer to the Pthreads \id{N\_Vector} content  
  structure.                                             
                                                            
  Implementation: 
  
  \verb|#define NV_CONTENT_PT(v) ( (N_VectorContent_Pthreads)(v->content) )|
  
\item \ID{NV\_OWN\_DATA\_PT}, \ID{NV\_DATA\_PT}, \ID{NV\_LENGTH\_PT}


  These macros give individual access to the parts of    
  the content of a Pthreads \id{N\_Vector}.                        
                                                               
  The assignment \id{v\_data = NV\_DATA\_PT(v)} sets \id{v\_data} to be     
  a pointer to the first component of the data for the \id{N\_Vector} \id{v}. 
  The assignment \id{NV\_DATA\_PT(v) = v\_data} sets the component array of \id{v} to     
  be \id{v\_data} by storing the pointer \id{v\_data}.                   
  
  The assignment \id{v\_len = NV\_LENGTH\_PT(v)} sets \id{v\_len} to be     
  the length of \id{v}. On the other hand, the call \id{NV\_LENGTH\_PT(v) = len\_v} 
  sets the length of \id{v} to be \id{len\_v}.
                                                            
  Implementation: 
  
  \verb|#define NV_OWN_DATA_PT(v) ( NV_CONTENT_PT(v)->own_data )|

  \verb|#define NV_DATA_PT(v) ( NV_CONTENT_PT(v)->data )|
  
  \verb|#define NV_LENGTH_PT(v) ( NV_CONTENT_PT(v)->length )|

  \verb|#define NV_NUM_THREADS_PT(v) ( NV_CONTENT_PT(v)->num_threads )|

\item \ID{NV\_Ith\_PT}                                               
                                                            
  This macro gives access to the individual components of the data
  array of an \id{N\_Vector}.

  The assignment \id{r = NV\_Ith\_PT(v,i)} sets \id{r} to be the value of 
  the \id{i}-th component of \id{v}. The assignment \id{NV\_Ith\_PT(v,i) = r}   
  sets the value of the \id{i}-th component of \id{v} to be \id{r}.        
  
  Here $i$ ranges from $0$ to $n-1$ for a vector of length $n$.

  Implementation:

  \verb|#define NV_Ith_PT(v,i) ( NV_DATA_PT(v)[i] )|

\end{itemize}
%%
%%----------------------------------------------
%%
The {\nvecpthreads} module defines Pthreads implementations of all vector operations listed 
in Table \ref{t:nvecops}. Their names are obtained from those in Table \ref{t:nvecops} by
appending the suffix \id{\_Pthreads}. The module {\nvecpthreads} provides the following additional
user-callable routines:
%%
\begin{itemize}

%%--------------------------------------

\item \ID{N\_VNew\_Pthreads}

  This function creates and allocates memory for a Pthreads \id{N\_Vector}.
  Arguments are the vector length and number of threads.

  

  \verb|N_Vector N_VNew_Pthreads(long int vec_length, int num_threads);|

%%--------------------------------------

\item \ID{N\_VNewEmpty\_Pthreads}

  This function creates a new Pthreads \id{N\_Vector} with an empty (\id{NULL}) data array.

  

  \verb|N_Vector N_VNewEmpty_Pthreads(long int vec_length, int num_threads);|

%%--------------------------------------

\item \ID{N\_VMake\_Pthreads}

 This function creates and allocates memory for a Pthreads vector
 with user-provided data array.

 

 \verb|N_Vector N_VMake_Pthreads(long int vec_length, realtype *v_data, int num_threads);|

%%--------------------------------------

\item \ID{N\_VCloneVectorArray\_Pthreads}

 This function creates (by cloning) an array of \id{count} Pthreads vectors.

 

 \verb|N_Vector *N_VCloneVectorArray_Pthreads(int count, N_Vector w);|

%%--------------------------------------

\item \ID{N\_VCloneEmptyVectorArray\_Pthreads}

 This function creates (by cloning) an array of \id{count} Pthreads vectors, each with an
 empty (\id{NULL}) data array.

 

 \verb|N_Vector *N_VCloneEmptyVectorArray_Pthreads(int count, N_Vector w);|

%%--------------------------------------

\item \ID{N\_VDestroyVectorArray\_Pthreads}

 This function frees memory allocated for the array of \id{count} variables of type
 \id{N\_Vector} created with \id{N\_VCloneVectorArray\_Pthreads} or with
 \id{N\_VCloneEmptyVectorArray\_Pthreads}.

 

 \verb|void N_VDestroyVectorArray_Pthreads(N_Vector *vs, int count);|

%%--------------------------------------

\item \ID{N\_VPrint\_Pthreads}

 This function prints the content of a Pthreads vector to \id{stdout}.

 
 
 \verb|void N_VPrint_Pthreads(N_Vector v);|

\end{itemize}
%%
%%------------------------------------
%%
\paragraph{\bf Notes}                                                      
           
\begin{itemize}
                                        
\item
  When looping over the components of an \id{N\_Vector} \id{v}, it is     
  more efficient to first obtain the component array via       
  \id{v\_data = NV\_DATA\_PT(v)} and then access \id{v\_data[i]} within the     
  loop than it is to use \id{NV\_Ith\_PT(v,i)} within the loop.        

\item
  {\warn}\id{N\_VNewEmpty\_Pthreads}, \id{N\_VMake\_Pthreads}, 
  and \id{N\_VCloneEmptyVectorArray\_Pthreads} set the field 
  {\em own\_data} $=$ \id{FALSE}. 
  \id{N\_VDestroy\_Pthreads} and \id{N\_VDestroyVectorArray\_Pthreads}
  will not attempt to free the pointer {\em data} for any \id{N\_Vector} with
  {\em own\_data} set to \id{FALSE}. In such a case, it is the user's responsibility to
  deallocate the {\em data} pointer.
                                     
\item
  {\warn}To maximize efficiency, vector operations in the {\nvecpthreads} implementation
  that have more than one \id{N\_Vector} argument do not check for
  consistent internal representation of these vectors. It is the user's 
  responsibility to ensure that such routines are called with \id{N\_Vector}
  arguments that were all created with the same internal representations.

\end{itemize}


%---------------------------------------------------------------------------
\section{NVECTOR functions used by KINSOL}

In Table \ref{t:nvecuse} below, we list the vector functions in the 
{\nvector} module within the {\kinsol} package.
The table also shows, for each function, which of the code modules uses
the function. The {\kinsol} column shows function usage within the main
solver module, while the remaining five columns show function
usage within each of the four {\kinsol} linear solvers ({\kinspils}
stands for any of {\kinspgmr}, {\kinspbcg}, or {\kinsptfqmr}),
the {\kinbbdpre} preconditioner module, and the {\fkinsol} module.

There is one subtlety in the {\kinspils} column hidden by the table, explained
here for the case of the {\kinspgmr} module. 
The \id{N\_VDotProd} function is called both within the interface file
\id{kinsol\_spgmr.c} and within the implementation
files \id{sundials\_spgmr.c} and \id{sundials\_iterative.c} for the generic
{\spgmr} solver upon which the {\kinspgmr} solver is built.  Also, although
\id{N\_VDiv} and \id{N\_VProd} are not called within the interface file
\id{kinsol\_spgmr.c}, they are called within the implementation file
\id{sundials\_spgmr.c}, and so are required by the {\kinspgmr} solver module.
Analogous statements apply to the {\kinspbcg} and {\kinsptfqmr} modules,
except that they do not use \id{sundials\_iterative.c}.
This issue does not arise for the direct {\kinsol} linear solvers because
the generic {\dense} and {\band} solvers (used in the implementation of
{\kindense} and {\kinband}) do not make calls to any vector functions.

At this point, we should emphasize that the {\kinsol} user does not need to know 
anything about the usage of vector functions by the {\kinsol} code modules in order 
to use {\kinsol}. The information is presented as an implementation detail for the 
interested reader.

\begin{table}[htb]
\centering
\caption{List of vector functions usage by KINSOL code modules}\label{t:nvecuse}
\medskip
\begin{tabular}{|r|c|c|c|c|c|c|} \hline
                                            &
\begin{sideways}{\kinsol}    \end{sideways} &
\begin{sideways}{\kindense}  \end{sideways} &
\begin{sideways}{\kinband}   \end{sideways} &
\begin{sideways}{\kinspils}  \end{sideways} &
\begin{sideways}{\kinbbdpre} \end{sideways} &
\begin{sideways}{\fkinsol}   \end{sideways} 
\\ \hline\hline
%                        KINSOL DENSE BAND  SPILS BPRE FKINSOL
\id{N\_VClone}           & \cm &     &     & \cm & \cm &     \\ \hline
\id{N\_VCloneEmpty}      &     &     &     &     &     & \cm \\ \hline
\id{N\_VDestroy}         & \cm &     &     & \cm & \cm & \cm \\ \hline
\id{N\_VSpace}           & \cm &     &     &     &     &     \\ \hline
\id{N\_VGetArrayPointer} &     & \cm & \cm &     & \cm & \cm \\ \hline
\id{N\_VSetArrayPointer} &     & \cm &     &     &     & \cm \\ \hline
\id{N\_VLinearSum}       & \cm & \cm &     & \cm &     &     \\ \hline
\id{N\_VConst}           &     &     &     & \cm &     &     \\ \hline
\id{N\_VProd}            & \cm & \cm & \cm & \cm &     &     \\ \hline
\id{N\_VDiv}             & \cm &     &     & \cm &     &     \\ \hline
\id{N\_VScale}           & \cm & \cm & \cm & \cm & \cm &     \\ \hline
\id{N\_VAbs}             & \cm &     &     &     &     &     \\ \hline
\id{N\_VInv}             & \cm &     &     &     &     &     \\ \hline
\id{N\_VDotProd}         &     & \cm & \cm & \cm &     &     \\ \hline
\id{N\_VMaxNorm}         & \cm &     &     &     &     &     \\ \hline
\id{N\_VMin}             & \cm &     &     &     &     &     \\ \hline
\id{N\_VWL2Norm}         & \cm & \cm & \cm & \cm &     &     \\ \hline
\id{N\_VL1Norm}          &     &     &     & \cm &     &     \\ \hline
\id{N\_VConstrMask}      & \cm &     &     &     &     &     \\ \hline
\id{N\_VMinQuotient}     & \cm &     &     &     &     &     \\ \hline
\end{tabular}
\end{table}

The vector functions listed in Table \ref{t:nvecops} that are {\em not} used by
{\kinsol} are: \id{N\_VAddConst}, \id{N\_VWrmsNorm}, \id{N\_VWrmsNormMask},
\id{N\_VCompare}, and \id{N\_VInvTest}.
Therefore a user-supplied {\nvector} module for {\kinsol} could omit these five
functions.
